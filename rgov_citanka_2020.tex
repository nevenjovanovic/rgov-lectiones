\documentclass[a4paper,12pt,twoside]{report}
\usepackage[quiet]{polyglossia}
\setdefaultlanguage{latin}
\setotherlanguage[variant=ancient]{greek}
\setotherlanguage{croatian}

%\usepackage{fontspec}
\usepackage{verse}
\usepackage{metalogo}

\usepackage{textcomp}
\defaultfontfeatures{Ligatures=TeX}

\usepackage{import}
\usepackage[small,sf,bf]{titlesec}
\usepackage{tabto}
\usepackage{ulem}
\usepackage{hyperref}
\usepackage{enumitem}

\usepackage{fancyhdr}
\renewcommand{\chaptermark}[1]{\markboth{#1}{}}
\renewcommand{\sectionmark}[1]{\markright{#1}}
\pagestyle{fancy}
\fancyhf{}
\fancyhead[LE,RO]{\thepage}
\fancyhead[RE]{\itshape\nouppercase{Rimsko govorništvo – čitanka}}
\fancyhead[LO]{\itshape\nouppercase{\leftmark}}
\renewcommand{\headrulewidth}{0pt}

\usepackage{titling}
\newcommand{\subtitle}[1]{%
  \posttitle{%
    \par\end{center}
    \begin{center}\large#1\end{center}
    \vskip0.5em}%
}
 
\setmainfont{Old Standard TT}
\setsansfont{Old Standard TT}
%\setsansfont{Tahoma}

\hyphenation{δυσ-σέ-βει-αν βού-λεσ-θαί κα-τη-γο-ρού-σης τα-χέ-ως πε-πλημ-μέ-λη-κε νε-α-νί-ας αὐ-τῇ ἀ-φε-λό-με-νος ἀ-πή-γε-το ἐ-πέ-πληξ-άς ἠ-γό-μην ἤ-νεγ-κεν νο-μί-ζει ἄν-θρω-πον παν-το-δα-πά Αἰ-θί-οψ-ιν ἀν-έρ-χε-ται φρον-τί-δων αὐ-τὸς δι-ῃ-ρη-μέ-νος κλέπ-τον-τας ἑ-κα-τόμ-βῃ μέ-γισ-τον κιν-δύ-νου}

\hyphenation{Ελ-λή-νων ξυν-έ-μει-νεν ἐ-πι-όντ-ων ἀνα-σκευ-α-σά-με-νοι ἀ-πω-σά-με-νοι Λα-κε-δαι-μό-νι-οι Ελ-λη-νες δι-ε-φά-νη}

\hyphenation{εὐ-δο-κι-μή-σας ἐν-ταῦ-θα βα-σι-λεύ-εις ἐ-χρή-σα-το συλ-ληφ-θέν-τος}

\hyphenation{θε-ρά-πον-τες ἡ-γοῦ-μαι}

\hyphenation{ἐπι-φα-νέσ-τε-ρον το-σοῦ-τον ἔ-χον-τας συγ-γιγ-νο-μέ-νους μᾶλ-λον με-γίσ-την}

\hyphenation{ἐπ-έσ-κηπ-τε}

\hyphenation{κοι-νω-νοῦ-σιν δη-λῶ-σαι δι-α-λε-γο-μέ-νους πλη-σι-ά-ζον-τας πα-ρα-λι-πεῖν}

\hyphenation{καρ-πῶν ὀ-νο-μα-ζό-με-νον τηκ-τὰ ὅ-σα}

\hyphenation{με-μά-θη-κας Μά-λισ-τα πο-λυ-μα-θής}

\hyphenation{δι-α-βε-βλη-μέ-νος Α-ρι-στό-βου-λος Σω-κρά-τους Δı-α-τ
ρι-βαί ἐξ-ελ-κύ-σαι Ἐγ-χει-ρί-δι-ον}

\hyphenation{δά-μα-λιν δύ-να-μαι}

\hyphenation{πυ-θο-μέ-νου ἀ-πο-κρί-νε-ται τρα-πέ-ζαις ἀ-πέσ-τει-λαν γρά-ψαν-τος με-τα-τί-θη-σι με-γά-λου Τισ-σα-φέρ-νῃ προσ-ε-πι-μετ-ρῆ-σαι Α-θη-ναί-οις Τισ-σα-φέρ-νην Αλ-κι-βι-ά-δην πα-τρί-δα}

\hyphenation{ἡ-γοῦν-ται ἀ-πο-ρω-τά-των πό-λεις Ras-pra-va Ἀ-πο-λο-γί-α Λα-κε-δαı-μο-νί-ων Κυ-νη-γε-τι-κός}

\hyphenation{πλε-όν-των γε-ωρ-γὸς ἕ-τε-ρον}

\hyphenation{κα-θεῖ-ναι προ-ελ-θοῦ-σαν ἀ-πο-δοῦ-ναι ἀ-πο-φαί-νον-τος ἀ-πεσ-τά-λη ἅ-παν-τας τρί-πο-δα τρί-πο-δος κα-θι-ερώ-θη}

\hyphenation{χρεί-αν παν-τὸς ἀν-επί-σκεπ-τον ἀ-πο-θα-νόν-των πα-ρα-σκευ-ά-ζον-τας φρον-τί-ζον-τας πολ-λοὺς κτω-μέ-νους ἐ-λατ-τοῦσ-θαι ἀ-με-λοῦν-τας ἀ-θε-ρά-πευ-τον κτη-μά-των πει-ρω-μέ-νους ὀ-λι-γω-ροῦν-τας δε-ο-μέ-νων ἐ-ῶν-τας ἔ-μοι-γε}

\hyphenation{αὐ-το-κρά-το-ρας Λά-μα-χον ποι-ή-σαν-τες ἑξή-κον-τα Σε-λι-νουν-τί-ους χρη-μά-των μισ-θόν Νι-κη-ρά-του πε-ρι-γίγ-νη-ται}



\begin{document}

\title{Rimsko govorništvo\\(164. pr. Kr. – 430. po Kr.)}
\subtitle{Čitanka}
\author{Odsjek za klasičnu filologiju\\
Filozofski fakultet Sveučilišta u Zagrebu}
\date{Listopad 2020.}
\maketitle

\clearpage
\thispagestyle{empty}


\tableofcontents


\chapter*{Predgovor}

%\section*{O ovoj čitanci}

Izbor koji je pred vama donosi odabrane odlomke tekstova važnih za poznavanje i razumijevanje rimskog govorništva i njegove uloge u društvu, od početka rimske književnosti do kasne antike.

Zadatak je studenata da, uz pomoć uobičajene referentne literature, kod kuće prirede svaku cjelinu, tako da budu spremni raspravljati o sadržaju na nastavi.

Čitanka je također dio lektire propisane za ovaj kolegij, uz Ciceronov govor \textit{Pro Milone}.

Izbor je priredio Neven Jovanović, nastavnik Odsjeka za klasičnu filologiju Filozofskog fakulteta Sveučilišta u Zagrebu.

Izvorni kod prijeloma ovog izbora, priređen pomoću programa \LaTeX\ i \XeLaTeX, dostupan je u repozitoriju \url{http://bitbucket.org/nevenjovanovic/romrhet-lectiones} na servisu BitBucket.

%\newpage


\medskip

U Zagrebu, listopada 2020.

\vspace*{\fill}

\noindent Ovo djelo je ustupljeno pod Creative Commons licencom Imenovanje 3.0 nelokalizirana licenca. Da biste vidjeli primjerak te licence, posjetite \url{http://creativecommons.org/licenses/by/3.0/} ili pošaljite pismo na Creative Commons, PO Box 1866, Mountain View, CA 94042, SAD.

%\section*{Bibliografija}


\chapter[Cato]{M.\ Porcius Cato \\(234–149 a.~Chr.~n)}

\import{lectiones/}{c01-cato.tex}

\chapter[C.\ Gracchus]{C.\ Sempronius Gracchus \\(154–121 a.~Chr.~n)}

\import{lectiones/}{c02-gracchus.tex}

\chapter[Cicero]{M.\ Tullius Cicero \\(106-43 a.~Chr.~n)}

\import{lectiones/}{c04a-cicero-1.tex}

\import{lectiones/}{c04a-cicero-2.tex}

\import{lectiones/}{c04a-cicero-3.tex}

\chapter[Rhetorica ad Herennium]{Rhetorica ad Herennium \\(90-80 a.~Chr.~n)}

\import{lectiones/}{c04-rhetherr.tex}

\chapter[Ciceronis De oratore]{Ciceronis De oratore \\(55 a.~Chr.~n)}

\import{lectiones/}{c04b-cicdeor.tex}

%\clearpage
%\thispagestyle{empty}
% kraj

%\backmatter


\end{document}

\chapter[Crassus]{L.\ Licinius Crassus \\(140-91 a.~Chr.~n)}

\import{lectiones/}{c03-crassus.tex}




\chapter[Caesar]{C.\ Iulius Caesar \\(100-44 a.~Chr.~n)}

\import{lectiones/}{c05-caesar-bc2-32.tex}

\chapter[Livius]{T.\ Livius \\(64 aut 59 a.~Chr.~n. – 12 aut 17 a.~D)}

\import{lectiones/}{c06-livius.tex}

\chapter[Laudatio Turiae]{Laudatio Turiae \\(ca. 5 a.\ Chr.\ n)}

\import{lectiones/}{c07-laudatioturiae.tex}

\chapter[Seneca maior]{Lucius Annaeus Seneca maior \\(ca.\ 54 a.~Chr.\ n. – ca.\ 39 a.~D)}

\import{lectiones/}{c08-senecamaior.tex}

\chapter[Claudius]{Tiberius Claudius Caesar Augustus Germanicus \\(10 a.~Chr.~n. – 54 a.~D)}

\import{lectiones/}{c08a-claudius.tex}

\chapter[Petronius]{Gaius Petronius Arbiter \\(27–66 a.~D)}

\import{lectiones/}{c08b-petronius.tex}

\chapter[Quintilianus]{M.\ Fabius Quintilianus \\(ca.\ 35 – ca.\ 100 a.~D)}

\import{lectiones/}{c09-quintilianus.tex}

\chapter[Tacitus]{C.\ Cornelius Tacitus \\(ca.\ 55 – 116/120 a.~D)}

\import{lectiones/}{c09a-tacitus.tex}

\chapter[Plinius minor]{C.\ Plinius Secundus \\(61 – ca.\ 113 a.~D)}

\import{lectiones/}{c12-plinius.tex}

\chapter[Fronto]{M.\ Cornelius Fronto \\(90-95 – ca.\ 167 a.~D)}

\import{lectiones/}{c13-fronto.tex}

\chapter[Apuleius]{Apuleius \\(ca.\ 125 – ca.\ 180 a.~D)}

\import{lectiones/}{c14-apuleius.tex}

\chapter[Tertullianus]{Quintus Septimius Florens Tertullianus \\(150-230 a. D)}

\import{lectiones/}{c15-tertullianus.tex}

\chapter[Panegyricus Constantino]{$\langle$Incerti$\rangle$ panegyricus dictus Constantino filio Constantii (313 a.~D)}

\import{lectiones/}{c15a-panegyricus.tex}

\chapter[Ambrosius]{Ambrosius Mediolanensis \\(340-397 a.~D)}

\import{lectiones/}{c16-ambrosius.tex}

\chapter[Symmachus]{Quintus Aurelius Symmachus \\(ca. 340 – post 402 a.~D)}

\import{lectiones/}{c17-symmachus.tex}

\chapter[Augustinus]{Augustinus Hipponensis \\(354-430 a.~D)}

\import{lectiones/}{c20-augustinus.tex}

