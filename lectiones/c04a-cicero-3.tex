%\section*{O autoru}

\section*{Ciceronis in Verrem orationes II, a. 70}

\subsection*{Cic. Verr. 1, 1, 1-3}

quod erat optandum maxime, iudices, et quod unum ad invidiam vestri ordinis infamiamque iudiciorum sedandam maxime pertinebat, id non humano consilio sed prope divinitus datum atque oblatum vobis summo rei publicae tempore videtur. inveteravit enim iam opinio perniciosa rei publicae vobisque periculosa, quae non modo apud nos sed apud exteras nationes omnium sermone percrebruit, his iudiciis quae nunc sunt pecuniosum hominem, quamvis sit nocens, neminem posse damnari.

nunc in ipso discrimine ordinis iudiciorumque vestrorum, cum sint parati qui contionibus et legibus hanc invidiam senatus inflammare conentur, reus in iudicium adductus est C.~Verres, homo vita atque factis omnium iam opinione damnatus, pecuniae magnitudine sua spe et praedicatione absolutus. huic ego causae, iudices, cum summa voluntate et exspectatione populi Romani actor accessi, non ut augerem invidiam ordinis, sed ut infamiae communi succurrerem. adduxi enim hominem in quo reconciliare existimationem iudiciorum amissam, redire in gratiam cum populo Romano, satis facere exteris nationibus possetis, depeculatorem aerari, vexatorem Asiae atque Pamphyliae, praedonem iuris urbani, labem atque perniciem provinciae Siciliae.

de quo si vos severe ac religiose iudicaveritis, auctoritas ea, quae in vobis remanere debet, haerebit; sin istius ingentes divitiae iudiciorum religionem veritatemque perfregerint, ego hoc tamen adsequar, ut iudicium potius rei publicae quam aut reus iudicibus aut accusator reo defuisse videatur.

\subsection*{Cic. Verr. 2, 5, 26–27: Verris itinera}



itinerum primum laborem, qui vel maximus est in re militari, iudices, et in Sicilia maxime necessarius, accipite quam facilem sibi iste et iucundum ratione consilioque reddiderit. primum temporibus hibernis ad magnitudinem frigorum et tempestatum vim ac fluminum praeclarum hoc sibi remedium compararat. Vrbem Syracusas elegerat, cuius hic situs atque haec natura esse loci caelique dicitur ut nullus umquam dies tam magna ac turbulenta tempestate fuerit quin aliquo tempore eius diei solem homines viderint. hic ita vivebat iste bonus imperator hibernis mensibus ut eum non facile non modo extra tectum, sed ne extra lectum quidem quisquam viderit; ita diei brevitas conviviis, noctis longitudo stupris et flagitiis continebatur.

cum autem ver esse coeperat — cuius initium iste non a Favonio neque ab aliquo astro notabat, sed cum rosam viderat, tum incipere ver arbitrabatur — dabat se labori atque itineribus; in quibus eo usque se praebebat patientem atque impigrum ut eum nemo umquam in equo sedentem viderit. nam, ut mos fuit Bithyniae regibus, lectica octaphoro ferebatur, in qua pulvinus erat perlucidus Melitensis rosa fartus; ipse autem coronam habebat unam in capite, alteram in collo, reticulumque ad naris sibi admovebat tenuissimo lino, minutis maculis, plenum rosae. sic confecto itinere cum ad aliquod oppidum venerat, eadem lectica usque in cubiculum deferebatur. eo veniebant Siculorum magistratus, veniebant equites Romani, id quod ex multis iuratis audistis; controversiae secreto deferebantur, paulo post palam decreta auferebantur. deinde ubi paulisper in cubiculo pretio non aequitate iura discripserat, Veneri iam et Libero reliquum tempus deberi arbitrabatur.

\section*{Ciceronis in Catilinam oratio I, a. 63}

\subsection*{Cic. Cat. 1, 32-33: Peroratio magnifica}

qua re secedant improbi, secernant se a bonis, unum in locum congregentur, muro denique, quod saepe iam dixi, secernantur a nobis; desinant insidiari domi suae consuli, circumstare tribunal praetoris urbani, obsidere cum gladiis curiam, malleolos et faces ad inflammandam urbem comparare; sit denique inscriptum in fronte unius cuiusque quid de re publica sentiat. polliceor hoc vobis, patres conscripti, tantam in nobis consulibus fore diligentiam, tantam in vobis auctoritatem, tantam in equitibus Romanis virtutem, tantam in omnibus bonis consensionem ut Catilinae profectione omnia patefacta, inlustrata, oppressa, vindicata esse videatis.

hisce ominibus, Catilina, cum summa rei publicae salute, cum tua peste ac pernicie cumque eorum exitio qui se tecum omni scelere parricidioque iunxerunt, proficiscere ad impium bellum ac nefarium. tu, Iuppiter, qui isdem quibus haec urbs auspiciis a Romulo es constitutus, quem Statorem huius urbis atque imperi vere nominamus, hunc et huius socios a tuis ceterisque templis, a tectis urbis ac moenibus, a vita fortunisque civium omnium arcebis et homines bonorum inimicos, hostis patriae, latrones Italiae scelerum foedere inter se ac nefaria societate coniunctos aeternis suppliciis vivos mortuosque mactabis.

\section*{Ciceronis Philippica II, a. 43}

\subsection*{Cic. Phil. 2, 43-50: Antonii vitia privata et publica}


(\dots) iam enim, quoniam criminibus eius satis respondi, de ipso emendatore et correctore nostro quaedam dicenda sunt. nec enim omnia effundam, ut, si saepius decertandum sit, ut erit, semper novus veniam: quam facultatem mihi multitudo istius vitiorum peccatorumque largitur.

visne igitur te inspiciamus a puero? sic opinor; a principio ordiamur. tenesne memoria praetextatum te decoxisse? 'Patris' inquies 'ista culpa est.' concedo. etenim est pietatis plena defensio. illud tamen audaciae tuae quod sedisti in quattuordecim ordinibus, cum esset lege Roscia decoctoribus certus locus constitutus, quamvis quis fortunae vitio, non suo decoxisset. sumpsisti virilem, quam statim muliebrem togam reddidisti. primo volgare scortum; certa flagiti merces nec ea parva; sed cito Curio intervenit qui te a meretricio quaestu abduxit et, tamquam stolam dedisset, in matrimonio stabili et certo conlocavit.

nemo umquam puer emptus libidinis causa tam fuit in domini potestate quam tu in Curionis. quotiens te pater eius domu sua eiecit, quotiens custodes posuit ne limen intrares? cum tu tamen nocte socia, hortante libidine, cogente mercede, per tegulas demitterere. quae flagitia domus illa diutius ferre non potuit. scisne me de rebus mihi notissimis dicere? recordare tempus illud cum pater Curio maerens iacebat in lecto; filius se ad pedes meos prosternens, lacrimans, te mihi commendabat; orabat ut se contra suum patrem, si sestertium sexagiens peteret, defenderem; tantum enim se pro te intercessisse dicebat. ipse autem amore ardens confirmabat, quod desiderium tui discidi ferre non posset, se in exsilium iturum. 

quo tempore ego quanta mala florentissimae familiae sedavi vel potius sustuli! patri persuasi ut aes alienum fili dissolveret; redimeret adulescentem, summa spe et animi et ingeni praeditum, rei familiaris facultatibus eumque non modo tua familiaritate sed etiam congressione patrio iure et potestate prohiberet. haec tu cum per me acta meminisses, nisi illis quos videmus gladiis confideres, maledictis me provocare ausus esses?

sed iam stupra et flagitia omittamus: sunt quaedam quae honeste non possum dicere; tu autem eo liberior quod ea in te admisisti quae a verecundo inimico audire non posses. sed reliquum vitae cursum videte, quem quidem celeriter perstringam. ad haec enim quae in civili bello, in maximis rei publicae miseriis fecit, et ad ea quae cotidie facit, festinat animus. quae peto ut, quamquam multo notiora vobis quam mihi sunt, tamen, ut facitis, attente audiatis. debet enim talibus in rebus excitare animos non cognitio solum rerum sed etiam recordatio; etsi incidamus, opinor, media ne nimis sero ad extrema veniamus. 

intimus erat in tribunatu Clodio qui sua erga me beneficia commemorat; eius omnium incendiorum fax, cuius etiam domi iam tum quiddam molitus est. quid dicam ipse optime intellegit. Inde iter Alexandream contra senatus auctoritatem, contra rem publicam et religiones; sed habebat ducem Gabinium, quicum quidvis rectissime facere posset. qui tum inde reditus aut qualis? prius in ultimam Galliam ex Aegypto quam domum. quae autem domus? Suam enim quisque domum tum obtinebat nec erat usquam tua. domum dico? quid erat in terris ubi in tuo pedem poneres praeter unum Misenum quod cum sociis tamquam Sisaponem tenebas?

venis e Gallia ad quaesturam petendam. aude dicere te prius ad parentem tuam venisse quam ad me. acceperam iam ante Caesaris litteras ut mihi satis fieri paterer a te: itaque ne loqui quidem sum te passus de gratia. postea sum cultus a te, tu a me observatus in petitione quaesturae; quo quidem tempore P. Clodium approbante populo Romano in foro es conatus occidere, cumque eam rem tua sponte conarere, non impulsu meo, tamen ita praedicabas, te non existimare, nisi illum interfecisses, umquam mihi pro tuis in me iniuriis satis esse facturum. in quo demiror cur Milonem impulsu meo rem illam egisse dicas, cum te ultro mihi idem illud deferentem numquam sim adhortatus. quamquam, si in eo perseverares, ad tuam gloriam rem illam referri malebam quam ad meam gratiam. 

quaestor es factus: deinde continuo sine senatus consulto, sine sorte, sine lege ad Caesarem cucurristi. id enim unum in terris egestatis, aeris alieni, nequitiae perditis vitae rationibus perfugium esse ducebas. ibi te cum et illius largitionibus et tuis rapinis explevisses, si hoc est explere, expilare quod statim effundas, advolasti egens ad tribunatum, ut in eo magistratu, si posses, viri tui similis esses.
