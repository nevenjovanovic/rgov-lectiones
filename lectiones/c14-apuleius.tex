%\section*{O autoru}

\section*{Floridorum liber IV, 18}
	
Tanta multitudo ad audiendum convenistis, ut potius gratulari Carthagini debeam, quod tam multos eruditionis amicos habet, quam excusare, quod philosophus non recusaverim dissertare. Nam et pro amplitudine civitatis frequentia collecta et pro magnitudine frequentiae locus delectus est. Praeterea in auditorio hoc genus spectari debet non pavimenti marmoratio nec proscaenii contabulatio nec scaenae columnatio, sed nec culminum eminentia nec lacunarium refulgentia nec sedilium circumferentia, nec quod hic alias mimus halucinatur, comoedus sermocinatur, tragoedus vociferatur, funerepus periclitatur, praestigiator furatur, histrio gesticulatur ceterique omnes ludiones ostentant populo quod cuiusque artis est, sed istis omnibus supersessis nihil amplius spectari debet quam convenientium ratio et dicentis oratio. 

Quapropter, ut poetae solent hic ibidem varias civitates substituere, ut ille tragicus, qui in theatro dici facit: ``Liber, qui augusta haec loca Cithaeronis colis'', item ille comicus:
\begin{verse}
perparvam partim postulat Plautus loci \\
de vostris magnis atque amoenis moenibus, \\
Athenas quo sine architectis conferat,
\end{verse}
non secus et mihi liceat nullam longinquam et transmarinam civitatem hic, sed enim ipsius Carthaginis vel curiam vel bybliothecam substituere. Igitur proinde habetote, si curia digna protulero, ut si in ipsa curia me audiatis, si erudita fuerint, ut si in bybliotheca legantur. 

Quod utinam mihi pro amplitudine auditorii prolixa oratio suppeteret ac non hic maxime clauderet, ubi me facundissimum cuperem. Sed verum verbum est profecto, qui aiunt nihil quicquam homini tam prosperum divinitus datum, quin ei tamen admixtum sit aliquid difficultatis, ut etiam in amplissima quaque laetitia subsit quaepiam vel parva querimonia, coniugatione quadam mellis et fellis: ubi uber, ibi tuber. Id ego cum [in] alias, tum etiam nunc inpraesentiarum usu experior. Nam quanto videor plura apud vos habere ad commendationem suffragia, tanto sum ad dicendum nimia reverentia vestri cunctatior, et qui penes extrarios saepenumero promptissime disceptavi, idem nunc penes meos haesito ac - mirum dictu - ipsis illecebris deterreor et stimulis refrenor et incitamentis cohibeor. An non multa mihi apud vos adhortamina suppetunt, quod sum vobis nec lare alienus nec pueritia invisitatus nec magistris peregrinus nec secta incognitus nec voce inauditus nec libris inlectus improbatusve? 

Ita mihi et patria in concilio Africae, id est vestro, et pueritia apud vos et magistri vos et secta, licet Athenis Atticis confirmata, tamen hic inchoata est, et vox mea utraque lingua iam vestris auribus ante proxumum sexennium probe cognita; quin et libri mei non alia ubique laude carius censentur quam quod iudicio vestro comprobantur. Haec tanta ac totiuga invitamenta communia non minus vos ad audiendum prolectant quam me ad audendum retardant, faciliusque laudes vestras alibi gentium quam apud vos praedicarim: ita apud suos cuique modestia obnoxia est, apud extrarios autem veritas libera. 

Semper adeo et ubique vos quippe ut parentis ac primos magistros meos celebro mercedemque vobis rependo, non illam, quam Protagora sophista pepigit nec accepit, sed quam Thales sapiens nec pepigit et accepit. Video quid postuletis: utramque narrabo. Protagora, qui sophista fuit longe multiscius et cum primis rhetoricae repertoribus perfacundus, Democriti physici civis aequaevus – inde ei suppeditata doctrina est – eum Protagoran aiunt cum suo sibi discipulo Evathlo mercedem nimis uberem condicione temeraria pepigisse, uti sibi tum demum id argenti daret, si primo tirocinio agendi penes iudices vicisset. Igitur Evathlus postquam cuncta illa exorabula iudicantium et decipula adversantium et artificia dicentium versutus alioqui et ingeniatus ad astutiam facile perdidicit, contentus scire quod concupierat, coepit nolle quod pepigerat, sed callide nectendis moris frustrari magistrum diutuleque nec agere velle nec reddere, usque dum Protagoras eum ad iudices provocavit, expositaque condicione, qua docendum receperat, anceps argumentum ambifariam proposuit. ``Nam sive ego vicero'', inquit, ``solvere mercedem debebis ut condemnatus, seu tu viceris, nihilo minus reddere debebis ut pactus, quippe qui hanc causam primam penes iudices viceris. Ita, si vincis, in condicionem incidisti; si vinceris, in damnationem.'' Quid quaeris? ratio conclusa iudicibus acriter et invincibiliter videbatur. Enimvero Evathlus, utpote tanti veteratoris perfectissimus discipulus, biceps illud argumentum retorsit. Nam ``si ita est'', inquit, ``neutro modo quod petis debeo. Aut enim vinco et iudicio dimittor, aut vincor et pacto absolvor, ex quo non debeo mercedem, si hanc primam causam fuero penes iudices victus. Ita me omni modo liberat, si vincor, condicio, si vinco, sententia.'' Nonne vobis videntur haec sophistarum argumenta obversa invicem vice spinarum, quas ventus convolverit, inter se cohaerere, paribus utrimque aculeis, simili penetratione, mutuo vulnere? Atque ideo merces Protagorae tam aspera, tam senticosa versutis et avaris relinquenda est: cui scilicet multo tanta praestat illa altera merces, quam Thalen memorant suasisse. 

Thales Milesius ex septem illis sapientiae memoratis viris facile praecipuus – enim geometriae penes Graios primus repertor et naturae rerum certissimus explorator et astrorum peritissimus contemplator – maximas res parvis lineis repperit: temporum ambitus, ventorum flatus, stellarum meatus, tonitruum sonora miracula, siderum obliqua curricula, solis annua reverticula, itidem lunae vel nascentis incrementa vel senescentis dispendia vel delinquentis obstiticula. Idem sane iam proclivi senectute divinam rationem de sole commentus est, quam equidem non didici modo, verum etiam experiundo comprobavi, quoties sol magnitudine sua circulum quem permeat metiatur. Id a se recens inventum Thales memoratur edocuisse Mandraytum Prienensem, qui nova et inopinata cognitione impendio delectatus optare iussit quantam vellet mercedem sibi pro tanto documento rependi. ``Satis'' inquit, ``mihi fuerit mercedis'', Thales sapiens, ``si id quod a me didicisti, cum proferre ad quospiam coeperis, tibi non adsciveris, sed eius inventi me potius quam alium repertorem praedicaris.'' Pulchra merces prorsum ac tali viro digna et perpetua; nam et in hodiernum ac dein semper Thali ea merces persolvetur ab omnibus nobis, qui eius caelestia studia vere cognovimus. 

Hanc ego vobis, mercedem, Carthaginienses, ubique gentium dependo pro disciplinis, quas in pueritia sum apud vos adeptus. Vbique enim me vestrae civitatis alumnum fero, ubique vos omnimodis laudibus celebro, vestras disciplinas studiosius percolo, vestras opes gloriosius praedico, vestros etiam deos religiosius veneror. 

Nunc quoque igitur principium mihi apud vestras auris auspicatissimum ab Aesculapio deo capiam, qui arcem nostrae Carthaginis indubitabili numine propitius respicit. Eius dei hymnum Graeco et Latino carmine vobis etiam canam [iam] illi a me dedicatum. Sum enim non ignotus illi sacricola nec recens cultor nec ingratus antistes, ac iam et prorsa et vorsa facundia veneratus sum, ita ut etiam nunc hymnum eius utraque lingua canam, cui dialogum similiter Graecum et Latinum praetexui, in quo sermocinabuntur Sabidius Severus et Iulius Persius, viri et inter se mutuo et vobis et utilitatibus publicis merito amicissimi, doctrina et eloquentia et benivolentia paribus, incertum modestia quietiores an industria promptiores an honoribus clariores. Quibus cum sit summa concordia, tamen haec sola aemulatio et in hoc unum certamen est, uter eorum magis Carthaginem diligat, atque summis medullitus viribus contendunt ambo, vincitur neuter. Eorum ego sermonem ratus et vobis auditu gratissimum et mihi compositu congruentem et deo dedicatu religiosum, in principio libri facio quendam ex his, qui mihi Athenis condidicerunt, percontari a Persio Graece quae ego pridie in templo Aesculapi disserverim, paulatimque illis Severum adiungo, cui interim Romanae linguae partes dedi. Nam et Persius, quamvis et ipse optime possit, tamen hodie vobis atticissabit. 

\section*{Apologia (ca. 150-160)}
\begin{quotation}
\noindent Et sane verisimile videtur, cum Apuleius minus se locupletem cerneret, ideo facilius impulsum ut Aemiliae Pudentillae nuptias non aspernaretur; nam alioquin parum dignum illius ingenio, ut juvenis corpore et animo elegans duceret viduam, nullis formae blandimentis, et annis rebus voluptariis jam fere abnuentibus, nisi prolixa dote et molli conditione invitatus: quamvis ipse multis verbis amoliri conatur hanc suspicionem, nullaque avaritiae causa concupisse uxorem ex tabulis docet. Qua vero occasione Oeam venerit, qua conditione Prudentillae matrimonio junctus sit, copiose narrat. Ex hac tamen utriusque dissimilitudine nata opinio, veneno et magicis maleficiis coactam mulierem nupsisse; inde apud Cl.\ Maximum, Africae procos.\ a Licinio Aemiliano reus Magiae postulatus, quod scilicet sexagenariam, post longam viduitatem, carminibus ad sui aetatem flexisset ut grandem dotem amanti mulieri extorqueret; sed adversus intentata crimina insignis eloquentiae praesidio munitus peritissime velitatur, eaque innocenti sine causa objecta, longa oratione ostendit.
\end{quotation}

\subsection*{Apologia, 78-79}

sed ne longius ab ordine digrediar: Pudentilla postquam filium uidet praeter opinionem contra suam esse sententiam deprauatum, rus profecta scripsit ad eum obiurgandi gratia illas famosissimas litteras, quibus, ut isti aiebant, confessa est sese mea magia in amorem inductam dementire; quas tamen litteras tabulario Pontiani praesente et contra scribente Aemiliano nudius tertius tuo iussu, Maxime, testato describsimus; in quibus omnia contra praedicationem istorum pro me reperiuntur. 

quamquam, etsi destrictius magum me dixisset, posset uideri excusabunda se filio uim meam quam uoluntatem suam causari maluisse. an sola Phaedra falsum epistolium de amore commenta est, ac non omnibus mulieribus haec ars usitata est, ut, cum aliquid eius modi uelle coeperunt, malint coactae uideri? quod si etiam animo ita putauit, me magum esse, idcircone magus habear, quia hoc scripsit Pudentilla? uos tot argumentis, tot testibus, tanta oratione magum me non probatis: illa uno uerbo probaret? et quanto tandem grauius habendum est quod in iudicio subscribitur quam quod in epistola scribitur. quin tu me meismet factis, non alienis uerbis reuincis? 

ceterum eadem uia multi rei cuiusuis maleficii postulabuntur, si ratum futurum est quod quisque in epistola sua uel amore uel odio cuiuspiam scripserit. `magum te scripsit Pudentilla: igitur  magus es.' quid, si consulem me scripsisset: consul essem? quid enim, si pictorem, si medicum, quid denique, si innocentem? num aliquid horum putares idcirco, quod illa dixisset? nihil scilicet. atqui periniurium est ei fidem in peioribus $\langle$habere, cui in melioribus$\rangle$ non haberes, posse litteras eius ad perniciem, non posse ad salutem. 

`sed' inquit `animi $\langle$furens$\rangle$ fuit, efflictim te amabat.' concedo interim. num tamen omnes qui amantur magi sunt, si hoc forte qui amat scripserit? c$\langle$r$\rangle$edo nunc quod Pudentilla me in eo tempore non amabat, siquidem id foras scripsit, quod palam erat mihi obfuturum. 

postremo quid uis, sanam an insanam fuisse, dum scriberet? sanam dices? nihil ergo erat magicis artibus passa. insanam respondebis? nesciit ergo quid scribserit, eoque ei fides non habenda est; immo etiam, si fuisset insana, insanam se esse nescisset. nam ut absurde facit qui tacere se dicit, quod ibidem dicendo tacere sese non tacet et ipsa professione quod profitetur infirmat, ita uel magis hoc repugnat: `ego insanio', quod uerum non est, nisi sciens dicit; porro sanus est, qui scit quid sit insania, quippe insania scire se non potest, non magis quam caecitas se uidere. igitur Pudentilla compos mentis fuit, si compotem mentis se non putabat. 

possum, si uelim, pluribus, sed mitto dialectica. ipsas litteras longe aliud clamantis et quasi dedita opera ad iudicium istud praeparatas et accommodatas recitabo. accipe tu et lege, usque dum ego interloquar.— 
