%\section*{O autoru}

\section*{Laudes fumi et pulveris}
	
Caesari suo Fronto

\medskip

\noindent Plerique legentium forsan rem de titulo contemnant: nihil serium potuisse fieri de fumo et pulvere; tu pro tuo excellenti ingenio profecto existimabis lusa sit opera ista an locata. 	

Sed res poscere videtur de ratione scribendi pauca praefari, quod nullum huiusce modi scriptum Romana lingua exstat satis nobile, nisi quod poetae in comoedis vel Atellanis adtigerunt. Qui se in eius modi rebus scribendis exercebit, crebras sententias conquiret easque dense conlocabit et subtiliter coniunget, neque verba multa geminata supervacanea inferciet; tum omnem sententiam breviter et scite concludet. Aliter in orationibus iudiciariis, ubi sedulo curamus ut pleraeque sententiae durius interdum et incomptius finiantur. Sed contra istic laborandum est, ne quid inconcinnum vel hiulcum relinquatur, quin omnia ut in tenui veste oris detexta et revimentis sint cincta. Postremo, ut novissimos in epigrammatis versus habere oportet aliquid luminis, sententia clavi aliqua vel fibula terminanda est. 	

In primis autem sectanda est suavitas. Namque hoc genus orationis non capitis defendendi nec suadendae legis nec exercitus adhortandi nec inflammandae contionis scribitur, sed facetiarum et voluptatis. Ubique vero ut de re ampla et magnifica loquendum, parvaeque res magnis adsimulandae comparandaeque. Summa denique in hoc genere orationis virtus est adseveratio. Fabulae deum vel heroum tempestive inserendae, item versus congruentes et proverbia accommodata et non inficete conficta mendacia, dum id mendacium argumento aliquo lepido iuvetur.

(\dots)

Laudabo igitur deos infrequentes quidem a laudibus, verum in usu cultuque humano frequentissimos, fumum et pulverem, sine quis neque asae neque foci neque viae, quod volgo aiunt, nec semitae usurpantur. Quod si quis hoc primum a$\langle$m$\rangle$bigit, habendusve sit fumus in numero deorum, cogitet ventos quoque in deum numero haberi, quaeque sunt fumo simillimae, nebulas nubesque putari deas et in caelo conspici et, ut poetae ferunt, amiciri deos nubibus, et Iovi Iunonique cubantibus nubem ab arbitris obstitisse. Quod $\langle$si$\rangle$ nunc divinae naturae proprium est, nec fumum manu prehendere nec solem queas, neque vincire neque verberare neque detinere neque, vel minimum rimae si deposcat, excludere * * * 
