%\section*{O autoru}
\section*{De Ciceronis oratione Miloniana}

\subsection*{Testimonia}

Cassius Dio Cocceianus, Historiae Romanae, 40, 48–55

\bigskip


\begin{greek}

48. τοιαύτης οὖν τότε τῆς ἐν τῷ ἄστει καταστάσεως οὔσης, καὶ μηδενὸς τοῖς πράγμασιν ἐπιτεταγμένου, σφαγαὶ καθ᾽ ἑκάστην ἡμέραν ὡς εἰπεῖν ἐγίγνοντο, τάς τε ἀρχαιρεσίας, καίτοι σπεύδοντες ἐπὶ τὰς ἀρχὰς καὶ δεκασμοῖς καὶ φόνοις δι᾽ αὐτὰς χρώμενοι, οὐκ ἐπετέλουν. ὁ γοῦν Μίλων ὑπατείαν αἰτῶν τὸν Κλώδιον ἐν τῇ Ἀππίᾳ ὁδῷ συντυχόντα οἱ τὸ μὲν πρῶτον ἁπλῶς πως ἔτρωσεν, ἔπειτα δὲ φοβηθεὶς μὴ ἐπεξέλθῃ τῷ γεγονότι κατέσφαξεν, ἐλπίσας, ἐπειδὴ πάντας τοὺς οἰκέτας τοὺς τοῦτο ποιήσαντας εὐθὺς ἠλευθέρωσε, ῥᾷον τοῦ φόνου τελευτήσαντος αὐτοῦ ἢ τοῦ τραύματος εἰ περιγίγνοιτο ἀφεθήσεσθαι. 

ἀκούσαντες οὖν τοῦθ᾽ οἱ ἐν τῇ πόλει πρὸς ἑσπέραν δεινῶς ἐταράχθησαν: ταῖς τε γὰρ στάσεσιν ἀφορμὴ πολέμου καὶ κακῶν ἐγίγνετο, καὶ οἱ διὰ μέσου, εἰ καὶ ἐμίσουν τὸν Κλώδιον, ὅμως διά τε τὸ ἀνθρώπινον καὶ ὅτι καὶ τοῦ Μίλωνος στερηθῆναι ἐπὶ τῇ προφάσει ταύτῃ ἤθελον, ἠγανάκτουν. 

49. παραλαβόντες δὲ αὐτοὺς οὕτως ἔχοντας ὅ τε Ῥοῦφος καὶ Τίτος Μουνάτιος Πλάγκος προσπαρώξυναν: δημαρχοῦντες γὰρ ἔς τε τὴν ἀγορὰν τὸν νεκρὸν ὑπὸ τὴν ἕω ἐσεκόμισαν καὶ ἐπὶ τὸ βῆμα ἐπέθεσαν πᾶσί τε ἐπεδείκνυσαν, καὶ ἐπέλεγον οἷα εἰκὸς ἦν ὀδυρόμενοι, ὥστε τὸν ὅμιλον καὶ ἐξ ὧν ἑώρων καὶ ἐξ ὧν ἤκουον συνταραχθῆναι, καὶ μήτε τοῦ ὁσίου μήτε τοῦ θείου ἔτι φροντίσαι, ἀλλὰ πάντα μὲν τὰ περὶ τὰς ταφὰς νόμιμα συγχέαι, πᾶσαν δὲ ὀλίγου τὴν πόλιν καταπρῆσαι. τὸ γὰρ σῶμα τοῦ Κλωδίου ἀράμενοι ἔς τε τὸ βουλευτήριον ἐσήνεγκαν, καὶ εὐθέτησαν, καὶ μετὰ τοῦτο πυρὰν ἐκ τῶν βάθρων συννήσαντες ἔκαυσαν καὶ ἐκεῖνο καὶ τὸ συνέδριον. 

οὕτω τε οὐχ ὁρμῇ τινι, οἵα που τοὺς ὄχλους ἐξαπιναία καταλαμβάνει, ἀλλὰ ἐκ προαιρέσεως αὐτὸ ἔπραξαν ὥστε καὶ τὴν ἐνάτην τὸ περίδειπνον ἐν αὐτῇ τῇ ἀγορᾷ, τυφομένου ἔτι τοῦ βουλευτηρίου, ποιῆσαι, καὶ προσέτι καὶ τὴν οἰκίαν τὴν τοῦ Μίλωνος καταφλέξαι ἐπιχειρῆσαι. ἐκείνη μὲν οὖν πολλῶν αὐτῇ ἀμυνάντων οὐκ ἐκαύθη: ὁ δὲ δὴ Μίλων τέως μὲν περίφοβος ἐπὶ τῷ φόνῳ ὢν ἐκρύπτετο, οὐχ ὑπὸ ἰδιωτῶν μόνον ἀλλὰ καὶ ἱππέων βουλευτῶν τέ τινων φρουρούμενος: ἐπεὶ δὲ τοῦτό τε ἐγένετο καὶ τὴν ὀργὴν τῆς γερουσίας ἐς τὸ τῶν ἀντιστασιωτῶν μίασμα περιχωρήσειν ἤλπισεν ῾εὐθὺς γοῦν τῆς δείλης ἐς τὸ Παλάτιον δι᾽ αὐτὸ τοῦτο συλλεγέντες τόν τε μεσοβασιλέα προχειρισθῆναι, καὶ τῆς φυλακῆς τῆς πόλεως καὶ ἐκεῖνον καὶ τοὺς δημάρχους καὶ προσέτι καὶ τὸν Πομπήιον ἐπιμεληθῆναι ὥστε μηδὲν ἀπ᾽ αὐτῆς ἀποτριβῆναι, ἐψηφίσαντὀ, προῄει τε ἐς τὸ μέσον καὶ τῆς ἀρχῆς ὁμοίως ἢ καὶ μᾶλλον ἀντεποιεῖτο.

\end{greek}

\bigskip


\begin{croatian}

(Senat proglasio Pompeja konzulom, bez kolege, da to ne bi postao Cezar; Pompej postigao da mu kolega bude Kvint Scipion, optužen za podmićivanje; Pompej reformirao postupanje sudova, kontrolira popis porotnika i broj odvjetnika u pojedinoj parnici, ukinuo karakterne svjedoke, poticao tužbe zbog podmićivanja.)

\end{croatian}

\bigskip

\begin{greek}

53. ἄλλοι τε οὖν ἐκ τούτου πολλοὶ ἑάλωσαν καὶ Πλαύτιος Ὑψαῖος ἀνταιτήσας τῷ τε Μίλωνι καὶ τῷ Σκιπίωνι τὴν ὑπατείαν. τῶν γὰρ δὴ τριῶν δωροφορησάντων μόνος ἐκεῖνος κατεδικάσθη. ὅ τε γὰρ Σκιπίων ἐγράφη μὲν καὶ ὑπὸ δυοῖν γε, οὐκ ἐκρίθη δὲ διὰ τὸν Πομπήιον: καὶ ὁ Μίλων ἐπὶ μὲν τούτῳ οὐκ ἐσήχθη ῾τὸ γὰρ τοῦ φόνου ἔγκλημα μεῖζον εἶχεν᾽, ὑπαχθεὶς δὲ ἐπ᾽ ἐκείνῳ ἑάλω, μηδὲν δυνηθεὶς βίαιον δρᾶσαι. ὁ γὰρ Πομπήιος τήν τε ἄλλην πόλιν διὰ φυλακῆς ἐποιήσατο, καὶ ἐς τὸ δικαστήριον σὺν ὁπλίταις ἐσῆλθε, θορυβησάντων τε ἐπὶ τούτῳ τινῶν προσέταξε τοῖς στρατιώταις ἐκδιῶξαι αὐτοὺς ἐκ τῆς ἀγορᾶς πλαγίοις καὶ πλατέσι τοῖς ξίφεσι παίοντας. ἐπειδή τε οὐχ ὑπεῖκον ἀλλὰ καὶ καθάπερ ἐν παιδιᾷ τινι πλαγιαζόμενοι ὕβριζον, καὶ ἐτρώθησάν τινες αὐτῶν καὶ ἀπέθανον. 

54. τά τε οὖν δικαστήρια ἡσύχως ἐκ τούτων συνήγετο, καὶ ἐδικαιώθησαν ἐπί τε ἑτέροις τισὶ πολλοὶ καὶ ἐπὶ τῷ τοῦ Κλωδίου φόνῳ ἄλλοι τε καὶ ὁ Μίλων, καίτοι τὸν Κικέρωνα συναγωνιστὴν ἔχων. ὁ γὰρ ῥήτωρ ἐκεῖνος τόν τε Πομπήιον καὶ τοὺς στρατιώτας ἐν τῷ δικαστηρίῳ παρὰ τὸ καθεστηκὸς ἰδὼν ἐξεπλάγη καὶ κατέδεισεν, ὥστε τῶν μὲν παρεσκευασμένων μηδὲν εἰπεῖν, βραχὺ δέ τι καὶ τεθνηκὸς χαλεπῶς φθεγξάμενος ἀγαπητῶς μεταστῆναι. τοῦτον γὰρ τὸν λόγον τὸν νῦν φερόμενον ὡς καὶ ὑπὲρ τοῦ Μίλωνος τότε λεχθέντα χρόνῳ ποθ᾽ ὕστερον καὶ κατὰ σχολὴν ἀναθαρσήσας ἔγραψε: καὶ δὴ καὶ τοιόνδε τι περὶ αὐτοῦ παραδέδοται. 

ὁ Μίλων τῷ λόγῳ πεμφθέντι οἱ ὑπ᾽ αὐτοῦ ἐντυχών ῾ἐπεφυγάδευτο γὰρ᾽ ἀντεπέστειλε λέγων ὅτι ἐν τύχῃ αὐτῷ ἐγένετο τὸ μὴ ταῦθ᾽ οὕτω καὶ ἐν τῷ δικαστηρίῳ λεχθῆναι: οὐ γὰρ ἂν τοιαύτας ἐν τῇ Μασσαλίᾳ ῾ἐν ᾗ κατὰ τὴν φυγὴν ἦν᾽ τρίγλας ἐσθίειν, εἴπερ τι τοιοῦτον ἀπελελόγητο. τοῦτο δὲ ἔγραψεν οὐχ ὅτι τοῖς παροῦσιν ἠρέσκετο ῾πολλὰ γὰρ ἐπὶ τῇ καθόδῳ ἐπετόλμησεν᾽ ἀλλ᾽ ἐς τὸν Κικέρωνα ἀποσκώπτων, ὅτι μηδὲν χρηστὸν ἐν τῷ τῆς ἀπολογίας καιρῷ εἰπὼν ἔπειτα ἀκάρπους λόγους καὶ ἐμελέτα καὶ ἔπεμπεν αὐτῷ, ὥσπερ τι ὠφελῆσαι τότε αὐτὸν δυναμένους.


55. ὅ τε οὖν Μίλων οὕτως ἑάλω, καὶ ὁ Ῥοῦφος ὅ τε Πλάγκος ἐπειδὴ πρῶτον ἐκ τῆς ἀρχῆς ἐξῆλθον, ἄλλοι τε σὺν αὐτοῖς συχνοὶ διὰ τὴν τοῦ βουλευτηρίου ἔμπρησιν, καίτοι τῷ Πλάγκῳ καὶ τοῦ Πομπηίου συσπουδάσαντος, ὥστε καὶ βιβλίον ἔπαινόν τε ἅμα αὐτοῦ καὶ ἱκετείαν ἔχον ἐς τὸ δικαστήριον ἔπεμψεν: ὁ γὰρ Κάτων ὁ Μᾶρκος ῾δικάζειν δὲ ἔμελλεν᾽ οὐκ ἔφη τὸν ἐπαινέτην ἐπὶ τῇ τῶν ἑαυτοῦ νόμων καταλύσει προσίεσθαι. καὶ ὁ μὲν οὐκέτι τὴν ψῆφον ἔδωκεν: ὁ γὰρ Πλάγκος ὡς καὶ τὴν καθαιρήσουσαν αὐτὸν οἴσοντα ἐξέκρινεν ῾ἐξῆν γάρ, ἐκ τῶν Πομπηίων νόμων, πέντε ἑκατέρῳ τῶν διαδικούντων ἐκ τῶν δικάσειν σφίσι μελλόντων ἀπολέγειν᾽: οἱ μέντοι ἄλλοι δικασταὶ κατεψηφίσαντο αὐτοῦ. οὔτε γὰρ ἄλλως ὀρθῶς ἔχειν ἔδοξέ σφισι, τοῦ Ῥούφου κατεγνωκόσιν, ἐκεῖνον ἐπὶ τοῖς αὐτοῖς κρινόμενον ἀφεῖναι: καὶ ἐπειδὴ τὸν Πομπήιον συναιρόμενόν οἱ εἶδον, ἀντεσπούδασαν αὐτῷ, μὴ καὶ δοῦλοί τινες ἄντικρυς αὐτοῦ μᾶλλον ἢ δικασταὶ νομισθῶσιν εἶναι. 

καίπερ καὶ τότε ὁ Κικέρων οὐδὲν βέλτιον τοῦ Πλάγκου κατηγόρησεν ἢ ὑπὲρ τοῦ Μίλωνος ἀπελογήσατο: ἥ τε γὰρ τοῦ δικαστηρίου ὄψις ἡ αὐτὴ ἦν, καὶ ὁ Πομπήιος ἐν ἑκατέρῳ τἀναντία οἱ καὶ ἐβουλεύετο καὶ ἔπραττεν, ὅθεν οὐχ ἥκιστα αὗθις αὐτῷ προσέκρουσε. 

\end{greek}

\subsection*{Argumentum}

Summae erant inimicitiae Titum Annium Milonem inter et Publium Clodium, utrumque civem Romanum. Odii Clodiani in Milonem haec causa fuit quod Milo peteret consulatum eo anno quo Clodius praeturam. Itaque Clodius summa vi ac studio nitebatur ne Milo eo anno consul esset; quod cum impedire non posset (Milo enim omnium voce consul futurus pronuntiabatur) Clodius de Milone interficiendo cogitavit. Egressus Urbe in villam quandam in Albano se recepit; unde vesperi rediens obvium in via Appia Milonem habuit paenulatum, cum uxore, ancillis et pueris. Ibi ante fundum Clodianum orta inter servos rixa et impetu in Milonem facto Clodius occiditur. Reus caedis Milo in ius vocatur. Qui plura volet, consulat Asconium Pedianum qui rem fusissime pertractat.

\subsection*{Personae}

Reus caedis Milo; accusatores Appius maior, M. Antonius et P. Valerius Nepos. Patronus solus Cicero. Quaesitor extra ordinem, populi suffragio creatus, L. Domitius Aenobarbus; iudices quinquaginta et unus, senatores nempe octodecim, equites septemdecim, tribuni aerarii sexdecim.

\subsection*{Tempus}

Habita est haec oratio III. Idus April. Pompeio Magno tertium consule sine collega, anno Urbis conditae 701, Ciceronis aetatis 54. Qui cum inciperet dicere, Clodianorum clamoribus exceptus est; unde non ea libertate et vi, qua solitus erat, pronuntiavit.

\subsection*{Locus}

Acta fuit in foro quod dispositis armatis militibus complevit Pompeius; ipse vero cum praesidio ad aerarium stabat.

\subsection*{Causa}

Publica est, de vi, et in genere iudiciali; et, quia non ambiguum erat quin a Milone Clodius interfectus fuisset, quaeritur uter utri insidias fecerit.

\subsection*{Partes}

Illius orationes partes sunt: exordium, propositio cum distributione, confutatio, narratio, constitutio causae, confirmatio, peroratio.

\subsection*{Eventus}

Triginta et octo iudicum suffragiis damnatus est Milo. Sciebant illi quidem inscio Milone Clodium vulneratum esse; sed compererant, inquit Asconius Pedianus, postquam vulneratus esset iussu Milonis interfectum fuisse. Statim atque condemnatus est Milo, in exsilium Massiliam profectus est, quo cum ei delata fuisset haec oratio tanta ingenii subtilitate elaborata, dixisse fertur: »Si sic egisses, M. Tulli, barbatos pisces Milo Massiliae non ederet.«

\subsection*{Exordium}

Exorditur Cicero ab adiunctis huius iudicii

\subsection*{Confutatio}

Continet haec refutatio tria praeiudicia quae contra Milonem a Clodianis iactabantur. Primum, nefas esse ei vivere qui a se hominem interfectum esse fatebatur. Secundum, Milonem iam damnatum esse senatus iudicio. Tertio, eundem Milonem esse damnatum sententia Cn. Pompeii.

\subsection*{Narratio}

Tria capita continet haec narratio. 1. Clodii consilia. 2. Profectionem. 3. Pugnam.

\subsection*{Confirmatio}

Duas partes continet haec confirmatio. In prima ostendit insidias fecisse Clodium Miloni. In secunda, gloriosum Miloni fuisse Clodium interfecisse. Primam partem tractat 1. per coniecturam, ab antecedentibus, sive personis; 2. a concomitantibus, sive ab ipsis facti circumstantiis; 3. a confutatione quorundam quae ab adversariis obiiciuntur; 4. a consequentibus.

\subsection*{Prima probatio}

Ab antecedentibus: non Milonem Clodio, sed Clodium Miloni struxisse insidias.

\subsection*{Secunda probatio}

A concomitantibus sive ab ipsis facti circumstantiis.

\subsection*{Tertia probatio}

Ubi refutantur quae de servis obiiciebantur.

\subsection*{Quarta probatio}

A consequentibus Milonem non fecisse insidias.

\subsection*{Peroratio}

Tota haec peroratio ad misericordiam comparata est.


\section*{Q. Asconii Pediani orationum Ciceronis quinque enarratio. Pro Milone (c. a. 54 a. D.)}

Quintus Asconius Pedianus (c. 9 a. Chr. n. – c. 76 a. D.) erat historicus Romanus, qui post a. 54 a. D. ad filios commentarios de orationibus Ciceronis composuit; quinque commentarii servantur, inter quos commentarius orationis pro Milone (a. 52 a. Chr. n. habitae).

\subsection*{Enarratio}


Orationem hanc dixit Cn. Pompeio III cos. a. d. VII Id. April. Quod iudicium cum ageretur, exercitum in foro et in omnibus templis quae circum forum sunt collocatum a Cn. Pompeio fuisse non tantum ex oratione et annalibus, sed etiam ex libro apparet qui Ciceronis nomine inscribitur de optimo genere oratorum.

\subsection*{Argumentum hoc est}

T.~Annius Milo et P. Plautius Hypsaeus et Q. Metellus Scipio consulatum petierunt non solum largitione palam profusa sed etiam factionibus armatorum succincti. Miloni et Clodio summae erant inimicitiae, quod et Milo Ciceronis erat amicissimus in reducendoque eo enixe operam tr.~pl. dederat, et P. Clodius restituto quoque Ciceroni erat infestissimus ideoque summe studebat Hypsaeo et Scipioni contra Milonem. Ac saepe inter se Milo et Clodius cum suis factionibus Romae depugnaverant; et erant uterque audacia pares, sed Milo pro melioribus partibus stabat. Praeterea in eundem annum consulatum Milo, Clodius praeturam petebat, quam debilem futuram consule Milone intellegebat. 

Deinde cum diu tracta essent comitia consularia perficique ob eas ipsas perditas candidatorum contentiones non possent, et ob id mense Ianuario nulli dum neque consules neque praetores essent trahereturque dies eodem quo antea modo – cum Milo quam primum comitia confici vellet confideretque cum bonorum studiis, quod obsistebat Clodio, tum etiam populo propter effusas largitiones impensasque ludorum scaenicorum ac gladiatorii muneris maximas, in quas tria patrimonia effudisse eum Cicero significat; competitores eius trahere vellent, ideoque Pompeius gener Scipionis et T.~Munatius tribunus plebis referri ad senatum de patriciis convocandis qui interregem proderent non essent passi, cum interregem prodere stata res esset; a. d. XIII Kal. Febr. – Acta etenim magis sequenda et ipsam orationem, quae Actis congruit, puto quam Fenestellam qui a. d. XIIII Kal. Febr. tradit – Milo Lanuvium, ex quo erat municipio et ubi tum dictator, profectus est ad flaminem prodendum postera die. 

Occurrit ei circa horam nonam Clodius paulo ultra Bovillas, rediens ab Aricia, prope eum locum in quo Bonae Deae sacellum est; erat autem allocutus decuriones Aricinorum. Vehebatur Clodius equo; servi XXX fere expediti, ut illo tempore mos erat iter facientibus, gladiis cincti sequebantur. Erant cum Clodio praeterea tres comites eius, ex quibus eques Romanus unus C.~Causinius Schola, duo de plebe noti homines P.~Pomponius, C.~Clodius. Milo raeda vehebatur cum uxore Fausta, filia L.~Sullae dictatoris, et M.~Fufio familiari suo. Sequebatur eos magnum servorum agmen, inter quos gladiatores quoque erant, ex quibus duo noti Eudamus et Birria. Ii in ultimo agmine tardius euntes cum servis P.~Clodi rixam commiserunt. Ad quem tumultum cum respexisset Clodius minitabundus, umerum eius Birria rumpia traiecit. Inde cum orta esset pugna, plures Miloniani accurrerunt. Clodius vulneratus in tabernam proximam in Bovillano delatus est. Milo ut cognovit vulneratum Clodium, cum sibi periculosius illud etiam vivo eo futurum intellegeret, occiso autem magnum solacium esset habiturus, etiam si subeunda esset poena, exturbari taberna iussit. Fuit antesignanus servorum eius M. Saufeius. 

Atque ita Clodius latens extractus est multisque vulneribus confectus. Cadaver eius in via relictum, quia servi Clodi aut occisi erant aut graviter saucii latebant, Sex.~Teidius senator, qui forte ex rure in urbem revertebatur, sustulit et lectica sua Romam ferri iussit; ipse rursus eodem unde erat egressus se recepit. Perlatum est corpus Clodi ante primam noctis horam, infimaeque plebis et servorum maxima multitudo magno luctu corpus in atrio domus positum circumstetit. Augebat autem facti invidiam uxor Clodi Fulvia quae cum effusa lamentatione vulnera eius ostendebat. Maior postera die luce prima multitudo eiusdem generis confluxit, compluresque noti homines visi sunt. 

Erat domus Clodi ante paucos menses empta de M.~Scauro in Palatio; eodem T.~Munatius Plancus, frater L.~Planci oratoris, et Q.~Pompeius Rufus, Sullae dictatoris ex filia nepos, tribuni plebis accurrerunt; eisque hortantibus vulgus imperitum corpus nudum ac calcatum, sicut in lecto erat positum, ut vulnera videri possent in forum detulit et in rostris posuit. Ibi pro contione Plancus et Pompeius qui competitoribus Milonis studebant invidiam Miloni fecerunt. Populus duce Sex.~Clodio scriba corpus P.~Clodi in curiam intulit cremavitque subselliis et tribunalibus et mensis et codicibus librariorum; quo igne et ipsa quoque curia flagravit, et item Porcia basilica quae erat ei iuncta ambusta est. Domus quoque M.~Lepidi interregis – is enim magistratus curulis erat creatus – et absentis Milonis eadem illa Clodiana multitudo oppugnavit, sed inde sagittis repulsa est. Tum fasces ex luco Libitinae raptos attulit ad domum Scipionis et Hypsaei, deinde ad hortos Cn.~Pompeii, clamitans eum modo consulem, modo dictatorem.

Incendium curiae maiorem aliquanto indignationem civitatis moverat quam interfectio Clodi. Itaque Milo, quem opinio fuerat ivisse in voluntarium exsilium, invidia adversariorum recreatus nocte ea redierat Romam qua incensa erat curia. Petebatque nihil deterritus consulatum; aperte quoque tributim in singulos milia assium dederat. Contionem ei post aliquot dies dedit M.~Caelius tribunus plebis ac Cicero ipse etiam causam egit ad populum. Dicebant uterque Miloni a Clodio factas esse insidias.

Fiebant interea alii ex aliis interreges, quia comitia consularia propter eosdem candidatorum tumultus et easdem manus armatas haberi non poterant. Itaque primo factum erat S.~C. ut interrex et tribuni plebis et Cn.~Pompeius, qui pro cos.\ ad urbem erat, viderent ne quid detrimenti res publica caperet, dilectus autem Pompeius tota Italia haberet. Qui cum summa celeritate praesidium comparasset, postulaverunt apud eum familiam Milonis, item Faustae uxoris eius exhibendam duo adulescentuli qui Appii Claudii ambo appellabantur; qui filii erant C.~Claudi, qui frater fuerat Clodi, et ob id illi patrui sui mortem velut auctore patre persequebantur. Easdem Faustae et Milonis familias postulaverunt duo Valerii, Nepos et Leo, L.~Herennius Balbus P.~Clodi quoque familiam et comitum eius postulavit; eodem tempore Caelius familiam Hypsaei et Q.~Pompeii postulavit. Adfuerunt Miloni Q.~Hortensius, M.~Cicero, M.~Marcellus, M.~Calidius, M.~Cato, Faustus Sulla. 

Verba pauca Q.~Hortensius dixit, liberos esse eos qui pro servis postularentur; nam post recentem caedem manu miserat eos Milo sub hoc titulo quod caput suum ulti essent. Haec agebantur mense intercalari. Post diem tricesimum fere quam erat Clodius occisus Q. Metellus Scipio in senatu contra Q.~Caepionem conquestus est de hac caede P.~Clodi. Falsum esse dixit, quod Milo sic se defenderet, sed Clodium Aricinos decuriones alloquendi gratia abisse profectum cum sex ac XX servis; Milonem subito post horam quartam, senatu misso, cum servis amplius CCC armatis obviam ei contendisse et supra Bovillas inopinantem in itinere aggressum. Ibi P.~Clodium tribus vulneribus acceptis Bovillas perlatum; tabernam in quam perfugerat expugnatam a Milone; semianimem Clodium extractum . . . in via Appia occisum esse anulumque eius ei morienti extractum. Deinde Milonem, cum sciret in Albano parvolum filium Clodi esse, venisse ad villam et, cum puer ante subtractus esset, ex servo Halicore quaestionem ita habuisse ut eum articulatim consecaret; vilicum et duos praeterea servos iugulasse. Ex servis Clodi qui dominum defenderant undecim esse interfectos, Milonis duos solos saucios factos esse; ob quae Milonem postero die XII servos qui maxime operam navassent manu misisse populoque tributim singula milia aeris ad defendendos de se rumores dedisse. Milo misisse ad Cn.~Pompeium dicebatur qui Hypsaeo summe studebat, quod fuerat eius quaestor, desistere se petitione consulatus, si ita ei videretur; Pompeius respondisse nemini se neque petendi neque desistendi auctorem esse, neque populi Romani potestatem aut consilio aut sententia interpellaturum. Deinde per C.~Lucilium, qui propter M.~Ciceronis familiaritatem amicus erat Miloni, egisse quoque dicebatur ne se de hac re consulendo invidia oneraret.

Inter haec cum crebresceret rumor Cn.~Pompeium creari dictatorem oportere neque aliter mala civitatis sedari posse, visum est optimatibus tutius esse eum consulem sine collega creari, et cum tractata ea res esset in senatu, facto in M.~Bibuli sententiam S.~C. Pompeius ab interrege Servio Sulpicio V Kal.~Mart. mense intercalario consul creatus est statimque consulatum iniit. Deinde post diem tertium de legibus novis ferendis rettulit; duas ex S.~C. promulgavit, alteram de vi qua nominatim caedem in Appia via factam et incendium curiae et domum M.~Lepidi interregis oppugnatam comprehendit, alteram de ambitu; poena graviore et forma iudiciorum breviore. Utraque enim lex prius testes dari, deinde uno die atque eodem et ab accusatore et a reo perorari iubebat, ita ut duae horae accusatori, tres reo darentur. 

His legibus obsistere M.~Caelius tr.~pl. studiosissimus Milonis conatus est, quod et privilegium diceret in Milonem ferri et iudicia praecipitari. Et cum pertinacius leges Caelius vituperaret, eo processit irae Pompeius ut diceret, si coactus esset, armis se rem publicam defensurum. Timebat autem Pompeius Milonem seu timere se simulabat; plerumque non domi suae sed in hortis manebat, idque ipsum in superioribus circa quos etiam magna manus militum excubabat. Senatum quoque semel repente dimiserat Pompeius, quod diceret timere se adventum Milonis. Dein proximo senatu P.~Cornificius ferrum Milonem intra tunicam habere ad femur alligatum dixerat; postulaverat ut femur nudaret, et ille sine mora tunicam levarat; tum M.~Cicero exclamaverat omnia illi similia crimina esse quae in Milonem dicerentur alia.

Deinde T.~Munatius Plancus tribunus plebis produxerat in contionem M.~Aemilium Philemonem, notum hominem, libertum M.~Lepidi. Is se dicebat pariterque secum quattuor liberos homines iter facientes supervenisse cum Clodius occideretur, et ob id proclamassent, abreptos et perductos per duos menses in villa Milonis praeclusos fuisse; eaque res seu vera seu falsa magnam invidiam Miloni contraxerat. Idem quoque Munatius et Pompeius tribuni plebis in rostra produxerant triumvirum capitalem, eumque interrogaverant an Galatam Milonis servum caedes facientem deprehendisset. Ille dormientem in taberna pro fugitivo prehensum et ad se perductum esse responderat. Denuntiaverant tamen triumviro, ne servum remitteret; sed postera die Caelius tribunus plebis et Manilius Cumanus collega eius ereptum e domo triumviri servum Miloni reddiderant. Haec, etsi nullam de his criminibus mentionem fecit Cicero, tamen, quia ita compereram, putavi exponenda. 

Inter primos et Q.~Pompeius et C.~Sallustius et T.~Munatius Plancus tribuni plebis inimicissimas contiones de Milone habebant, invidiosas etiam de Cicerone, quod Milonem tanto studio defenderet. Eratque maxima pars multitudinis infensa non solum Miloni sed etiam propter invisum patrocinium Ciceroni. Postea Pompeius et Sallustius in suspicione fuerunt redisse in gratiam cum Milone ac Cicerone; Plancus autem infestissime perstitit, atque in Ciceronem quoque multitudinem instigavit. Pompeio autem suspectum faciebat Milonem, ad perniciem eius comparari vim vociferatus; Pompeiusque ob ea saepius querebatur sibi quoque fieri insidias et id palam, ac maiore manu se armabat. Dicturum quoque diem Ciceroni Plancus ostendebat postea, ante Q.~Pompeius idem meditatus erat. 

Tanta tamen constantia ac fides fuit Ciceronis ut non populi a se alienatione, non Cn.~Pompeii suspicionibus, non periculo futurum ut sibi dies ad populum diceretur, non armis quae palam in Milonem sumpta erant deterreri potuerit a defensione eius; cum posset omne periculum suum et offensionem inimicae multitudinis declinare, redimere autem Cn.~Pompeii animum, si paulum ex studio defensionis remisisset.

Perlata deinde lege Pompei, in qua id quoque scriptum erat ut quaesitor suffragio populi ex iis qui consules fuerant crearetur, statim comitia habita, creatusque est L.~Domitius Ahenobarbus quaesitor. Album quoque iudicum qui de ea re iudicarent Pompeius tale proposuit ut numquam neque clariores viros neque sanctiores propositos esse constaret. Post quod statim nova lege Milo postulatus est a duobus Appiis Claudiis adulescentibus iisdem a quibus antea familia eius fuerat postulata; itemque de ambitu ab iisdem Appiis, et praeterea a C. Ateio et L.~Cornificio; de sodaliciiis etiam a P.~Fulvio Nerato. Postulatus autem erat et de sodaliciis et de ambitu ea spe, quod primum iudicium de vi futurum apparebat, quo eum damnatum iri confidebant nec postea responsurum.

Divinatio de ambitu accusatorum facta est quaesitore A.~Torquato, atque ambo quaesitores, Torquatus et Domitius, prid. Non.~April. reum adesse iusserunt. Quo die Milo ad Domiti tribunal venit, ad Torquati amicos misit; ibi postulante pro eo M.~Marcello obtinuit ne prius causam de ambitu diceret quam de vi iudicium esset perfectum. Apud Domitium autem quaesitorem maior Appius postulavit a Milone servos exhiberi numero IIII et L, et cum ille negaret eos qui nominabantur in sua potestate esse, Domitius ex sententia iudicum pronuntiavit ut ex servorum suorum numero accusator quot vellet ederet. Citati deinde testes secundum legem quae, ut supra diximus, iubebat ut prius quam causa ageretur testes per triduum audirentur, dicta eorum iudices consignarent, quarta die adesse omnes iuberentur ac coram accusatore ac reo pilae in quibus nomina iudicum inscripta essent aequarentur; dein rurusus postera die sortitio iudicum fieret unius et LXXX; qui numerus cum sorte obtigisset, ipsi protinus sessum irent; tum ad dicendum accusator duas horas, reus tres haberet, resque eodem die illo iudicaretur; prius autem quam sententiae ferrentur, quinos ex singulis ordinibus accusator, totidem reus reiceret, ita ut numerus iudicum relinqueretur qui sententias ferrent quinquaginta et unus.

Primo die datus erat in Milonem testis C. Causinius Schola, qui se cum P. Clodio fuisse, cum is occisus esset, dixit, atrocitatemque rei factae quam maxime potuit auxit. Quem cum interrogare M. Marcellus coepisset, tanto tumultu Clodianae multitudinis circumstantis exterritus est ut vim ultimam timens in tribunal a Domitio reciperetur. Quam ob causam Marcellus et ipse Milo a Domitio praesidium imploraverunt. Sedebat eo tempore Cn. Pompeius ad aerarium, perturbatusque erat eodem illo clamore; itaque Domitio promisit se postero die cum praesidio descensurum, idque fecit. Qua re territi Clodiani silentio verba testium per biduum audiri passi sunt. Interrogaverunt eos M. Cicero et M. Marcellus et Milo ipse. Multi ex iis qui Bovillis habitabant testimonium dixerunt de eis quae ibi facta erant; coponem occisum, tabernam expugnatam, corpus Clodi in publicum extractum esse. Virgines quoque Albanae dixerunt mulierem ignotam venisse ad se quae Milonis mandato votum solveret, quod Clodius occisus esset. Ultimae testimonium dixerunt Sempronia, Tuditani filia, socrus P. Clodi, et uxor Fulvia, et fletu suo magnopere eos qui assistebant commoverunt. Dimisso circa horam decimam iudicio T. Munatius pro contione populum adhortatus est ut postero die frequens adesset et elabi Milonem non paterentur, iudiciumque et dolorem suum ostenderet euntibus ad tabellam ferendam. Postero die, qui fuit iudicii summus a. d. VII Idus Aprilis, clausae fuerunt tota urbe tabernae; praesidia in foro et circa omnis fori aditus Pompeius disposuit; ipse pro aerario, ut pridie, consedit saeptus delecta manu militum. Sortitio deinde iudicum a prima die facta est; post tantum silentium toto foro fuit quantum esse in aliquo foro posset. Tum intra horam secundam accusatores coeperunt dicere Appius maior et M. Antonius et P. Valerius Nepos. Usi sunt ex lege horis duabus.

Respondit his unus M. Cicero; et cum quibusdam placuisset ita defendi crimen, interfici Clodium pro re publica fuisse – quam formam M. Brutus secutus est in ea oratione quam pro Milone composuit et edidit quasi egisset – Ciceroni id non placuit ut, quisquis bono publico damnari, idem etiam occidi indemnatus posset. Itaque cum insidias Milonem Clodio fecisse posuissent accusatores, quia falsum id erat – nam forte illa rixa commissa fuerat – Cicero apprehendit et contra Clodium Miloni fecisse insidias disputavit, eoque tota oratio eius spectavit. Sed ita constitit ut diximus, nec utrius consilio pugnatum esse eo die, verum et forte occurrisse et ea rixa servorum ad eam denique caedem perventum. Notum tamen erat utrumque mortem alteri saepe minatum esse, et sicut suspectum Milonem maior quam Clodi familia faciebat, ita expeditior et paratior ad pugnam Clodianorum quam Milonis fuerat. Cicero cum inciperet dicere, exceptus est acclamatione Clodianorum, qui se continere ne metu quidem circumstantium militum potuerunt. Itaque non ea qua solitus erat constantia dixit. Manet autem illa quoque excepta eius oratio; scripsit vero hanc quam legimus ita perfecte ut iure prima haberi possit.


\subsection*{Commentarius}

Vers. a primo L \textit{Unum genus est adversum infestumque nobis} et cetera.

Ita ut in causae expositione diximus, Munatius Plancus pridie pro contione populum adhortatus erat ne pateretur elabi Milonem.

Ver. a primo CC \textit{Declarant huius ambusti tribuni plebis illae intermortuae contiones quibus cotidie meam potentiam invidiose criminabatur.}

T. Munatius Plancus et Q. Pompeius Rufus tribuni pl., de quibus in argumento huius orationis diximus, cum contra Milonem Scipioni et Hypsaeo studerent, contionati sunt eo ipso tempore plebemque in Milonem accenderunt quo propter Clodi corpus curia incensa est, nec prius destiterunt quam flamma eius incendii fugati sunt e contione. Erant enim tunc rostra non eo loco quo nunc sunt sed ad comitium, prope iuncta curiae. Ob hoc T. Munatium ambustum tribunum appellat; fuit autem paratus ad dicendum.

Paulo post: \textit{Cur igitur incendium curiae, oppugnationem aedium M. Lepidi, caedem hanc ipsam contra rem publicam senatus factam esse decrevit?}

Post biduum medium quam Clodius ocisus erat interrex primus proditus est M. Aemilius Lepidus. Non fuit autem moris ab eo qui primus interrex proditus erat comitia haberi. Sed Scipionis et Hypsaei factiones, quia recens invidia Milonis erat, cum contra ius postularent ut interrex ad comitia consulum creandorum descenderet, idque ipse non faceret, domum eius per omnes interregni dies – fuerunt autem ex more quinque – obsederunt. Deinde omni vi ianua expugnata et imagines maiorum deiecerunt et lectulum adversum uxoris eius Corneliae, cuius castitas pro exemplo habita est, fregerunt, itemque telas quae ex vetere more in atrio texebantur diruerunt. Post quae supervenit Milonis manus et ipsa postulans comitia; cuius adventus fuit saluti Lepido; in se enim ipsae conversae sunt factiones inimicae, atque ita oppugnatio domus interregis omissa est.

Paulo post: \textit{Quod si per furiosum illum tribunum pl. senatui quod sentiebat perficere licuisset, novam quaestionem nullam haberemus. Decernebat enim ut veteribus legibus, tantum modo extra ordinem, quaereretur. Divisa sententia est postulante nescio quo. – Sic reliqua auctoritas senatus empta intercessione sublata est.}

Quid sit dividere sententiam ut enarrandum sit vestra aetas, filii, facit.

Cum aliquis in dicenda sententia duas pluresve res complectitur, si non omnes eae probantur, postulatur ut dividatur, id est de rebus singulis referatur. Forsitan nunc hoc quoque velitis scire qui fuerit qui id postulaverit. Quod non fere adicitur; non enim ei qui hoc postulat oratione longa utendum ac ne consurgendum quidem utique est; multi enim sedentes hoc unum verbum pronuntiant Divide: quod cum auditum est, liberum est ei qui facit relationem dividere. Sed ego, ut curiosius aetati vestrae satisfaciam, Acta etiam totius illius temporis persecutus sum; in quibus cognovi pridie Kal. Mart. S.~C. esse factum, P. Clodi caedem et incendium curiae et oppugnationem aedium M. Lepidi contra rem p. factam; ultra relatum in Actis illo die nihil; postero die, id est Kal.~Mart., T. Munatium in contione exposuisse populo quae pridie acta erant in senatu; in qua contione haec dixit ad verbum: »Cum Hortensius dixisset ut extra ordinem quaereretur apud quaesitorem; existimaret futurum ut, cum pusillum dedisset dulcedinis, largiter acerbitatis devorarent; adverus hominem ingeniosum nostro ingenio usi sumus; invenimus Fufium, qui diceret Divide; reliquae parti sententiae ego et Sallustius intercessimus.« Haec contio, ut puto, explicat et quid senatus decernere voluerit, et quis divisionem postulaverit, et quis intercesserit et cur. Illud vos meminisse non dubito per Q. Fufium illo quoque tempore quo de incesto P. Clodi actum est factum ne a senatu asperius decerneretur.

De L. Domitio dicit: \textit{Dederas enim quam contemneres populares insanias iam ab adulescentia documenta maxima.}

Constantiam L. Domiti quam in quaestura praestitit significat. Nam eo tempore cum M. Manilius tribunus plebis subnixus libertinorum et servorum manu perditissimam legem ferret ut libertinis in omnibus tribubus suffragium esset, idque per tumultum ageret et clivum Captitolinum obsideret, discusserat perruperatque coetum Domitius ita ut multi Manilianorum occiderentur. Quo facto et plebem infimam offenderat et senatus magnam gratiam inierat.

\textit{Itaque illud Cassianum indicium in his personis valeat.}

L. Cassius fuit, sicut iam saepe diximus, summae vir severitatis. Is quotiens quaesitor iudicii alicuius esset in quo quaerebatur de homine occiso suadebat atque etiam praeibat iudicibus hoc quod Cicero nunc admonet, ut quaereretur cui bono fuisset perire eum de cuius morte quaeritur. Ob quam severitatem, quo tempore Sex.Peducaeus tribunus plebis criminatus est L. Metellum pontificem max. totumque collegium pontificum male iudicasse de incesto virginum Vestalium, quod unam modo Aemiliam damnaverat, absolverat autem duas Marciam et Liciniam, populus hunc Cassium creavit qui de eisdem virginibus quaereret. Isque et ultrasque eas et praeterea complures alias nimia etiam, ut existimatio est, asperitate usus damnavit.

\textit{Et aspexit me illis quidem oculis quibus tunc solebat cum omnibus omnia minabatur. Movet me quippe lumen curiae!}

Hic est Sex.Clodius quem in argumento huius orationis diximus corpus Clodi in curiam intulisse et ibi cremasse eoque incenso curiam conflagrasse; ideo lumen curiae dicit.

\textit{Quando illius postea sica illa quam a Catilina acceperat conquievit? Haec intenta nobis est, huic ego obici vos pro me passus non sum, haec insidiata Pompeio est.}

Haec intenta nobis est et obici vos pro me non sum passus, manifestum est pertinere ad id tempus quo post rogationem a P. Clodio in eum promulgatam urbe cessit. Qua re dicat insidiata Pompeio est fortassis quaeratis. Pisone et Gabinio coss. pulso Cicerone in exilium, cum III Idus Sextiles Pompeius in senatum venit, dicitur servo P. Clodi sica excidisse, eaque ad Gabinium consulem delata dictum est servo imperatum a P. Clodio ut Pompeius occideretur. Pompeius statim domum rediit et ex eo domi se tenuit. Obsessus est etiam a liberto Clodi Damione, ut ex Actis eius anni cognovi, in quibus XV Kal. Sept. L. Novius tribunus plebis, collega Clodi, cum Damio adversum L. Flavium praetorem appellaret tribunos et tribuni de appellatione cognoscerent, ita sententiam dixit: »Et si ab hoc apparitore P. Clodi vulneratus sum, et hominibus armatis praesidiis dispositis a re publica remotus Cn. Pompeius obsessusque est, cum appeller, non utar eius exemplo quem vitupero et iudicium tollam«, et reliqua de intercessione.

\textit{Haec viam Appiam monumentum nominis sui nece Papiri cruentavit.}

Pompeius post triumphum Mithridaticum Tigranis filium in catenis deposuerat apud Flavium senatorem; qui postea cum esset praetor eodem anno quo tribunus plebis Clodius, petiit ab eo Clodius super cenam ut Tigranem adduci iuberet ut eum videret. Adductum collocavit in convivio, dein Flavio non reddidit Tigranem; domum misit et habuit extra catenas nec repetenti Pompeio reddidit. Postea in navem deposuit, et cum profugeret ille, tempestate delatus est Antium. Inde ut deduceretur ad se, Clodius Sex. Clodium, de quo supra diximus, misit. Qui cum reduceret, Flavius quoque re cognita ad eripiendum Tigranem profectus est. Ad quartum lapidem ab urbe pugna facta est in qua multi ex utraque parte ceciderunt, plures tamen ex Flavi, inter quos et M. Papirius eques Romanus, publicanus, familiaris Pompeio. Flavius sine comite Romam vix perfugit.

\textit{Haec eadem longo intervallo conversa rursus est in me; nuper quidem, ut scitis, me ad Regiam paene confecit.}

Quo die periculum hoc adierit, ut Clodius eum ad Regiam paene confecerit, nusquam inveni; non tamen adducor ut putem Ciceronem mentitum, praesertim cum adiciat ut scitis. Sed videtur mihi loqui de eo die quo consulibus Domitio et Messala qui praecesserant eum annum cum haec oratio dicta est inter candidatorum Hypsaei et Milonis manus in via Sacra pugnatum est, multique ex Milonianis ex improviso ceciderunt. De cuius diei periculo suo ut putem loqui eum facit et locus pugnae – nam in Sacra via traditur commissa, in qua est Regia – et quod adsidue simul erant cum candidatis suffragatores, Milonis Cicero, Hypsaei Clodius.

\textit{Potuitne L. Caecili, iustissimi fortissimique praetoris, obpugnata domo?}

L. Caecilius Rufus de quo dicitur fuit praetor P. Lentulo Spinthere Q. Metello Nepote coss., quo anno Cicero restitutus est. Is cum faceret ludos Apollinares, ita infima coacta multitudo annonae caritate tumultuata est ut omnes qui in theatro spectandi causa consederant pellerentur. De oppugnata domo nusquam adhuc legi; Pompeius tamen cum defenderet Milonem apud populum, de vi accusante Clodio, obiecit ei, ut legimus apud Tironem libertum Ciceronis in libro IIII de vita eius, oppressum L. Caecilium praetorem.

Paulo post: \textit{At quo die? quo, ut ante dixi, fuit insanissima contio ab ipsius mercennario tribuno plebis concitata.}

Hoc significat eo die quo Clodius occisus est contionatum esse mercennarium eius tribunum plebis. Sunt autem contionati eo die, ut ex Actis apparet, C. Sallustius et Q. Pompeius, utrique et inimici Milonis et satis inquieti. Sed videtur mihi Q. Pompeium significare; nam eius seditiosior fuit contio.

\textit{Dixit C. Causinius Schola Interamnanus, familiarissimus et idem comes Clodi, P. Clodium illo die in Albano mansurum fuisse.}

Hic fuit Causinius apud quem Clodius mansisse Interamnae videri volebat qua nocte deprehensus est in Caesaris domo, cum ibi in operto virgines pro populo Romano sacra facerent.

Paulo post: \textit{Scitis, iudices, fuisse qui in hac rogatione suadenda diceret Milonis manu caedem esse factam, consilio vero maioris alicuius. Me videlicet latronem et sicarium abiecti homines ac perditi describebant.}

Q. Pompeius Rufus et C. Sallustius tribuni fuerunt quos significat. Hi enim primi de ea lege ferenda populum hortati sunt et dixerunt a manu Milonis occisum esse Clodium et cetera.

\textit{Atqui ut illi nocturnus adventus vitandus fuit, sic Miloni, cum insidiator esset, si illum ad urbem noctu accessurum sciebat, subsidendum} et cetera.

Via Appia est prope urbem monumentum Basili qui locus latrociniis fuit perquam infamis, quod ex aliis quoque multis intellegi potest.

\textit{Comites Graeculi quocumque ibat, etiam cum in castra Etrusca properabat.}

Saepe obiecit Clodio Cicero socium eum coniurationis Catilinae fuisse; quam rem nunc quoque reticens ostendit. Fuerat enim opinio, ut Catilina ex urbe profugerat in castra Manli centurionis qui tum in Etruria ad Faesulas exercitum ei comparabat, Clodium subsequi eum voluisse et coepisse, tum dein mutato consilio in urbem redisse.

\textit{Non iam hoc Clodianum crimen timemus, sed tuas, Cn. Pompei – te enim appello, et ea voce ut me exaudire possis – tuas, inquam, suspiciones perhorrescimus.}

Diximus in argumento orationis huius Cn. Pompeium simulasse timorem, seu plane timuisse Milonem, et ideo ne domi quidem suae sed in hortis superioribus ante iudicium mansisse, ita ut villam quoque praesidio militum circumdaret. Q. Pompeius Rufus tribunus plebis, qui fuerat familiarissimus omnium P. Clodio et sectam illam sequi se palam profitebatur, dixerat in contione paucis post diebus quam Clodius erat occisus: »Milo dedit quem in curia cremaretis; dabit quem in Capitolio sepeliatis.« In eadem contione idem dixerat – habuit enim eam a. d. VIII Kal. Febr. – cum Milo pridie, id est VIIII Kal. Febr., venire ad Pompeium in hortos eius voluisset, Pompeium ei per hominem propinquum misisse nuntium ne ad se veniret. Prius etiam quam Pompeius ter consul crearetur, tres tribuni, Q. Pompeius Rufus, C. Sallustius Crispus, T. Munatius Plancus, cum cotidianis contionibus suis magnam invidiam Miloni propter occisum Clodium excitarent, produxerant ad populum Cn. Pompeium et ab eo quaesierant num ad eum delatum esset illius quoque rei indicium, suae vitae insidiari Milonem. Responderat Pompeius: Licinium quendam de plebe sacrificulum qui solitus esset familias purgare ad se detulisse servos quosdam Milonis itemque libertos comparatos esse ad caedem suam, nomina quoque servorum edidisse; se ad Milonem misisse utrum in potestate sua haberet; a Milone responsum esse, ex iis servis quos nominasset partim neminem se umquam habuisse, partim manumisisse; dein, cum Licinium apud se haberet, . . . Lucium quendam de plebe ad corrumpendum indicem venisse; qua re cognita in vincla eum publica esse coniectum. Decreverat enim senatus ut cum interrege et tribunis plebis Pompeius daret operam ne quid res publica detrimenti caperet. Ob has suspiciones Pompeius in superioribus hortis se continuerat; deinde ex S.~C. dilectu per Italiam habito cum redisset, venientem ad se Milonem unum omnium non admiserat. Item cum senatus in porticu Pompeii haberetur ut Pompeius posset interesse, unum eum excuti prius quam in senatum intraret iusserat. Hae sunt suspiciones quas se dicit pertimescere.

\textit{Quid enim minus illo dignum quam cogere ut vos eum condemnetis in quem animadvertere ipse et more maiorum et suo iure posset? sed praesidio esse} et cetera.

Idem T. Munatius Plancus, ut saepe diximus, post audita et obsignata testium verba dimissosque interim iudices vocata contione cohortatus erat populum ut clausis tabernis postero die ad iudicium adesset nec pateretur elabi Milonem.

\textit{Incidebantur iam domi leges quae nos servis nostris addicerent.}

Significasse iam puto nos fuisse inter leges P. Clodi quas ferre proposuerat eam quoque qua libertini, qui non plus quam in IIII tribubus suffragium ferebant, possent in rusticis quoque tribubus, quae propriae ingenuorum sunt, ferre.

\textit{Senatus, credo, praetorem eum circumscripsisset. Ne cum solebat quidem id facere, in privato eodem hoc aliquid profecerat.}

Significat id tempus quo P. Clodius, cum adhuc quaestor designatus esset, deprensus est, cum intrasset eo ubi sacrificium pro populo Romano fiebat. Quod factum notatum erat . . . S.~C., decretumque ut extra ordinem de ea re iudicium fieret.

Ver. a novis. CLX

Quo loco inducit loquentem Milonem cum bonarum partium hominibus de meritis suis: \textit{Plebem et infimam multitudinem, quae P. Clodio duce fortunis vestris imminebat, eam, quo tutior esset vestra vita, se fecisse commemorat ut non modo virtute flecteret, sed etiam tribus suis patrimoniis deleniret.}

Puto iam supra esse dictum Milonem ex familia fuisse Papia, deinde adoptatum esse ab T.Annio, avo suo materno. Tertium patrimonium videtur significare matris; aliud enim quod fuerit non inveni.

\subsection*{(Eventus)}

Peracta utrimque causa singuli quinos accusator et reus senatores, totidem equites et tribunos aerarios reiecerunt, ita ut unus et L sententias tulerint. Senatores condemnaverunt XII, absolverunt VI; equites condemnaverunt XIII, absolverunt IIII; tribuni aerarii condemnaverunt XIII, absolverunt III. Videbantur non ignorasse iudices inscio Milone initio vulneratum esse Clodium, sed compererant, post quam vulneratus esset, iussu Milonis occisum. Fuerunt qui crederent M. Catonis sententia eum esse absolutum; nam et bene cum re publica actum esse morte P. Clodi non dissimulaverat et studebat in petitione consulatus Miloni et reo adfuerat. Nominaverat quoque eum Cicero praesentem et testatus erat audisse eum a M. Favonio ante diem tertium quam facta caedes erat, Clodium dixisse periturum esse eo triduo Milonem . . . Sed Milonis quoque notam audaciam removeri a re publica utile visum est. Scire tamen nemo umquam potuit utram sententiam tulisset. Damnatum autem opera maxime Appi Claudi pronuntiatum est. 

Milo postero die factus reus ambitus apud Manlium Torquatum absens damnatus est. Illa quoque lege accusator fuit eius Appius Claudius, et cum ei praemium lege daretur, negavit se eo uti. Subscripserunt ei in ambitus iudicio P. Valerius Leo et Cn. Domitius Cn. f. Post paucos dies quoque Milo apud M. Favonium quaesitorem de sodaliciis damnatus est accusante P. Fulvio Nerato, cui e lege praemium datum est. Deinde apud L. Fabium quaesitorem iterum absens damnatus est de vi; accusavit L. Cornificius et Q. Patulcius. Milo in exilium Massiliam intra paucissimos dies profectus est. Bona eius propter aeris alieni magnitudinem semuncia venierunt.

Post Milonem eadem lege Pompeia primus est accusatus M. Saufeius M. f. qui dux fuerat in expugnanda taberna Bovillis et Clodio occidendo. Accusaverunt eum L. Cassius, L. Fulcinius C. f., C. Valerius; defenderunt M. Cicero, M. Caelius, obtinueruntque ut una sententia absolveretur. Condemnaverunt senatores X, absolverunt VIII; condemnaverunt equites Romani VIIII, absolverunt VIII; sed ex tribunis aerariis X absolverunt, VI damnaverunt; manifestumque odium Clodi saluti Saufeio fuit, cum eius vel peior causa quam Milonis fuisset, quod aperte dux fuerat expugnandae tabernae. Repetitus deinde post paucos dies apud C. Considium quaestiorem est lege Plautia de vi, subscriptione ea quod loca edita occupasset et cum telo fuisset; nam dux fuerat operarum Milonis. Accusaverunt C. Fidius, Cn. Aponius Cn. f., M. Seius . . . Sex.f.; defenderunt M. Cicero, M. Terentius Varro Gibba. Absolutus est sententiis plenius quam prius; graves habuit XVIIII, absolutorias duas et XXX; sed e contrario hoc ac priore iudicio accidit; equites enim ac senatores eum absolverunt, tribuni aerarii damnaverunt.

Sex. autem Clodius quo auctore corpus Clodi in curiam illatum fuit accusantibus C. Caesennio Philone, M. Alfidio, defendente T.Flacconio, magno consensu damnatus est, sententiis sex et XL; absolutorias quinque omnino habuit, duas senatorum, tres equitum.

Multi praeterea et praesentes et cum citati non respondissent damnati sunt, ex quibus maxima pars fuit Clodianorum.

