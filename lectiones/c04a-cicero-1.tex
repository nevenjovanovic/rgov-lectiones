%\section*{O autoru}

\section*{De claris oratoribus liber, qui dicitur Brutus (46 a. Chr. n)}

\subsection*{Summarium}

In praefatione libri Cicero primum Q.~Hortensii mortem deplorat, additis consolationis causis, c. 1. 2. deinde occasionem huius dialogi exponit. Nimirum ad eum M.~Brutus cum T.~Pomponio Attico venerant, eumque rogaverant, ut, quem nuper in Tusculano, solo audiente Attico, inchoasset sermonem de claris oratoribus, nunc utrique plenum exponeret. c. 3-5. 

Sequitur ipse dialogus, in quo Cicero, postquam, unde ductus esset sermo cum Pomponio in Tusculano habitus, exposuerat c. 6., primum breviter Graecos oratores artisque rhetoricae scriptores recenset, c. 7-13; deinde ad Romanos accedit, in quibus primum antiquiores, de quorum eloquentia nihil certi habebat dicere, enumerat, ut L.~Brutum, M.~Valerium Maximum, L.~Valerium Potitum, Ap.~Claudium, C.~Fabricium, Ti.~Coruncanium , M'.~Curium, M. Popillium, C.~Flaminium, Q.~Fabium Maximum Verrucosum, Q.~Metellum. c. 14.


Sequuntur ii, de quibus certior eloquentiae memoria constabat; in hisque primus M.~Cornelius Cethegus c. 15. M.~Cato Censorius, cuius orationes et oratoriae laudes cum Lysia comparantur, et cur hic felicior sit ab omni laude, exponitur. c. 16-19. Post Catonem nominantur quidam, qui cum eo grandiores natu vixerunt, C.~Flaminius, C.~Varro, Q.~Maximus, Q.~Metellus, P.~Lentulus, P.~Crassus, P.~Scipio Africanus, eiusdem filius, Sex.~Aelius; deinde minores aetate C.~Sulpicius Gallus, Ti.~Gracchus, P.~F., P.~Scipio Nasica Corculum, Q.\ et M.\ Nobiliores, T.~Annius Luscus, L.~Paulus Macedonicus, Africani Iunioris pater; c. 20. Tum A.~Albinus, Ser.~Fabius Pictor, Q.~Fabius Labeo, Q.~Metellus, L.~Cotta, C.~Laelius, P.~Africanus Minor, Ser.~Galba. c. 21. Inprimis de horum trium, Laelii, Africani et Galbae laudibus dicitur, c. 22-24.

Post illos nominantur L.\ et Sp.~Mummii fratres, Sp.~Albinus, L.\ et C.~Aurelii Orestae, P.~Popillius, eiusque filius, Caius. Porro C.~Tuditanus, M.~Octavius, praecipue autem M.~Aemilius Lepidus Porcina. Deinde Q.~Pompeius, L.~Cassius, M.~Antius Briso, duo Caepiones, Cn.\ et Q.\ c.\ 25.

Sequuntur P.~Crassus, valde probatus orator, eiusque aetati iuncti duo C.~Fannii, C.\ et M.\ filii; Q.~Scaevola augur, iuris civilis intelligentia, non oratoria laude clarus; L.~Coelius Antipater, c.\ 26. 

Praecipue autem eminent Tib.~Gracchus Caiusque Carbo. c. 27. Minorem vero eloquentiae laudem adepti D.~Brutus, M.\ F., Q.~Maximus, Pauli nepos, P.~Scipio Nasica Serapio, P.~Lentulus, L.~Furius Philus; P.~Scaevola, M'.~Manilius, M.~Fulvius Flaccus, C.~Cato, P.~Decius, M.~Drusus C.~F.\ eiusque frater C.~Drusus; M.~Pennus et T.~Flamininus. c. 98. 

His adiunguntur M.~Scaurus c. 29. P.~Rutilius c. 30. Q.~Aelius Tubero, omninoque Stoici oratores c. 31. C.~Curio, illustris orator, c. 32. et praestantissimus ingenio, studio, doctrina C.~Gracchus. c. 33. Huic successit aetati C.~Galba, P.~Scipio, L.~Bestia, C.~Licinius Nerva, C.~Fimbria, C.~Sextius Calvinus, M.~Brutus, accusator ille vehemens et molestus, L.~Caesulenus, T.~Albucius. c. 34.

Porro Q.~Catulus, pater et filius, Q.~Metellus Numidicus, M.~Silanus, M.~Aurelius Scaurus, A.~Albinus, Q.~Caepio, C.\ et L.~Memmii, Sp.~Thorius, M.~Marcellus, P.~Lentulus, L.~Cotta. c. 35.

Sequuntur summi oratores L.~Licinius Crassus, et M.~Antonius, de quibus diligenter agitur c. 36-44. interiecto Q.~Scaevola, qui cum Ser.~Sulpicio Rufo, Ciceronis aequali, comparatur. c. 41.

Tum recensentur Cn.~Domitius, C.~Coelius, M.~Herennius, C.~Clodius, C.~Titius, L.~Afranius, Q.~Varro, M.~Gratidius. c. 45.

His adiunguntur nonnulli ex sociis et Latinis oratoribus, Q.~Vettius Vettianus, Q.\ et D.~Valerii Sorani, C.~Rusticellus Bononiensis, T.~Betucius Barrus Asculanus, et prior aetate L.~Papirius Fregellanus. c. 46.

Hos excipiunt e Romanis L.~Philippus, orator Crasso et Antonio, sed longo intervallo, proximus, et eius aetati fere coniuncti, L.~Gellius, D.~Brutus, L.~Scipio, Cn.~Pompeius, Sexti F., M.~Brutus, C.~Bellienus, Cn.~Octavius, c. 47. Tum C.~Iulius L.~F.\ Caesar, P.~Cethegus, Q.~Lucretius Vispillo, Ofella, T.~Annius Velina, T.~Iuvencius, P.~Orbius, T.~Aufidius eiusque frater M.~Virgilius, P.~Magius, Q.~Sertorius, C.~Gorgonius, T.~Iunius. c. 48.

Sequuntur ii, cum quibus Cicero magis iam vixit et viguit. Inter quos facile primas tulisse ait, cum suo tum omnium iudicio C.~Cottam et P.~Sulpicium, c. 49. idque eum in hanc disputationem ducit, ut, num recte populus de oratoribus iudicet, disquirat. c. 49. — 54. Tum laudes Cottae et Sulpicii fusius exponit c. 55. 56.; agitque porro de Pomponio c. 57.; de Curione c. 58. — 60., ubi simul de domestica institutione ad linguae puritatem multum valente disputat; porro de C.~Carbone; Q.~Vario, L.~Fufio, compluribus aliis vel e numero oratorum exclusis, vel brevissime memoratis. c. 61. 62.

Sequitur P.~Antistius c. 63. L.~Sisenna c. 64. et Hortensio, de quo postea pluribus verbis agit, memorato, agitur de M.~Crasso, C.~Fimbria, Cn.~et P.~Lentulis c. 66. de M.~Pisone, P.~Murena, C.~Censorino, L.~Turio , C.~Macro c.~67. de C.~Pisone, L.~Torquato, Cn.~Pompeio M., D.~Silano, Q.~Pompeio, P.~Autronio, L.~Octavio, C.~Staleno c. 68.; porro de C. et L.~Caepasiis fratribus, de Cosconio et Arrio c. 69. His adiunguntur L.~Torquatus, M.~Messala, Celer et Nepos Metelli, Cn.~Lentulus Marcellimus et C.~Memmius. c. 70. Tum rogatus a Bruto Cicero, ut de Caesare et M.~Marcello, utroque vivo, iudicium suum exponat, primum Marcellum ipsum laudat, Caesaris autem laudes fere ad Atticum remittit, ita tamen, ut iis ipse non minus ! quam Brutus assentiatur. c. 71-75. Post illos recensentur C.~Sicinius, C.~Visellius Varro, L.~Torquatus, C.~Triarius c.~76. M.~Bibulus, Ap.~Claudius, L.~Domitius, P.~et L.~Lentuli, T.~Postumius c. 77. P.~Cominius, T.~Attius, C.~Piso c. 78. Sequuntur M.~Coelius c.\ 79. M.~Calidius c. 80. C.~Curio, P.~Crassus, C.~Licinius Calvus c.\ 81. 82. a cuius exilitate sumit occasionem, de Attico dicendi genere disputandi c. 83. qua degressione laudata Atticus Ciceronem pervellit, quod nimiis laudibus multos Romanos extulerit oratores c. 83—87. Sequuntur iam Q.~Hortensii laudes c. 88., quibuscum coniungit Cicero de suis studiis laboribusque forensibus narrationem, c. 88-94. Tum quaestioni, cur Hortensius magis adolescens quam provecta aetate orator floruerit, respondetur. c.\ 95. 96. Epilogus hortatur Brutum, ut, quamquam iniqua nunc sint reipublicae tempora et oratorum studiis, tamen eloquentiae laudem tueri, seque ex turba patronorum eripere velit. c.\ 97.

\subsection*{Brutus 21–36}

``Scio'', inquit, ``ab isto initio tractum esse sermonem teque Bruti dolentem vicem quasi deflevisse iudiciorum vastitatem et fori.''

``Feci'', inquam, ``istuc quidem et saepe facio. nam mihi, Brute, in te intuenti crebro in mentem venit vereri, ecquodnam curriculum aliquando sit habitura tua et natura admirabilis et exquisita doctrina et singularis industria. cum enim in maxumis causis versatus esses et cum tibi aetas nostra iam cederet fascisque submitteret, subito in civitate cum alia ceciderunt tum etiam ea ipsa, de qua disputare ordimur, eloquentia obmutuit.''

Tum ille: ``ceterarum rerum causa'', inquit, ``istuc et doleo et dolendum puto; dicendi autem me non tam fructus et gloria quam studium ipsum exercitatioque delectat: quod mihi nulla res eripiet te praesertim tam studiosum et * * * . dicere enim bene nemo potest nisi qui prudenter intellegit; quare qui eloquentiae verae dat operam, dat prudentiae, qua ne maxumis quidem in bellis aequo animo carere quisquam potest.''

``Praeclare'', inquam, ``Brute, dicis eoque magis ista dicendi laude delector, quod cetera, quae sunt quondam habita in civitate pulcherrima, nemo est tam humilis qui se non aut posse adipisci aut adeptum putet; eloquentem neminem video factum esse victoria. sed quo facilius sermo explicetur, sedentes, si videtur, agamus.'' Cum idem placuisset illis, tum in pratulo propter Platonis statuam consedimus.

Hic ego: ``laudare igitur eloquentiam et quanta vis sit eius expromere quantamque eis, qui sint eam consecuti, dignitatem afferat, neque propositum nobis est hoc loco neque necessarium. hoc vero sine ulla dubitatione confirmaverim, sive illa arte pariatur aliqua sive exercitatione quadam sive natura, rem unam esse omnium difficillumam. quibus enim ex quinque rebus constare dicitur, earum una quaeque est ars ipsa magna per sese. quare quinque artium concursus maxumarum quantam vim quantamque difficultatem habeat existimari potest.

Testis est Graecia, quae cum eloquentiae studio sit incensa iamdiuque excellat in ea praestetque ceteris, tamen omnis artes vetustiores habet et multo ante non inventas solum, sed etiam perfectas, quam haec est a Graecis elaborata dicendi vis atque copia. in quam cum intueor, maxime mihi occurrunt, Attice, et quasi lucent Athenae tuae, qua in urbe primum se orator extulit primumque etiam monumentis et litteris oratio est coepta mandari.

Tamen ante Periclem, cuius scripta quaedam feruntur, et Thucydidem, qui non nascentibus Athenis sed iam adultis fuerunt, littera nulla est, quae quidem ornatum aliquem habeat et oratoris esse videatur. Quamquam opinio est et eum, qui multis annis ante hos fuerit, Pisistratum et paulo seniorem etiam Solonem posteaque Clisthenem multum, ut temporibus illis, valuisse dicendo.

Post hanc aetatem aliquot annis, ut ex Attici monumentis potest perspici, Themistocles fuit, quem constat cum prudentia tum etiam eloquentia praestitisse; post Pericles, qui cum floreret omni genere virtutis, hac tamen fuit laude clarissumus. Cleonem etiam temporibus illis turbulentum illum quidem civem, sed tamen eloquentem constat fuisse.

Huic aetati suppares Alcibiades Critias Theramenes; quibus temporibus quod dicendi genus viguerit ex Thucydidi scriptis, qui ipse tum fuit, intellegi maxume potest. Grandes erant verbis, crebri sententiis, compressione rerum breves et ob eam ipsam causam interdum subobscuri.

Sed ut intellectum est quantam vim haberet accurata et facta quodam modo oratio, tum etiam magistri dicendi multi subito exstiterunt. Tum Leontinus Gorgias, Thrasymachus Calchedonius, Protagoras Abderites, Prodicus Ceius, Hippias Eleius in honore magno fuit; aliique multi temporibus eisdem docere se profitebantur adrogantibus sane verbis, quemadmodum causa inferior – ita enim loquebantur – dicendo fieri superior posset.

Exstitit igitur iam senibus illis quos paulo ante diximus Isocrates, cuius domus cunctae Graeciae quasi ludus quidam patuit atque officina dicendi; magnus orator et perfectus magister, quamquam forensi luce caruit intraque parietes aluit eam gloriam, quam nemo meo quidem iudicio est postea consecutus. Is et ipse scripsit multa praeclare et docuit alios; et cum cetera melius quam superiores, tum primus intellexit etiam in soluta oratione, dum versum effugeres, modum tamen et numerum quendam oportere servari.

Ante hunc enim verborum quasi structura et quaedam ad numerum conclusio nulla erat; aut, si quando erat, non apparebat eam dedita opera esse quaesitam – quae forsitan laus sit – ; verum tamen natura magis tum casuque nonnunquam, quam aut ratione aliqua aut ulla observatione fiebat.

Ipsa enim natura circumscriptione quadam verborum comprehendit concluditque sententiam, quae cum aptis constricta verbis est, cadit etiam plerumque numerose. nam et aures ipsae quid plenum, quid inane sit iudicant et spiritu quasi necessitate aliqua verborum comprensio terminatur; in quo non modo defici, sed etiam laborare turpe est.

Tum fuit Lysias ipse quidem in causis forensibus non versatus, sed egregie subtilis scriptor atque elegans, quem iam prope audeas oratorem perfectum dicere. nam plane quidem perfectum et quoi nihil admodum desit Demosthenem facile dixeris. nihil acute inveniri potuit in eis causis quas scripsit, nihil, ut ita dicam, subdole, nihil versute, quod ille non viderit; nihil subtiliter dici, nihil presse, nihil enucleate, quo fieri possit aliquid limatius; nihil contra grande, nihil incitatum, nihil ornatum vel verborum gravitate vel sententiarum, quo quicquam esset elatius.

Huic Hyperides proxumus et Aeschines fuit et Lycurgus et Dinarchus et is, cuius nulla exstant scripta, Demades aliique plures. haec enim aetas effudit hanc copiam; et, ut opinio mea fert, sucus ille et sanguis incorruptus usque ad hanc aetatem oratorum fuit, in qua naturalis inesset, non fucatus nitor.

(\dots)''

\subsection*{Brutus 307-317}

``(\dots) Occiderat Sulpicius illo anno tresque proxumo trium aetatum oratores erant crudelissume interfecti, Q.~Catulus M.~Antonius C.~Iulius. eodem anno etiam Moloni Rhodio Romae dedimus operam et actori summo causarum et magistro. haec etsi videntur esse a proposita ratione diversa, tamen idcirco a me proferuntur, ut nostrum cursum perspicere, quoniam voluisti, Brute, possis — nam Attico haec nota sunt — et videre quem ad modum simus in spatio Q.~Hortensium ipsius vestigiis persecuti.

Triennium fere fuit urbs sine armis; sed oratorum aut interitu aut discessu aut fuga — nam aberant etiam adulescentes M.~Crassus et Lentuli duo — primas in causis agebat Hortensius, magis magisque cotidie probabatur Antistius, Piso saepe dicebat, minus saepe Pomponius, raro Carbo, semel aut iterum Philippus. at vero ego hoc tempore omni noctes et dies in omnium doctrinarum meditatione versabar.

Eram cum Stoico Diodoto, qui cum habitavisset apud me secumque vixisset, nuper est domi meae mortuus. a quo cum in aliis rebus tum studiosissime in dialectica exercebar, quae quasi contracta et astricta eloquentia putanda est; sine qua etiam tu, Brute, iudicavisti te illam iustam eloquentiam, quam dialecticam esse dilatatam putant, consequi non posse. huic ego doctori et eius artibus variis atque multis ita eram tamen deditus ut ab exercitationibus oratoriis nullus dies vacuus esset.

Commentabar declamitans — sic enim nunc loquuntur — saepe cum M.~Pisone et cum Q.~Pompeio aut cum aliquo cotidie, idque faciebam multum etiam Latine sed Graece saepius, vel quod Graeca oratio plura ornamenta suppeditans consuetudinem similiter Latine dicendi adferebat, vel quod a Graecis summis doctoribus, nisi Graece dicerem, neque corrigi possem neque doceri.

Tumultus interim recuperanda re publica et crudelis interitus oratorum trium, Scaevolae Carbonis Antisti, reditus Cottae Curionis Crassi Lentulorum Pompei; leges et iudicia constituta, recuperata res publica; ex numero autem oratorum Pomponius Censorinus Murena sublati. tum primum nos ad causas et privatas et publicas adire coepimus, non ut in foro disceremus, quod plerique fecerunt, sed ut, quantum nos efficere potuissemus, docti in forum veniremus.

Eodem tempore Moloni dedimus operam; dictatore enim Sulla legatus ad senatum de Rhodiorum praemiis venerat. itaque prima causa publica pro Sex.~Roscio dicta tantum commendationis habuit, ut non ulla esset quae non digna nostro patrocinio videretur. deinceps inde multae, quas nos diligenter elaboratas et tamquam elucubratas adferebamus.

Nunc quoniam totum me non naevo aliquo aut crepundiis sed corpore omni videris velle cognoscere, complectar nonnulla etiam quae fortasse videantur minus necessaria. erat eo tempore in nobis summa gracilitas et infirmitas corporis, procerum et tenue collum: qui habitus et quae figura non procul abesse putatur a vitae periculo, si accedit labor et laterum magna contentio. eoque magis hoc eos quibus eram carus commovebat, quod omnia sine remissione, sine varietate, vi summa vocis et totius corporis contentione dicebam.

Itaque cum me et amici et medici hortarentur ut causas agere desisterem, quodvis potius periculum mihi adeundum quam a sperata dicendi gloria discedendum putavi. sed cum censerem remissione et moderatione vocis et commutato genere dicendi me et periculum vitare posse et temperatius dicere, ut consuetudinem dicendi mutarem, ea causa mihi in Asiam proficiscendi fuit. itaque cum essem biennium versatus in causis et iam in foro celebratum meum nomen esset, Roma sum profectus.

Cum venissem Athenas, sex menses cum Antiocho veteris Academiae nobilissumo et prudentissumo philosopho fui studiumque philosophiae numquam intermissum a primaque adulescentia cultum et semper auctum hoc rursus summo auctore et doctore renovavi. eodem tamen tempore Athenis apud Demetrium Syrum veterem et non ignobilem dicendi magistrum studiose exerceri solebam. post a me Asia tota peragrata est cum summis quidem oratoribus, quibuscum exercebar ipsis lubentibus; quorum erat princeps Menippus Stratonicensis meo iudicio tota Asia illis temporibus disertissimus; et, si nihil habere molestiarum nec ineptiarum Atticorum est, hic orator in illis numerari recte potest.

adsiduissime autem mecum fuit Dionysius Magnes; erat etiam Aeschylus Cnidius, Adramyttenus Xenocles. hi tum in Asia rhetorum principes numerabantur. quibus non contentus Rhodum veni meque ad eundem quem Romae audiveram Molonem adplicavi cum actorem in veris causis scriptoremque praestantem tum in notandis animadvertendisque vitiis et instituendo docendoque prudentissimum. is dedit operam, si modo id consequi potuit, ut nimis redundantis nos et supra fluentis iuvenili quadam dicendi impunitate et licentia reprimeret et quasi extra ripas diffluentis coerceret. ita recepi me biennio post non modo exercitatior sed prope mutatus. nam et contentio nimia vocis resederat et quasi deferverat oratio lateribusque vires et corpori mediocris habitus accesserat.

Duo tum excellebant oratores qui me imitandi cupiditate incitarent, Cotta et Hortensius; quorum alter remissus et lenis et propriis verbis comprendens solute et facile sententiam, alter ornatus, acer et non talis qualem tu eum, Brute, iam deflorescentem cognovisti, sed verborum et actionis genere commotior. itaque cum Hortensio mihi magis arbitrabar rem esse, quod et dicendi ardore eram propior et aetate coniunctior. etenim videram in isdem causis, ut pro M.~Canuleio, pro Cn.~Dolabella consulari, cum Cotta princeps adhibitus esset, priores tamen agere partis Hortensium. acrem enim oratorem, incensum et agentem et canorum concursus hominum forique strepitus desiderat. (\dots)''
