%\section*{O autoru}

\section*{C.\ Plinii Secundi consulis Panegyricus \\Nervae Traiano Augusto dictus (100-103)}


\subsection*{Synopsis panegyrici}
\begin{quotation}
{\noindent
In Exordio primum reddit rationem cur initio Deos precetur: quod haec instituta sit a majoribus consuetudo, quam nemini ait potius quam sibi observandum, cap. I. Quia consulem decet religio, quia laudat optimum Principem, qualis Deorum munus est, cap. II. Quia inprimis Trajanus a Diis datus, cujus adoptio ante Jovis pulvinar in Capitolio facta. Deinde ostendit hanc laudationem non tam ut Principi serviliter obsequatur a se susceptam, quam ut senatusconsulto pareat; utque boni principes, quae facere debeant, in laudatis Trajani virtutibus recognoscant. Quare abesse oportere adulationis suspicionem. Haeret in commendatione ejus humanitatis, cap. III. Huiusque exordium. Distributio videtur haec esse: I. Vita Trajani publica. II. Vita ejusdem privata laudatur, cap. IV. 

I. Pars. Universas Trajani laudes primum ponit sub uno aspectu, quas postea sigillatim persequatur. Adoptatur a Nerva, cap. V. Quam gratus erga eum fuerit, cap. X. Quam metuendus esset hostibus, cap. XII. Quam carus et venerabilis militibus suis, cap. XIII. Traducta in bello prima ejus pueritia, cap. XIV. Jam tum signa dedit victoriarum ab eo referendarum. Plinius quasi vaticinans triumphum describit, cap. XV. Trajanus disciplinam militarem emendat, cap. XVIII. Romam redit, cap. XX. Patris patriae appellationem accipit a senatu, cap. XXI. Describitur placidus ejus reditus, cap. XXII. Romam redux primum tributa remittit: congiaria, donativa, alimenta dat, cap. XXV. Ubertatem in Urbem invehit, cap. XXVIII. Inde Aegypti sterilitati succurrit, cap. XXX. Ludos dat populo, cap. XXXIII. Delatores punit, cap. XXXIV. Vicesimale tributum temperat, cap. XXXVII. Testamentis securitatem inducit, cap. XLIII. Bonis et maxime nobilitati favet, cap. XLIV. Censoris nomen recusat, cap. XLV. Literarum studia promovet: honorem dicendi magistris habet. Ejus comitas et humanitas, cap. XLVII. In publicis aedificiis ejus magnificentia, domi frugalitas, cap. L. Tanto major ceteris imperatoribus, quanto illis modestior, cap. LII. Secundum consulatum admittit sub Nerva, cap. LVI. Quo mortuo, jam sui juris factus, tertium recusat, cap. LVII. Exoratus tandem a republica eum suscipit anno sequenti, servatis tamen, quasi in privati hominis electione, comitiorum legibus, cap. LIX. In tertio consulatu nobiles vires promovet ad magistratus, maxime quos provinciae commendarent, cap. LXIX. Quam fuerit comis in candidatos, in cives, cap. LXXI. Quas acclamationes modestia meruerit, cap. LXXIV. Assiduus in foro, ibi omnes boni consulis partes agit, cap. LXXVI. Rogatur, ut quartum constilatum accipiat, cap. LXXVII. Quicquid supererat ab reip. rebus otii, non luxu aut desidia consumit, sed venatione ceterisque corporis exercitationibus impendit, cap. LXXXI. 

II. Pars. Hactenus qualis in imperio fuerit : nunc qualis in privata vita. Laus Plotinae Trajani conjugis, Marcianae sororis, quas exemplo suo instituit, cap. LXXXIV. Quanta esset in colendis amicitiis ejus constantia, cap. LXXXV. Quanta ipsius procuratorum integritas, cap. LXXXVII. Neque libertis plus quam par esset tribuit: ob quae omnia merito dicitur optimus a senatu, cap. LXXXVIII.

In fine compellat Nervam et Trajanum patrem; quibus de filii virtutibus gratulatur, cap. LXXXIX. Senatores; a quibus dicendi munus sibi impositum erat, cap. XC. Trajanum Imperatorem; cui et pro se et pro collega gratias agit, cap. XCI. Custodes imperii Deos, praecipue Capitolinum Jovem, quem orat pro aeternitate imperii et Principis, cap. XCIV. Iterum senatores, quibus obsequium et reverentiam pollicetur, cap. XCV. 
}
\end{quotation}



\subsection*{II–VII}

Equidem non Consuli modo, sed omnibus civibus enitendum reor, ne quid de Principe nostro ita dicant, ut idem illud de alio dici potuisse videatur. Quare abeant ac recedant voces illae, quas metus exprimebat: nihil, quale ante, dicamus; nihil enim, quale antea, patimur: nec eadem de principe [palam], quae prius, praedicemus; neque enim eadem secreto loquimur, quae prius. Discernatur orationibus nostris diversitas temporum, et ex ipso genere gratiarum agendarum intelligatur, cui, quando sint actae. Nusquam ut deo, nusquam ut numini blandiamur: non enim de tyranno, sed de cive; non de domino, sed de parente loquimur. Unum ille se ex nobis, et hoc magis excellit atque eminet, quod unum ex nobis putat; nec minus hominem se, quam hominibus praeesse meminit. Intelligamus ergo bona nostra, dignosque nos illis usu probemus, atque identidem cogitemus, quam sit indignum, si maius principibus praestemus obsequium, qui servitute civium, quam qui libertate laetantur. 

Et populus quidem Romanus dilectum principum servat, quantoque paullo ante concentu formosum alium, hunc fortissimum personat; quibusque aliquando clamoribus gestum alterius et vocem, huius pietatem, abstinentiam, mansuetudinem laudat. Quid nos ipsi? divinitatem principis nostri, an humanitatem, temperantiam, facilitatem, ut amor et gaudium tulit, celebrare universi solemus? Iam quid tam civile, tam senatorium, quam illud additum a nobis OPTIMI cognomen? quod peculiare huius et proprium arrogantia priorum principum fecit. Enimvero quam commune, quam ex aequo, quod FELICES NOS, FELICEM ILLUM praedicamus? alternisque votis, HAEC FACIAT, HAEC AUDIAT, quasi non dicturi, nisi fecerit, comprecamur? Ad quas ille voces lacrymis etiam ac multo pudore suffunditur. Agnoscit enim sentitque, sibi, non principi, dici.

Igitur quod temperamentum omnes in illo subito pietatis calore servavimus, hoc singuli quoque meditatique teneamus; sciamusque, nullum esse neque sincerius, neque acceptius genus gratiarum, quam quod illas acclamationes aemuletur, quae fingendi non habent tempus. Quantum ad me attinet, laborabo, ut orationem meam ad modestiam Principis moderationemque submittam, nec minus considerabo, quid aures eius pati possint, quam quid virtutibus debeatur. Magna et inusitata Principis gloria, cui gratias acturus, non tam vereor, ne me in laudibus suis parcum, quam ne nimium putet. Haec me cura, haec difficultas sola circumstat: nam merenti gratias agere facile est, Patres Conscripti. Non enim periculum est, ne, quum loquar de humanitate, exprobrari sibi superbiam credat; quum de frugalitate, luxuriam; quum de clementia, crudelitatem; quum de liberalitate, avaritiam; quum de benignitate, livorem; quum de continentia, libidinem; quum de labore, inertiam; quum de fortitudine, timorem. Ac ne illud quidem vereor, ne gratus ingratusve videar, prout satis aut parum dixero. Animadverto enim, etiam deos ipsos non tam accuratis adorantium precibus, quam innocentia et sanctitate, laetari; gratioremque existimari, qui delubris eorum puram castamquem mentem, quam qui meditatum carmen intulerit.

Sed parendum est Senatusconsulto, quo ex utilitate publica placuit, ut Consulis voce, sub titulo gratiarum agendarum, boni principes, quae facerent, recognoscerent; mali, quae facere deberent. Id nunc eo magis solemne ac necessarium est, quod parens noster privatas gratiarum actiones cohibet et comprimit, intercessurus etiam publicis, si permitteret sibi vetare, quod Senatus iuberet. Utrumque, Caesar Auguste, moderate, et quod alibi tibi gratias agi non sinis, et quod hic sinis. Non enim a te ipso tibi honor iste, sed agentibus habetur. Cedis affectibus nostris, nec nobis munera tua praedicare, sed audire tibi necesse est. Saepe ego mecum, Patres Conscripti, tacitus agitavi, qualem quantumque esse oporteret, cuius ditione nutuque maria, terrae, pax, bella regerentur: quum interea fingenti formantique mihi principem, quem aequata diis immortalibus potestas deceret, nunquam voto saltem concipere succurrit similem huic, quem videmus. Enituit aliquis in bello, sed obsolevit in pace: alium toga, sed non et arma honestarunt: reverentiam ille terrore, alius amorem humanitate captavit: ille quaesitam domi gloriam in publico, hic in publico partam domi perdidit. Postremo adhuc nemo exstitit, cuius virtutes nullo vitiorum confinio laederentur. At Principi nostro quanta concordia, quantusque concentus omnium laudum omnisque gloriae contigit! Ut nihil severitati eius hilaritate, nihil gravitati simplicitate, nihil maiestati humanitate detrahitur! Iam firmitas, iam proceritas corporis, iam honor capitis, et dignitas oris, ad hoc aetatis indeflexa maturitas, nec sine quodam munere deum festinatis senectutis insignibus ad augendam maiestatem ornata caesaries, nonne longe lateque principem ostentant?

Talem esse oportuit, quem non bella civilia nec armis oppressa respublica, sed pax, et adoptio, et tandem exorata terris numina, dedissent. An fas erat, nihil differre inter imperatorem, quem homines, et quem dii fecissent? quorum quidem in te, Caesar Auguste, iudicium et favor, tunc statim, quum ad exercitum proficiscereris, et quidem inusitato indicio enituit. Nam ceteros principes aut largus cruor hostiarum, aut sinister volatus avium consulentibus nuntiavit: tibi ascendenti de more Capitolium, quamquam non id agentium civium clamor, ut iam principi, occurrit. Siquidem omnis turba, quae limen insederat, ad ingressum tuum foribus reclusis, illa quidem ut tunc arbitrabatur, deum, ceterum, ut docuit eventus, te consalutavit imperatorem. Nec aliter a cunctis omne acceptum est. Nam ipse intelligere nolebas: recusabas enim imperare, recusabas; quod bene erat imperaturi. Igitur cogendus fuisti. Cogi porro non poteras, nisi periculo patriae, et nutatione reipublicae. Obstinatum enim tibi non suscipere imperium, nisi servandum fuisset. Quare ego illum ipsum furorem motumque castrensem reor exstitisse, quia magna vi magnoque terrore modestia tua vincenda erat. Ac sicut maris coelique temperiem turbines tempestatesque commendant; ita ad augendam pacis tuae gratiam illum tumultum praecessisse crediderim. Habet has vices conditio mortalium, ut adversa ex secundis, ex adversis secunda nascantur. Occultat utrorumque semina deus, et plerumque bonorum malorumque caussae sub diversa specie latent.

Magnum quidem illud seculo dedecus, magnum reipublicae vulnus impressum est. Imperator, et parens generis humani, obsessus, captus, inclusus: ablata mitissimo seni servandorum hominum potestas; ereptumque principi illud in principatu beatissimum, quod nihil cogitur. Si tamen haec sola erat ratio, quae te publicae salutis gubernaculis admoveret; prope est ut exclamem, tanti fuisse. Corrupta est disciplina castrorum, ut tu corrector emendatorque contingeres: inductum pessimum exemplum, ut optimum opponeretur: postremo coactus princeps, quos nollet, occidere, ut daret principem, qui cogi non posset. Olim tu quidem adoptari merebare; sed nescissemus, quantum tibi deberet imperium, si ante adoptatus esses. Exspectatum est tempus, in quo liqueret, non tam accepisse te beneficium, quam dedisse. Confugit in sinum tuum concussa respublica, ruensque imperium super imperatorem imperatoris tibi voce delatum est. Imploratus adoptione, et accitus es, ut olim duces magni a peregrinis externisque bellis ad opem patriae ferendam revocari solebant. Ita filius ac parens uno eodemque momento rem maximam invicem praestitistis: ille tibi imperium dedit, tu illi reddidisti. Solus ergo ad hoc aevi pro munere tanto paria accipiendo fecisti, immo ultro dantem obligasti: communicato enim imperio, solicitior tu, ille securior factus est.

O novum atque inauditum ad principatum iter! Non te propria cupiditas, proprius metus; sed aliena utilitas, alienus timor principem fecit. Videaris licet quod est amplissimum consequutus inter homines; felicius tamen erat illud, quod reliquisti: sub bono principe privatus esse desiisti. Assumptus es in laborum curarumque consortium, nec te laeta et prospera stationis istius, sed aspera et dura ad capessendam eam compulerunt. Suscepisti imperium, postquam alium suscepti poenitebat. Nulla adoptati cum eo, qui adoptabat, cognatio, nulla necessitudo, nisi quod uterque optimus erat, dignusque alter eligi, alter eligere.

\newpage

\section*{C.\ Plinii Secundi ep. 6, 33 (106/107)}

C.\ Plinius Romano suo s.

\medskip

\noindent `Tollite cuncta' inquit `coeptosque auferte labores!' Seu scribis aliquid seu legis, tolli auferri iube et accipe orationem meam ut illa arma divinam — num superbius potui? — re vera ut inter meas pulchram; nam mihi satis est certare mecum. Est haec pro Attia Viriola, et dignitate personae et exempli raritate et iudicii magnitudine insignis. Nam femina splendide nata, nupta praetorio viro, exheredata ab octogenario patre intra undecim dies quam illi novercam amore captus induxerat, quadruplici iudicio bona paterna repetebat. Sedebant centum et octoginta iudices — tot enim quattuor consiliis colliguntur — ingens utrimque advocatio et numerosa subsellia, praeterea densa circumstantium corona latissimum iudicium multiplici circulo ambibat. Ad hoc stipatum tribunal, atque etiam ex superiore basilicae parte qua feminae qua viri et audiendi — quod difficile — et — quod facile — visendi studio imminebant. Magna exspectatio patrum, magna filiarum, magna etiam novercarum. Secutus est varius eventus; nam duobus consiliis vicimus, totidem victi sumus. Notabilis prorsus et mira eadem in causa, isdem iudicibus, isdem advocatis, eodem tempore tanta diversitas. Accidit casu, quod non casus videretur: victa est noverca, ipsa heres ex parte sexta, victus Suburanus, qui exheredatus a patre singulari impudentia alieni patris bona vindicabat, non ausus sui petere.

Haec tibi exposui, primum ut ex epistula scires, quae ex oratione non poteras, deinde — nam detegam artes — ut orationem libentius legeres, si non legere tibi sed interesse iudicio videreris; quam, sit licet magna, non despero gratiam brevissimae impetraturam. Nam et copia rerum et arguta divisione et narratiunculis pluribus et eloquendi varietate renovatur. Sunt multa — non auderem nisi tibi dicere — elata, multa pugnacia, multa subtilia. Intervenit enim acribus illis et erectis frequens necessitas computandi ac paene calculos tabulamque poscendi, ut repente in privati iudicii formam centumvirale vertatur. Dedimus vela indignationi, dedimus irae, dedimus dolori, et in amplissima causa quasi magno mari pluribus ventis sumus vecti. In summa solent quidam ex contubernalibus nostris existimare hanc orationem — iterum dicam — ut inter meas  ὑπὲρ Κτησιφῶντος esse: an vere, tu facillime iudicabis, qui tam memoriter tenes omnes, ut conferre cum hac dum hanc solam legis possis. Vale.


