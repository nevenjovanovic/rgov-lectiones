%\section*{O autoru}




\section*{Gaius Iulius Caesar, Bellum civile (post 48)}

\subsection*{Argumentum}
\begin{quotation}

\noindent B. c. 23-42: Curio, Caesaris legatus, ad Uticam bene rem gerit contra Varum, mox proelio temere commisso ad Bagradam a Varo et Sabura, Iubae praefecto, superatus interficitur.
\end{quotation}

\subsection*{Bellum civile, 2.27-29}

Proxima nocte centuriones Marsi duo ex castris Curionis cum manipularibus suis XXII ad Attium Varum perfugiunt. hi sive vere quam habuerant opinionem ad eum perferunt, sive etiam auribus Vari serviunt — nam quae volumus, ea credimus libenter, et quae sentimus ipsi, reliquos sentire speramus — confirmant quidem certe totius exercitus animos alienos esse a Curione, maximeque opus esse in conspectum exercitus venire et conloquendi dare facultatem. qua opinione adductus Varus postero die mane legiones ex castris educit. facit idem Curio, atque una valle non magna interiecta suas uterque copias instruit. 

Erat in exercitu Vari Sex.\ Quintilius Varus quem fuisse Corfinii supra demonstratum est. hic dimissus a Caesare in Africam venerat, legionesque eas traduxerat Curio, quas superioribus temporibus Corfinio receperat Caesar, adeo ut paucis mutatis centurionibus idem ordines manipulique constarent. hanc nanctus appellationis causam Quintilius circumire aciem Curionis atque obsecrare milites coepit, ne primi sacramenti, quod apud Domitium atque apud se quaestorem dixissent, memoriam deponerent, neu contra eos arma ferrent, qui eadem essent usi fortuna eademque in obsidione perpessi, neu pro iis pugnarent, a quibus $\langle$per$\rangle$ contumeliam perfugae appellarentur. huc pauca ad spem largitionis addidit, quae ab sua liberalitate, si se atque Attium secuti essent, expectare deberent. hac habita oratione nullam in partem ab exercitu Curionis fit significatio, atque ita suas uterque copias reducit.
	
Atque in castris Curionis magnus omnium incessit timor animis; is variis hominum sermonibus celeriter augetur. unusquisque enim opiniones fingebat, et ad id, quod ab alio audierat, sui aliquid timoris addebat. hoc ubi uno auctore ad plures permanaverat atque alius alii tradiderat, plures auctores eius rei videbantur. civile bellum; genus hominum; quod liceret, libere facere et sequi quod vellet. legiones hae quae paulo ante apud adversarios fuerant; nam etiam Caesaris beneficia mutaverat consuetudo qua offerrentur; municipia etiam diversis partibus coniuncta — aeque enim ex Marsis Paelignisque veniebant — ut qui superiore nocte in contuberniis commilitonesque; nonnulli graviora; sermones militum; dubia durius accipiebantur, nonnulla etiam ab his, qui diligentiores videri volebant, fingebantur.

\subsection*{Bellum civile, 2.32-33}

Dimisso consilio contionem advocat militum. commemorat, quo sit eorum usus studio ad Corfinium Caesar, ut magnam partem Italiae beneficio atque auctoritate eorum suam fecerit. ``vos enim vestrumque factum,'' inquit, ``omnia deinceps municipia sunt secuta, neque sine causa et Caesar amicissime de vobis et illi gravissime iudicaverunt. Pompeius enim nullo proelio pulsus vestri facti praeiudicio demotus Italia excessit; Caesar me, quem sibi carissimum habuit, provinciam Siciliam atque Africam, sine quibus urbem atque Italiam tueri non potest, vestrae fidei commisit. at sunt qui vos hortentur, ut a nobis desciscatis. quid enim est illis optatius quam uno tempore et nos circumvenire et vos nefario scelere obstringere? aut quid irati gravius de vobis sentire possunt, quam ut eos prodatis, qui se vobis omnia debere iudicant, in eorum potestatem veniatis, qui se per vos perisse existimant? an vero in Hispania res gestas Caesaris non audistis? duos pulsos exercitus, duos superatos duces, duas receptas provincias? haec acta diebus XL, quibus in conspectum adversariorum venerit Caesar? an qui incolumes resistere non potuerunt, perditi resistant? vos autem incerta victoria Caesarem secuti diiudicata iam belli fortuna victum sequamini, cum vestri officii praemia percipere debeatis? desertos enim se ac proditos a vobis dicunt et prioris sacramenti mentionem faciunt. vosne vero L.~Domitium, an vos Domitius deseruit? nonne extremam pati fortunam paratos proiecit ille? non sibi clam vobis salutem fuga petivit? non proditi per illum Caesaris beneficio estis conservati? sacramento quidem vos tenere qui potuit, cum proiectis fascibus et deposito imperio privatus et captus ipse in alienam venisset potestatem? relinquitur nova religio, ut eo neglecto sacramento, quo tenemini, respiciatis illud, quod deditione ducis et capitis deminutione sublatum est. at, credo, si Caesarem probatis, in me offenditis. qui de meis in vos meritis praedicaturus non sum, quae sunt adhuc et mea voluntate et vestra exspectatione leviora; sed tamen sui laboris milites semper eventu belli praemia petiverunt, qui qualis sit futurus, ne vos quidem dubitatis. diligentiam quidem nostram aut, quem ad finem adhuc res processit, fortunam cur praeteream? an paenitet vos, quod salvum atque incolumem exercitum nulla omnino navi desiderata traduxerim? quod classem hostium primo impetu adveniens profligaverim? quod bis per biduum equestri proelio superaverim? quod ex portu sinuque adversariorum CC naves oneratas abduxerim eoque illos compulerim, ut neque pedestri itinere neque navibus commeatu iuvari possint? hac vos fortuna atque his ducibus repudiatis Corfiniensem ignominiam, Italiae fugam, Hispaniarum deditionem — Africi belli praeiudicia — sequimini! equidem me Caesaris militem dici volui, vos me imperatoris nomine appellavistis. cuius si vos paenitet, vestrum vobis beneficium remitto, mihi meum restituite nomen, ne ad contumeliam honorem dedisse videamini.''

Qua oratione permoti milites crebro etiam dicentem interpellabant, ut magno cum dolore infidelitatis suspicionem sustinere viderentur, discedentem vero ex contione universi cohortantur, magno sit animo neubi dubitet proelium committere et suam fidem virtutemque experiri. quo facto commutata omnium et voluntate et opinione consensu suorum constituit Curio, cum primum sit data potestas, proelio rem committere, posteroque die productos eodem loco, quo superioribus diebus constiterat, in acie conlocat.
