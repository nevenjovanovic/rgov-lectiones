%\section*{O autoru}

\section*{De doctrina Christiana, liber IV (426)}

\textit{Liber quartus de proferendis rebus agit.}

Hoc opus nostrum quod inscribitur De doctrina Christiana, in duo quaedam fueram prima distributione partitus. Nam post proemium, quo respondi eis qui hoc fuerant reprehensuri: Duae sunt res, inquam, quibus nititur omnis tractatio Scripturarum: modus inveniendi quae intellegenda sunt, et modus proferendi quae intellecta sunt. De inveniendo prius, de proferendo postea disseremus. Quia ergo de inveniendo multa iam diximus et tria de hac una parte volumina absolvimus, Domino adiuvante, de proferendo pauca dicemus, ut si fieri potuerit, uno libro cuncta claudamus totumque hoc opus quattuor voluminibus terminetur.

\textit{Rhetoricae praecepta tradere non est huius instituti.}

Primo itaque exspectationem legentium, qui forte me putant rhetorica daturum esse praecepta quae in scolis saecularibus et didici et docui, ista praelocutione cohibeo, atque ut a me non exspectentur admoneo; non quod nihil habeant utilitatis, sed quod, si quid habent, seorsum discendum est, si cui fortassis bono viro etiam haec vacat discere, non autem a me vel in hoc opere vel in aliquo alio requirendum.

\textit{Rhetorica facultate christianum doctorem uti convenit.}

Nam cum per artem rhetoricam et vera suadeantur et falsa, quis audeat dicere, adversus mendacium in defensoribus suis inermem debere consistere veritatem, ut videlicet illi qui res falsas persuadere conantur, noverint auditorem vel bene volum vel intentum vel docilem proemio facere; isti autem non noverint? Illi falsa breviter, aperte, verisimiliter et isti vera sic narrent ut audire taedeat, intellegere non pateat, credere postremo non libeat? Illi fallacibus argumentis veritatem oppugnent, asserant falsitatem, isti nec vera defendere nec falsa valeant refutare? Illi animos audientium in errorem moventes impellentesque dicendo terreant, contristent, exhilarent, exhortentur ardenter; isti pro veritate lenti frigidique dormitent? Quis ita desipiat ut hoc sapiat? Cum ergo sit in medio posita facultas eloquii, quae ad persuadenda seu prava seu recta valet plurimum, cur non bonorum studio comparatur, ut militet veritati, si eam mali ad obtinendas perversas vanasque causas in usus iniquitatis et erroris usurpant?

\textit{Rhetoricae praecepta qua aetate quave ratione disci possunt.}

Sed quaecumque sunt de hac re observationes atque praecepta, quibus, cum accedit in verbis plurimis ornamentisque verborum exercitatioris linguae solertissima consuetudo, fit illa quae facundia vel eloquentia nominatur, extra istas litteras nostras, seposito ad hoc congruo temporis spatio, apta et convenienti aetate discenda sunt eis qui hoc celeriter possunt. Nam et ipsos Romanae principes eloquentiae non piguit dicere quod hanc artem nisi quis cito possit numquam omnino possit perdiscere. Quod utrum verum sit, quid opus est quaerere? Non enim, etiam si possint haec a tardioribus tandem aliquando perdisci, nos ea tanti pendimus ut eis discendis iam maturas vel etiam graves hominum aetates velimus impendi. Satis est ut adulescentulorum ista sit cura, nec ipsorum omnium quos utilitati ecclesiasticae cupimus erudiri, sed eorum quos nondum magis urgens, et huic rei sine dubio praeponenda necessitas occupavit. Quoniam si acutum et fervens adsit ingenium, facilius adhaeret eloquentia legentibus et audientibus eloquentes, quam eloquentiae praecepta sectantibus. Nec desunt ecclesiasticae litterae, etiam praeter canonem in auctoritatis arce salubriter collocatum, quas legendo homo capax, etsi id non agat, sed tantummodo rebus quae ibi dicuntur intentus sit, etiam eloquio quo dicuntur, dum in his versatur, imbuitur, accedente vel maxime exercitatione sive scribendi sive dictandi, postremo etiam dicendi, quae secundum pietatis ac fidei regulam sentit. Si autem tale desit ingenium, nec illa rhetorica praecepta capiuntur nec, si magno labore inculcata quantulacumque ex parte capiantur, aliquid prosunt. Quandoquidem etiam ipsi qui ea didicerunt et copiose ornateque dicunt, non omnes ut secundum ipsa dicant, possunt ea cogitare cum dicunt, si non de his disputant. Immo vero vix ullos eorum esse existimo qui utrumque possint, et dicere bene et ad hoc faciendum praecepta illa dicendi cogitare cum dicunt. Cavendum est enim ne fugiant ex animo, quae dicenda sunt, dum attenditur ut arte dicantur. Et tamen in sermonibus atque dictionibus eloquentium impleta reperiuntur praecepta eloquentiae, de quibus illi ut eloquerentur vel cum eloquerentur non cogitaverunt, sive illa didicissent sive ne attigissent quidem. Implent quippe illa, quia eloquentes sunt, non adhibent ut sint eloquentes.

\textit{Pueri ex ore loquentium locutiones discunt.}

Quapropter, cum ex infantibus loquentes non fiant, nisi locutiones discendo loquentium, cur eloquentes fieri non possunt, nulla eloquendi arte tradita, sed elocutiones eloquentium legendo et audiendo et, quantum assequi conceditur, imitando? Quid, quod ita fieri ipsis quoque experimur exemplis? Nam sine praeceptis rhetoricis novimus plurimos eloquentiores plurimis qui illa didicerunt, sine lectis vero et auditis eloquentium disputationibus vel dictionibus neminem. Nam neque ipsa arte grammatica, qua discitur locutionis integritas, indigerent pueri, si eis inter homines, qui integre loquerentur, crescere daretur et vivere. Nescientes quippe ulla nomina vitiorum, quidquid vitiosum cuiusquam ore loquentis audirent, sana sua consuetudine reprehenderent et caverent, sicut rusticos urbani reprehendunt, etiam qui litteras nesciunt.

\textit{Officium doctoris christiani\dots}

Debet igitur divinarum Scripturarum tractator et doctor, defensor rectae fidei ac debellator erroris, et bona docere et mala dedocere atque in hoc opere sermonis conciliare aversos, remissos erigere, nescientibus quid agatur quid exspectare debeant intimare. Ubi autem benevolos, intentos, dociles aut invenerit aut ipse fecerit, cetera peragenda sunt, sicut postulat causa. Si docendi sunt qui audiunt, narratione faciendum est, si tamen indigeat, ut res de qua agitur innotescat. Ut autem quae dubia sunt certa fiant, documentis adhibitis ratiocinandum est. Si vero qui audiunt movendi sunt potius quam docendi, ut in eo quod iam sciunt agendo non torpeant et rebus assensum quas veras esse fatentur accomodent, maioribus dicendi viribus opus est. Ibi obsecrationes et increpationes, concitationes et coercitiones et quaecumque alia valent ad commovendos animos, sunt necessaria.

\textit{\dots\ et cuiusvis eruditi hominis.}

Et haec quidem cuncta quae dixi omnes fere homines in his quae eloquendo agunt, facere non quiescunt.

Interest magis ut sapienter dicat christianus orator, quam ut eloquenter.

Sed cum alii faciant obtunse, deformiter, frigide, alii acute, ornate, vehementer, illum ad hoc opus unde agimus iam oportet accedere, qui potest disputare vel dicere sapienter, etiamsi non potest eloquenter, ut prosit audientibus, etiamsi minus, quam prodesset si et eloquenter posset dicere. Qui vero affluit insipienti eloquentia, tanto magis cavendus est quanto magis ab eo in his quae audire inutile est, delectatur auditor et eum quoniam diserte dicere audit, etiam vere dicere existimat. Haec autem sententia nec illos fugit qui artem rhetoricam docendam putarunt. Fassi sunt enim sapientiam sine eloquentia parum prodesse civitatibus, eloquentiam vero sine sapientia nimium obesse plerumque, prodesse numquam. Si hoc ergo illi, qui praecepta eloquentiae tradiderunt, in eisdem libris in quibus id egerunt, veritate instigante coacti sunt confiteri, veram, hoc est supernam quae a Patre luminum descendit sapientiam nescientes, quanto magis nos non aliud sentire debemus, qui huius sapientiae filii et ministri sumus? Sapienter autem dicit homo tanto magis vel minus, quanto in Scripturis sanctis magis minusve profecit, non dico in eis multum legendis memoriaeque mandandis, sed bene intellegendis et diligenter earum sensibus indagandis. Sunt enim qui eas legunt et neglegunt: legunt ut teneant, neglegunt ne intellegant. Quibus longe sine dubio praeferendi sunt qui verba earum minus tenent et cor earum sui cordis oculis vident. Sed utrisque ille melior qui et cum volet eas dicit et sicut oportet intellegit.

\textit{Dicendi ars legendo oratorum libros acquiritur.}

Huic ergo qui sapienter debet dicere, etiam quod non potest eloquenter, verba Scripturarum tenere maxime necessarium est. Quanto enim se pauperiorem cernit in suis, tanto eum oportet in istis esse ditiorem, ut quod dixerit suis verbis probet ex illis, et qui propriis verbis minor erat, magnorum testimonio quodammodo crescat. Probando enim delectat qui minus potest delectare dicendo. Porro qui non solum sapienter, verum etiam eloquenter vult dicere, quoniam profecto plus proderit, si utrumque potuerit, ad legendos vel audiendos et exercitatione imitandos eloquentes eum mitto libentius quam magistris artis rhetoricae vacare praecipio; si tamen hi qui leguntur et audiuntur, non solum eloquenter, sed etiam sapienter dixisse vel dicere veraci praedicatione laudantur. Qui enim eloquenter dicunt, suaviter; qui sapienter, salubriter audiuntur. Propter quod non ait Scriptura: Multitudo eloquentium, sed: Multitudo sapientium sanitas est orbis terrarum. Sicut autem saepe sumenda sunt et amara salubria, ita semper vitanda est perniciosa dulcedo. Sed salubri suavitate vel suavi salubritate quid melius? Quanto enim magis illic appetitur suavitas, tanto facilius salubritas prodest. Sunt ergo ecclesiastici viri qui divina eloquia non solum sapienter, sed eloquenter etiam tractaverunt, quibus legendis magis non sufficit tempus quam deesse ipsi studentibus et vacantibus possunt.

\textit{Eloquentia in sacris auctoribus profanam eloquentiam transcendit.}

Hic aliquis forsitan quaerit utrum auctores nostri, quorum scripta divinitus inspirata canonem nobis saluberrima auctoritate fecerunt, sapientes tantummodo an eloquentes etiam nuncupandi sunt. Quae quidem quaestio apud me ipsum et apud eos qui mecum quod dico sentiunt, facillime solvitur. Nam ubi eos intellego, non solum nihil eis sapientius, verum etiam nihil eloquentius mihi videri potest. Et audeo dicere omnes qui recte intellegunt quod illi loquuntur, simul intellegere non eos aliter loqui debuisse. Sicut est enim quaedam eloquentia quae magis aetatem iuvenilem decet, est quae senilem, nec iam dicenda est eloquentia si personae non congruat eloquentis; ita est quaedam, quae viros summa auctoritate dignissimos planeque divinos decet. Haec illi locuti sunt, nec ipsos decet alia nec alios ipsa. Ipsis enim congruit; alios autem, quanto videtur humilior, tanto altius non ventositate, sed soliditate transcendit. Ubi vero non eos intellego, minus quidem mihi apparet eorum eloquentia, sed eam tamen non dubito esse talem, qualis est ubi intellego. Ipsa quoque obscuritas divinorum salubriumque dictorum tali eloquentiae miscenda fuerat, in qua proficere noster intellectus, non solum inventione, verum etiam exercitatione deberet.

\section*{Sermo 350. De caritate (425-430)}


\begin{quote}
\textit{Caritas mandatum novum faciens hominem novum.}
\end{quote}

Divinarum Scripturarum multiplicem abundantiam, latissimamque doctrinam, fratres mei, sine ullo errore comprehendit, et sine ullo labore custodit, cuius cor plenum est caritate, dicente Apostolo: Plenitudo autem legis caritas; et alio loco: Finis autem praecepti est caritas de corde puro, et conscientia bona, et fide non ficta Quis est autem finis praecepti, nisi praecepti adimpletio? et quid est praecepti adimpletio, nisi legis plenitudo? Quod ergo ibi dixit: Plenitudo legis est caritas, hoc etiam hic dixit: Finis praecepti est caritas. Nec dubitari ullo modo potest, quod templum Dei sit homo, in quo habitat caritas. Dicit enim et Ioannes: Deus caritas est. Haec autem dicentes Apostoli et nobis caritatis excellentiam commendantes, non utique aliud, nisi quod comederant, ructuare potuerunt. Ipse quippe Dominus pascens eos verbo veritatis, verbo caritatis, quod est ipse panis vivus, qui de caelo descendit, Mandatum, inquit, novum do vobis, ut diligatis invicem. Et iterum: In hoc scient omnes quia discipuli mei estis, si vos invicem dilexeritis. Ille enim qui venit per crucis irrisionem carnis perimere corruptionem, et vetustatem vinculi mortis nostrae suae mortis novitate dissolvere, mandato novo fecit hominem novum. Res enim vetus erat, ut homo moreretur. Quod ne semper valeret in homine, res nova facta est, ut Deus moreretur. Sed quia in carne mortuus est, non in divinitate, per sempiternam vitam divinitatis non permisit esse sempiternum interitum carnis. Itaque, sicut dicit Apostolus: Mortuus est propter de licta nostra, et resurrexit propter iustificationem nostram. Qui ergo contra mortis vetustatem attulit vitae novitatem, ipse contra vetus peccatum opponit novum mandatum. Quapropter quisquis vetus peccatum vis exstinguere, mandato novo exstingue cupiditatem, et amplectere caritatem. Sicut enim radix omnium malorum est cupiditas; ita et radix omnium bonorum est caritas.

\begin{quote}
\textit{Scripturarum tota doctrina caritate possidetur.}
\end{quote}

Totam magnitudinem et latitudinem divinorum eloquiorum secura possidet caritas, qua Deum proximumque diligimus. Docet enim nos caelestis unus Magister, et dicit: Diliges Dominum Deum tuum ex toto corde tuo, et ex tota anima tua, et ex tota mente tua: et diliges proximum tuum sicut te ipsum. In his duobus praeceptis universa Lex pendet, et Prophetae. Si ergo non vacat omnes paginas sanctas perscrutari, omnia involucra sermonum evolvere, omnia Scripturarum secreta penetrare; tene caritatem, ubi pendent omnia: ita tenebis quod ibi didicisti; tenebis etiam quod nondum didicisti. Si enim nosti caritatem, aliquid nosti unde et illud pendet quod forte non nosti: et in eo quod in Scripturis intellegis, caritas patet; in eo quod non intellegis, caritas latet. Ille itaque tenet et quod patet et quod latet in divinis sermonibus, qui caritatem tenet in moribus.

\begin{quote}
\textit{Caritatis laus.}
\end{quote}


Quapropter, fratres, sectamini caritatem, dulce ac salubre vinculum mentium, sine qua dives pauper est, et cum qua pauper dives est. Haec in adversitatibus tolerat, in prosperitatibus temperat; in duris passionibus fortis, in bonis operibus hilaris; in tentatione tutissima, in hospitalitate latissima; inter veros fratres laetissima, inter falsos patientissima. In Abel per sacrificium grata, in Noe per diluvium secura, in Abrahae peregrinationibus fidelissima, in Moyse inter iniurias lenissima, in David tribulationibus mansuetissima. In tribus pueris blandos ignes innocenter exspectat: in Machabaeis saevos ignes fortiter tolerat. Casta in Susanna erga virum, in Anna post virum, in Maria praeter virum. Libera in Paulo ad arguendum, humilis in Petro ad oboediendum: humana in Christianis ad confitendum, divina in Christo ad ignoscendum. Sed quid ego de caritate maius aut uberius possum dicere, quam quas per os Apostoli laudes eius intonat Dominus, supereminentem viam demonstrantis atque dicentis: Si linguis hominum loquar et Angelorum, caritatem autem non habeam, factus sum aeramentum sonans, aut cymbalum tinniens. Et si habuero prophetiam, et sciero omnia sacramenta, et omnem scientiam, et si habuero omnem fidem, ita ut montes transferam, caritatem autem non habeam, nihil sum. Et si donavero omnes facultates meas, et si distribuero omnia mea pauperibus, et si tradidero corpus meum ut ardeam, caritatem autem non habeam, nihil mihi prodest. Caritas magnanima est, caritas benigna est. Caritas non aemulatur, non agit perperam, non inflatur, non dehonestatur, non quaerit quae sua sunt, non irritatur, non cogitat malum, non gaudet super iniquitate, congaudet autem veritati. Omnia tolerat, omnia credit, omnia sperat, omnia suffert. Caritas numquam cadit? Quanta est ista? Anima litterarum, prophetiae virtus, sacramentorum salus, scientiae solidamentum, fidei fructus, divitiae pauperum, vita morientium. Quid tam magnanimum, quam pro impiis mori? quid tam benignum, quam inimicos diligere? Sola est quam felicitas aliena non premit, quia non aemulatur. Sola est quam felicitas sua non extollit, quia non inflatur. Sola est quam conscientia mala non pungit, quia non agit perperam. Inter opprobria secura est, inter odia benefica est; inter iras placida est, inter insidias innocens; inter iniquitates gemens, in veritate respirans. Quid illa fortius, non ad retribuendas, sed ad non curandas iniurias? Quid illa fidelius, non vanitati, sed aeternitati? Nam ideo tolerat omnia in praesenti vita, quia credit omnia de futura vita; et suffert omnia quae hic immittuntur, quia sperat omnia quae ibi promittuntur: merito numquam cadit. Ergo sectamini caritatem, et eam sancte cogitantes afferte fructus iustitiae. Et quidquid uberius, quam ego dicere potui, vos inveneritis in eius laudibus, appareat in vestris moribus. Oportet enim ut senilis sermo non solum sit gravis, sed etiam brevis.

