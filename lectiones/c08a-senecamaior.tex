%\section*{O autoru}

\section*{L. Annaei Senecae Maioris controversiarum liber primus}

L. ANNAEI SENECAE MAIORIS \\CONTROVERSIARVM LIBER PRIMVS

\bigskip

Seneca Novato, Senecae, Melae filiis salutem

Exigitis rem magis iucundam mihi quam facilem; iubetis enim quid de his declamatoribus sentiam, qui in aetatem meam inciderunt, indicare et si qua memoriae meae nondum elapsa sunt ab illis dicta colligere, ut, quamvis notitiae vestrae subducti sint, tamen non credatis tantum de illis sed et iudicetis.

Est, fateor, iucundum mihi redire in antiqua studia melioresque ad annos respicere et vobis querentibus, quod tantae opinionis viros audire non potueritis, detrahere temporum iniuriam. sed cum multa iam mihi ex me desideranda senectus fecerit, oculorum aciem retuderit, aurium sensum hebetaverit, nervorum firmitatem fatigaverit, inter ea quae rettuli memoria est, res ex omnibus animi partibus maxime delicata et fragilis, in quam primam senectus incurrit. hanc aliquando in me floruisse, ut non tantum ad usum sufficeret sed in miraculum usque procederet, non nego; nam et duo milia nominum recitata, quo erant ordine dicta, reddebam, et ab his, qui ad audiendum praeceptorem meum convenerant, singulos versus a singulis datos, cum plures quam ducenti efficerentur, ab ultimo incipiens usque ad primum recitabam. nec ad complectenda tantum quae vellem velox mihi erat memoria sed etiam ad continenda quae acceperam solebat bonae fidei esse. nunc iam aetate quassata et longa desidia, quae iuvenilem quoque animum dissolvit, eo perducta est, ut, etiamsi potest aliquid praestare, non possit promittere: diu ab illa nihil repetivi. nunc quia iubetis, quid possit experiar et illam omni cura scrutabor; ex parte enim bene spero, nam quaecumque apud illam aut puer aut iuvenis deposui, quasi recentia aut modo audita sine cunctatione profert; at si qua illi intra proximos annos commisi, sic perdidit et amisit, ut, etiamsi saepius ingerantur, totiens tamen tamquam nova audiam.  ita ex memoria mea quantum vobis satis sit superest; neque enim de his me interrogatis, quos ipsi audistis, sed de his, qui ad vos usque non pervenerunt.

Fiat quod vultis: mittatur senex in scholas, illud necesse est impetrem, ne me quasi certum aliquem ordinem velitis sequi in contrahendis quae mihi occurrent; necesse est enim per omnia studia mea errem et passim quidquid obvenerit adprehendam. controversiarum sententias fortasse pluribus locis ponam in una declamatione dictas; non enim, dum quaero aliquid, invenio, sed saepe quod quaerenti non comparuit, aliud agenti praesto est; quaedam vero, quae obversantia mihi et ex aliqua parte se ostendentia non possum occupare, eadem securo et reposito animo subito emergunt. aliquando etiam seriam rem agenti et occupato sententia diu frustra quaesita intempestive molesta est. necesse est ergo me ad delicias componam memoriae meae, quae mihi iam olim precario paret.

Facitis autem, iuvenes mei, rem necessariam et utilem, quod non contenti exemplis saeculi vestri priores quoque vultis cognoscere; primum quia, quo plura exempla inspecta sunt, plus in eloquentiam proficitur. non est unus, quamvis praecipuus sit, imitandus, quia numquam par fit imitator auctori. haec rei natura est: semper citra veritatem est similitudo. deinde, ut possitis aestimare, in quantum cotidie ingenia decrescant et nescio qua iniquitate naturae eloquentia se retro tulerit. quidquid Romana facundia habet, quod insolenti Graeciae aut opponat aut praeferat, circa Ciceronem effloruit; omnia ingenia, quae lucem studiis nostris attulerunt, tunc nata sunt. in deterius deinde cotidie data res est sive luxu temporum‚ nihil enim tam mortiferum ingeniis quam luxuria est‚ sive, cum pretium pulcherrimae rei cecidisset, translatum est omne certamen ad turpia multo honore quaestuque vigentia, sive fato quodam, cuius maligna perpetuaque in rebus omnibus lex est, ut ad summum perducta rursus ad infimum velocius quidem quam ascenderant relabantur.

Torpent ecce ingenia desidiosae iuventutis, nec in unius honestae rei labore vigilatur: somnus languorque ac somno et languore turpior malarum rerum industria invasit animos; cantandi saltandique obscena studia effeminatos tenent; capillum frangere et ad muliebres blanditias extenuare vocem, mollitia corporis certare cum feminis et immundissimis se excolere munditiis nostrorum adulescentium specimen est. quis aequalium vestrorum quid dicam satis ingeniosus, satis studiosus, immo quis satis vir est? emolliti enervesque quod nati sunt inviti manent, expugnatores alienae pudicitiae, neglegentes suae. in hos ne dii tantum mali ut cadat eloquentia; quam non mirarer, nisi animos, in quos se conferret, eligeret. erratis, optimi iuvenes, nisi illam vocem non M. Catonis sed oraculi creditis; quid enim est oraculum? nempe voluntas divina hominis ore enuntiata; et quem tandem antistitem sanctiorem sibi invenire divinitas potuit quam M. Catonem, per quem humano generi non praeciperet sed convicium faceret? ille ergo vir quid ait? 'orator est, Marce fili, vir bonus dicendi peritus.' ite nunc et in istis vulsis atque epolitis et nusquam nisi in libidine viris quaerite oratores. merito talia habent exempla qualia ingenia. quis est qui memoriae studeat? quis est qui non dico magnis viribus sed suis placeat? sententias a disertissimis viris factas facile in tanta hominum desidia pro suis dicunt et sic sacerrimam eloquentiam, quam praestare non possunt, violare non desinunt.

Eo libentius quod exigitis faciam et quaecumque a celeberrimis viris facunde dicta teneo, ne ad quemquam privatim pertineant, populo dedicabo. ipsis quoque multum praestaturus videor, quibus oblivio imminet, nisi aliquid, quo memoria eorum producatur, posteris traditur. fere enim aut nulli commentarii maximorum declamatorum extant aut, quod peius est, falsi. itaque, ne aut ignoti sint aut aliter quam debent noti, summa cum fide suum cuique reddam.

Omnes autem magni in eloquentia nominis excepto Cicerone videor audisse. ne Ciceronem quidem aetas mihi eripuerat sed bellorum civilium furor, qui tunc orbem totum pervagabatur, intra coloniam meam me continuit; alioqui in illo atriolo, in quo duos grandes praetextatos ait secum declamare, potui adesse illudque ingenium, quod solum populus Romanus par imperio suo habuit, cognoscere et, quod vulgo aliquando dici solet sed in illo proprie debet, potui vivam vocem audire.

Declamabat autem Cicero non quales nunc controversias dicimus, ne tales quidem quales ante Ciceronem dicebantur, quas thesis vocabant. hoc enim genus materiae, quo nos exercemur, adeo novum est, ut nomen quoque eius novum sit. controversias nos dicimus: Cicero causas vocabat. hoc vero alterum nomen Graecum quidem, sed in Latinum ita translatum, ut pro Latino sit, 'scholastica', controversia multo recentius est, sicut ipsa 'declamatio' apud nullum antiquum auctorem ante Ciceronem et Calvum inveniri potest, qui declamationem a dictione distinguit; ait enim declamare iam se non mediocriter, dicere bene; alterum putat domesticae exercitationis esse, alterum verae actionis. modo nomen hoc prodiit, nam et studium ipsum nuper celebrari coepit. ideo facile est mihi ab incunabulis nosse rem post me natam.

In aliis autem an beneficium vobis daturus sim, nescio; in uno accipio: Latronis enim Porcii, carissimi mihi sodalis, memoriam saepius cogar retractare et a prima pueritia usque ad ultimum eius diem perductam familiarem amicitiam cum voluptate maxima repetam. nihil illo viro gravius, nihil suavius, nihil eloquentia dignius. nemo plus ingenio suo imperavit, nemo plus indulsit. in utramque partem vehementi viro modus deerat: nec intermittere studium sciebat nec repetere. cum se ad scribendum concitaverat, iungebantur noctibus dies et sine intervallo gravius sibi instabat nec desinebat nisi defecerat. rursus cum se remiserat, in omnes lusus, in omnes iocos se resolvebat. cum vero se silvis montibusque tradiderat, in silvis ac montibus natos homines illos agrestis laboris patientia et venandi sollertia provocabat et in tantam perveniebat sic vivendi cupiditatem, ut vix posset ad priorem consuetudinem retrahi. at cum sibi iniecerat manum et se blandienti otio abduxerat, tantis viribus incumbebat in studium, ut non tantum nihil perdidisse sed multum adquisisse desidia videretur. 

(\dots)

Sed iam non sustineo diutius vos morari: scio quam odiosa res mihi sit Circensibus pompa. Ab ea controversia incipiam quam primam Latronem meum declamasse memini admodum iuvenem in Marulli schola, cum iam coepisset ordinem ducere. 

\subsection*{Patruus abdicans}

Liberi parentes alant aut vinciantur.

\bigskip

Duo fratres inter se dissidebant; alteri filius erat. Patruus in egestatem incidit; patre vetante adulescens illum aluit; ob hoc abdicatus tacuit. Adoptatus a patruo est. Patruus accepta hereditate locuples factus est. Egere coepit pater: vetante patruo alit illum. Abdicatur. 	

\bigskip

PORCI LATRONIS. Quid mihi obicis? puto luxuriam: quidquid umquam inmodesta largitione effudimus, id omne †consumatur† in alimentum duorum senum. 

Cum vetaret me pater, aiebat: 'ipse mihi, si egerem, alimenta non daret.' 

Eo iam perductus erat ut omnem spem ultimorum alimentorum in ea domo poneret in qua habebat et abdicatum et inimicum. 

Ecce oppresserit mors egentem: quid facturus es? pluris tibi frater efferendus quam alendus est. 

Quisquis rogatus est, ait: 'quid porro? tam locuples frater alere non potest?' 

Miserrimus senex divitiis tuis etiam blandimentum in stipem perdidit. 

'Ipse' inquit 'me ali vetuit.' Imitationem alienae culpae innocentiam vocas? 

Ne eo quidem aestimas quanta ista crudelitas sit, quod, si quis fratrem non alit, ne a filio quidem alendus est? 

Quid adoptionem iactas? tunc ad te veni cum haberem divitem patrem. 

Parcius, quaeso, patres: praesentes habemus deos. 

Scis tuto te facere: etiamsi abdicaveris, alam. 

Fatendum est crimen meum: tardius miseritus sum; itaque do poenas: egeo. 

Parentibus meis, cum in cetera odium sit, tantum in meam notam convenit. 

O felix spectaculum si vos in gratiam possum reducere: faciamque hoc, vultus quoque vestri hortantur. Surgite patres, adeste iudices: alter mihi ex parentibus servatus, alter servandus est. Porrigite mutuas in gratiam manus; me foederi medium pignus addite: inter contendentes duos medius elidor. 

Ergo fame morientem videbo per cuius cineres iuraturus sum? 	

Omnis instabilis et incerta felicitas est: quis crederet iacentem supra crepidinem Marium aut fuisse consulem aut futurum? Quid porro tam longe exempla repeto, tamquam domi desit? qui illum vidit quid non timendum felicibus putat, quid desperandum infelicibus?

\bigskip

IUNI GALLIONIS Ego indicabo cur me abdices: tu indica cur adoptaveris. 

Quae iam accedunt nova? Equidem illud non miror, quod misericordia obicitur: illud miror, quod hic obicit; sic enim me gessi ut hoc crimine duos patres obligarem. Uterque me amat, uterque ali miser 	
desiderat, uterque prohibet. Nec secum nec cum fortuna bene convenit. Conponite aliquando bonos quidem sed contumaces viros. Uter discordiae causam praebuerit nolite a me exigere: uterque patruus est, uterque pater est. 

Transit ad istum fratris sui et fortuna et animus. Misericors sum: non mutassem patrem si naturam mutare potuissem! 	

\bigskip

P. ASPRENATIS Fortunae lex est praestare quae exegeris. Miserere: mutabilis est casus; dederunt victis terga victores et quos provexerat fortuna destituit. Quid referam Marium sexto consulatu Carthagini mendicantem, septimo Romae imperantem? Ne circa plura instabilis fortunae exempla te mittam, vide quis alimenta rogetur et quis roget. 	

\bigskip

IUNI OTHONIS patris Time mutationem: et ille nihil prius ex bonis quam filium perdidit.

\bigskip

ARELLI FUSCI patris Ecquid aperis mihi penates tuos? Non sum hospes gravis, unum senem adduco: hoc tibi vitio, pater, placui. Venit ignotus senex; volo transire tacentem; per patrem rogat. Ergo aliquis peribit fame qui filium suum optat superstitem? 

Quid hoc esse dicam quod me tam periculose abdicant? quod totiens isti fortunam mutant quotiens ego patrem? Redite in gratiam: inter funestas acies armatae manus in foedus porriguntur. Perierat totus orbis nisi iram finiret misericordia. Aut si tam pertinacia placent odia, parcite: iactatus inter duos patres, utriusque filius, semper tamen felicioris abdicatus, positus inter duo pericula, quid faciam? 

Qui alunt abdicantur, mendicant qui non alunt. 	

Illud tamen, pater, deos testor: divitem te relinquo.

\bigskip

CESTI PII Tali me operi praeparaveram: volebam fratres in gratiam reducere. Hoc tu obicis? At nisi impetravero ut boni fratres sint, impetrabo ne mali patres sint? Uterque me amavit, uterque pro me vota fecit; quantum est si dixero: 'uterque me aluit'? Quae causa fuerit discordiae, nescio; timeo ne iste prior iniuriam fecerit qui prior egere coepit. 

Quid obicis, pater? misericordiam? Scio quendam in hac civitate propter istud crimen adoptatum. 'Fratrem' inquit 'alere noluit.' 
†Invenisti quod possem defendere.† Possum liberos tollere ut primum hoc illis narrem, avum illorum fame perisse? 

Non fefelli te qualis essem: scivisti cum adoptares. 

Bis abdicatus sum: volo utramque causam meam agere, neutram per me volo; adsit mihi altus: semper causa mea habebit advocatum patrem. 

Alter alterum amet: uterque me amavit. Vis illum veras poenas dare? sentiat quam bono fratri iniuriam fecerit.

\bigskip

POMPEI SILONIS De patre vestro merui bene, quamquam eum per aetatem nosse non possum; sed habet et ille beneficium meum: duos eius filios alui. Surge, infelix senex. Quid? putatis illum flere quod eget? immo quod abdicavit, quod non aluit.

\bigskip
	
ARGENTARI Vides enim, liberalis in domo tua esse coepi! Ille propter me duxit uxorem, cum fortasse iuvenem adoptare posset. Haec abdicantis fuere verba: 'i ad illum quem magis amas quam patrem.' Non omnibus imperiis patris parendum est. 

Nihil in te novi facio: scis me et priori patri non paruisse. 

Venit immissa barba capilloque deformi, non senectute sed fame membris trementibus, summissa et tenui atque elisa ieiunio voce, ut vix exaudiri posset, introrsus conditos oculos vix allevans: alui. Quomodo, quaeritis? quomodo istum. 	

\bigskip

CORNELI HISPANI Puta me hodie non abdicari, sed adoptari. Volo quaedam futuro praedicere patri: Hic quem vis adoptare inimicum patris sui invito patre aluit. Reliquit aequo animo beatam 	
domum, ut cum mendico viveret. Noveris oportet hoc eius vitium: ad praestandam calamitosis misericordiam contumax est. 

Nec tamen habeo quod de hoc vitio meo queri possim: hoc inveni patrem, hoc perdidi. Quam multi patres optant similem filium! bis abdicor. 

Homo est: non vis alam hominem? Civis est: non vis alam civem? Amicus est: non vis alam amicum? Propinquus est: non vis alam propinquum? Sic pervenitur ad patrem. Homo est, civis est, amicus est, propinquus est; ista condicione ergo non erit vitium porrexisse stipem nisi dixero: 'pater est.' 

\bigskip	

VIBI GALLI Circumibo tecum, pater, aliena limina; ostendam omnibus et me, qui alimenta dedi, et te, qui negasti. 

\bigskip	

ROMANI HISPONIS Scio, pater, melius esse quod tu dicis: istud ego si possem, numquam abdicatus essem. Fateor, vitium est: hoc quoque in me prior emendare voluit pater nec potuit. Impulisti me in fraudem: qui me abdicabat aiebat: 'non oportebat fieri,' tu dicebas oportere, tibi credidi. 

'Non dedit' inquit 'mihi alimenta': defuerunt tibi? Quisquis alimenta a mendico rogatus est, nihil amplius quam monstrat: 'i ad fratrem, i ad filium.' 	

Iam quidam nobis eandem fortunam precantur. Crede mihi, sacra populi lingua est.

\bigskip

ALBUCI SILI Tollite vestras divitias, quas huc atque illuc incertae fortunae fluctus appellet; redite in gratiam: innocens sum. 	


\subsection*{Pars altera.}

VALLI SYRIACI Crescere ex mea proposuit invidia: sequemur senes quo vocat ambitio iuvenilis et contionem illi praebebimus? Melius se potest iactare quam defendere. Ecquid iustus metus meus est, ne heredem ingratum scribam, inimicum relinquam? 

Inter cetera quae mihi cum inimico fateor esse communia et hoc est: infelicissimam ambo et tristissimam egimus vitam, excepto uno quod alter alterum egentem vidimus. 

Proici me adiectis verborum contumeliis iussit: ad caelum manus sustulit, fassus huius se spectaculi debitorem, et tunc primum fratri vitam precatus est. Laetitiam parati patrimonii ut ex tanto calamitatium stupore nullam percepi, nisi quod isti daturus omnia eram, illi negaturus. 

Liquet nobis deos esse: qui non aluit eget, qui in domum suam fratrem non recepit in publico manet. 

Aequavit iam potentiam meam cum illius potentia fortuna, nisi quod haec prior facere non possum. 

Adoptavi te cum abdicatus es: cum adoptas abdico.

\bigskip

VIBI RUFI Cum egerem, aiebam: 'satis se vindicavit, quod a dispensatore locupletis inimici consors modo omnis fortunae diurnum petam.'

\bigskip

MARULLI Ille vitam audebit rogare, qui mori malet quam sua verba sibi dici? Multis debeo misericordiam, multis tuli. Quisquis est qui me ulla calamitate similem effingit, perinde habeo ac si gradu cognationis attingat. Scio quam acerbum sit supplicare exteris; scio quam grave sit repelli a domesticis; scio quam crudele sit cotidie et mortem optare et vitam rogare. Etiamsi tu non odisti eum qui mihi fecit iniuriam, ego odi eum qui fecit tibi. 

\bigskip	

Divisio. Divisio controversiarum antiqua simplex fuit; recens utrum subtilior an tantum operosior sit ipsi aestimabitis: ego exponam quae aut veteres invenerunt aut sequentes adstruxerunt. 	

\bigskip

Latro illas quaestiones fecit: divisit in ius et aequitatem, an abdicari possit, an debeat. 

An possit abdicari, sic quaesit: an necesse fuerit illum patrem alere, et ob id abdicari non possit quod fecit lege cogente. 

Hoc in has quaestiones divisit: an abdicatus non desinat filius esse; an is desinat qui non tantum abdicatus sed etiam ab alio adoptatus est. Etiamsi filius erat, an quisquis patrem non aluit puniatur, tamquam aeger, vinctus, captus; an aliquam filii lex excusationem accipiat; an in hoc accipere potuerit. 

An abdicari debeat, per haec quaesit: an, etiamsi ille indignus fuit qui aleretur, hic tamen recte fecerit qui aluit; deinde an dignus fuerit qui aleretur.

\bigskip 	

Novi declamatores Graecis auctoribus adiecerunt primam illam quaestionem: an adoptatus abdicari possit. Hac Cestius usus est. Adiecit quaestionem alteram: an, si abdicari possit etiam 	
adoptatus, possit ob id vitium quod antequam adoptaretur notum fuit adoptanti. Haec autem ex aequitatis parte pendet et tractatio magis est quam quaestio.

\bigskip

Gallio quaestionem primam Latronis duplicavit sic: licuit mihi alere etiam te vetante; deinde non licuit non alere. 

In priore parte hoc vindicavit, non posse filium ob id abdicari quod esset suae potestatis; nulli autem interdici misericordia: Quid si flere me vetes cum vidi hominem calamitosum? quid si vetes propter aliquod honestum factum periclitanti favere? Adfectus nostri in nostra potestate non sunt. Quaedam iura non scripta, sed omnibus scriptis certiora sunt: quamvis filius familiae sim, licet mihi et stipem porrigere mendico et humum cadaveri inicere. Iniquum est conlapsis manum non porrigere: commune hoc ius generis humani est. Nemo invidiosum ius postulat quo alteri profuturus est. 

\bigskip	

Latro illud vehementer pressit: Non feci ratione, adfectu victus sum. Cum vidissem patrem egentem, mens non constitit mihi; quid vetueris nescio. Hoc aiebat non esse tractandum tamquam quaestionem; esse tamen potentius quam ullam quaestionem.

Fuscus Arellius pater hoc movit in ultimo tamquam quaestionem: putavi te, quamvis vetares, nihilominus velle ali fratrem: eo vultu vetabas aut mihi ita videbaris. 	

Cestius audacius; non fuit contentus dicere: 'putavi velle te,' adiecit: 'voluisti et hodie quoque vis,' et sua figura dixit omnia propter quae velle deberet. 'Quare ergo abdicas? puto, indignaris praereptum tibi officium.'

\bigskip

Latro colorem simplicem pro adulescente introduxit: habere non quo excuset, sed quo glorietur. Non potui, inquit, sustinere illud durum spectaculum. Offensam mihi putas tantum excidisse? mens excidit, non animus mihi constitit, non in ministerium sustinendi corporis suffecerunt pedes, oculi subita caligine obtorpuerunt: alioqui ego, si tunc meae mentis fuissem, expectassem dum rogarer? 	

Fuscus illum colorem introduxit, quo frequenter uti solebat, religionis: movit, inquit, me natura, movit pietas, movit humanorum casuum tam manifesto approbata exemplo varietas. Stare ante oculos Fortuna videbatur et dicere talia: esuriunt qui suos non alunt. 	

Albucius hoc colore: accessit, inquit, ad me pater, nec summissis verbis locutus est; non rogavit, sciit quomodo agendum esset cum filio: alere me iussit; recitavit legem, quam ego semper scriptam etiam patruo putavi. Et deinde dixit: praestiti non quantum patri praestare debui, sed quantum vetanti subripere potui. 	

Blandus colore diverso: Venit subito deformis squalore, lacrimis. O graves, Fortuna, vices tuas! Ille dives modo superbus rogavit alimenta, rogavit filium suum, rogavit abdicatum suum. Interrogas quam diu rogaverit? Ne di istud nefas patiantur, ut diu rogaverit; diutius tamen quam tu. Quaeritis quid fecerim? quod solebam. 	

Silo Pompeius hoc colore: Movit, inquit, me quod nihil suo iure, nihil pro potestate, quod tamquam patruus accessit. Ego vero non expectavi verba, non preces: complexus sum et osculatus sum patrem, dedi alimenta. Hoc unum crudeliter feci, quod dixi fratrem dedisse: non alere sed exprobrare visus sum. 	

Triarius hoc colore: timui, inquit, si non aluissem, ne abdicarer a patre; sciebam quomodo illi placuissem.

Argentarius hoc colore: Accessit, inquit, ad me pater obrutus sordibus, tremens deficientibus membris; rogavit alimenta. Interrogo vos, iudices, quid me, haec si fiant, facere oporteat. Nam istum non interrogo: scit quid facturus sim. Num patiemini ut alteri patri faciam iniuriam, alteri invidiam? Cum vetuisset me alimenta praestare, si qua est fides, non putavi illum ex animo vetare; lenocinatur, inquam, gloriae meae, ut videar patrem etiam prohibitus aluisse.

Marullus novo colore egit: Cecidit in pedes meos senex squalidus barba capilloque. Novit, inquam, nescio qui iste misericordiam meam. Adlevavi, cum ignorarem quis esset: vultis repellam quod pater est?

Cestius hoc colore: Haec mecum cogitavi: pater meus eget; egentem videt frater: non miseretur, non praestat alimenta; hoc est, inquam, non vult eripere filio officium, scit in hac fortuna meorum has iam meas esse partes. Hoc peccavi, quod non ultro ad patrem accessi; sed aiebam: nolo huic quicquam amplius praestare quam illi praestiti: non expectavi donec patruus ad me veniret? et nunc expectabo. Venit ad me pater: quid habui facere? perducere illum ad patruum? Non feci. Merito irascitur; potuit enim, si aluisset, levare quidem fortunam fratris, sed causam adgravare.

Buteonis colorem non probabat Latro: praestitisse se dixit exiguum, tantum quo spiritum posset producere; et cum descripsisset pallorem eius ac maciem, adiecit: apparet illum ab inimicis ali. Hunc colorem cum improbaret Latro hac sententia usus est: non est, inquit, abdicato quicquam ex gloria criminis sui detrahendum. 	

Hispanus hunc colorem venustius; nam et miserationi eius qui non benignissime alitur adiecit aliquid et pietati suae nihil detraxit: Quomodo autem, inquit, illum alo? exiguos furtive cibos mitto, et si quid de mensa mea detrahere potui, famelico seni porrigo. Non credis, qui scis quomodo te aluerim?

\bigskip

Colorem ex altera parte, quae durior est, Latro aiebat hunc sequendum, ut gravissimarum iniuriarum inexorabilia et ardentia induceremus odia Thyesteo more; aiebat patrem non irasci tantum debere sed furere. Ipse in declamatione usus est summis clamoribus illo versu tragico: 'cur fugis fratrem? scit ipse.' 	

Hunc colorem secutus Syriacus Vallius durum sensum videbatur non dure posuisse in narratione sic: infelicissimam ambo et tristissimam egimus vitam, excepto uno quod alter alterum egentem vidimus. Aeque efficaciter odium videbatur expressisse fraternum hac sententia: vos, iudices, audite quam valde eguerim: fratrem rogavi. 	

Hanc partem memini apud Cestium declamari ab Alfio Flavo, ad quem audiendum me fama perduxerat; qui cum praetextatus esset, tantae opinionis fuit ut populo Romano puer eloquentia notus esset. Semper de illius ingenio Cestius et praedicavit et timuit: aiebat tam inmature magnum ingenium non esse vitale; sed tanto concursu hominum audiebatur ut raro auderet post illum Cestius dicere. Ipse omnia mala faciebat ingenio suo; naturalis tamen illa vis eminebat, quae post multos annos, iam et desidia obruta et carminibus enervata, vigorem tamen suum tenuit. Semper autem commendabat eloquentiam eius aliqua res extra eloquentiam: in puero lenocinium erat ingenii aetas, in iuvene desidia.

Hic cum declamaret partem abdicantis, hanc summis clamoribus dixit sententiam: Quis es tu qui de facto patrum sententiam feras? ille tunc peccavit, tu nunc peccas. Ad te arbitrum odia nostra non mittimus: iudices habemus deos. Et illam sententiam: Audimus fratrum fabulosa certamina et incredibilia nisi nos fuissemus: impias epulas, detestabili parricidio fugatum diem: hoc uno modo iste frater a fratre ali meruit. Quam innocenter me contra parricidium vindico! filium illi suum reddo.

Cestius hunc colorem tam strictum non probavit, sed dixit temperandum esse, et ipse hoc colore usus est, quem statim a principio induxit: Miratur aliquis quod, cum duo gravissimam acceperimus iniuriam, ego et filius, ego solus irascor? Non est quod miretur: iam filio satisfactum est. Debuisti, inquit, me rogare ut ipse praestarem, debuisti illum ad me perducere, debuisti reconciliationem temptare, non famam pietatis ex nostra captare discordia. Fortasse ego cum egerem fratrem rogassem si tu non fuisses; fortasse ille me rogasset si tu non fuisses; poterit nobis convenire si non fuerit in medio quem potius miseri contumaces rogent.

\bigskip

Hermagoras in hac controversia transit a prooemio in narrationem eleganter, rarissimo quidem genere, ut in eadem re transitus esset, sententia esset, schema esset, sed, ut Latroni placebat, schema quod vulnerat, non quod titillat: * * * 

Ex altera parte transit a prooemio in narrationem Gallio et ipse per sententiam sic: quidni filium mihi nolim cum isto communem esse, cum quo utinam communem nec patrem habuissem? 

Diocles Carystius illum sensum a Latinis iactatum dixit brevissime, rarissimo genere, quo duobus sententia verbis consummatur (nec enim paucioribus potest): * * * 

Euctemon, levis declamator sed dulcis, dixit nove et amabiliter illum aeque ab omnibus vexatum sensum, quo reconciliatio fratrum temptatur: * * * 

