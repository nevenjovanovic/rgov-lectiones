%\section*{O autoru}


\section*{Senatus consultum Claudianum (oratio Claudii) \\de iure honorum Gallis dando (48 a.~D)}
\label{sec:claudius}


\dots\ equidem primam omnium illam cogitationem hominum quam maxime primam occursuram mihi provideo deprecor ne quasi novam istam rem introduci exhorrescatis sed illa potius cogitetis quam multa in hac civitate novata sint et quidem statim ab origine urbis nostrae in quod formas statusque res publica nostra diducta sit 

quondam reges tenuere urbem nec tamen domesticis successoribus eam tradere contigit; supervenere alieni et quidam externi ut Numa Romulo successerit ex Sabinis veniens, vicinus quidem, sed tunc externus; ut Anco Marcio Priscus Tarquinius. is propter temeratum sanguinem, quod patre Demaratho Corinthio natus erat et Tarquiniensi matre, generosa, sed inopi, ut quae tali marito necesse habuerit succumbere, cum domi repelleretur a gerendis honoribus, postquam Romam migravit, regnum adeptus est. huic quoque et filio nepotive eius (nam et hoc inter auctores discrepat) insertus Servius Tullius; si nostros sequimur, captiva natus Ocresia, si Tuscos, Caeli quondam Vivennae sodalis fidelissimus omnisque eius casus comes, postquam varia fortuna exactus cum omnibus reliquis Caeliani exercitus Etruria excessit, montem Caelium occupavit et a duce suo Caelio ita appellitatus mutatoque nomine (nam Tusce Mastarna ei nomen erat) ita appellatus est ut dixi, et regnum summa cum rei publicae utilitate optinuit. deinde postquam Tarquini Superbi mores invisi civitati nostrae esse coeperunt, qua ipsius, qua filiorum eius, nempe pertaesum est mentes regni et ad consules, annuos magistratus, administratio rei publicae translata est.

quid nunc commemorem dictaturae hoc ipso consulari imperium valentius repertum ad maiores nostros quo in asperioribus bellis aut in civile motu difficiliore uterentur? aut in auxilium plebis creatos tribunos plebei? quid a consulibus ad decemviros translatum imperium solutoque postea decemvirali regno ad consules rusus reditum? quid in pluris distributum consulare imperium tribunosque militum consulari imperio appellatos, qui seni et saepe octoni crearentur? quid communicatos postremo cum plebe honores, non imperii solum, sed sacerdotiorum quoque? iam si narrem bella, a quibus coeperint maiores nostri, et quo processerimus, vereor ne nimio insolentior esse videar et quaesisse iactationem gloriae prolati imperi ultra oceanum. sed illoc potius revertar. civitatem (\dots)

\dots\ potest. sane novo more et divus Augustus avonculus meus et patruus Tiberius Caesar omnem florem ubique coloniarum ac municipiorum, bonorum scilicet virorum et locupletium, in hac curia esse voluit. quid ergo? non Italicus senator provinciali potior est? iam vobis, cum hanc partem censurae meae adprobare coepero, quid de ea re sentiam, rebus ostendam. sed ne provinciales quidem, si modo ornare curiam poterint, reiciendos puto.

ornatissima ecce colonia valentissimaque Viennensium quam longo iam tempore senatores huic curiae confert! ex qua colonia inter paucos equestris ordinis ornamentum Lucium Vestinum familiarissime diligo et hodieque in rebus meis detineo, cuius liberi fruantur quaeso primo sacerdotiorum gradu, postmodo cum annis promoturi dignitatis sui incrementa; ut dirum nomen latronis taceam et odi illud palaestricum prodigium, quod ante in domum consulatum intulit quam colonia sua solidum civitatis Romanae benificium consecuta est. idem de fratre eius possum dicere, miserabili quidem indignissimoque hoc casu ut vobis utilis senator esse non possit (\dots)

tempus est, Tiberi Caesar Germanice, detegere te patribus conscriptis quo tendat oratio tua; iam enim ad extremos fines Galliae Narbonensis venisti (\dots)

tot ecce insignes iuvenes, quot intueor, non magis sunt paenitendi 
senatores quam paenitet Persicum, nobilissimum virum, amicum meum inter imagines maiorum suorum Allobrogici nomen legere. quodsi haec ita esse consentitis, quid ultra desideratis quam ut vobis digito demonstrem solum ipsum ultra fines provinciae Narbonensis iam vobis senatores mittere, quando ex Luguduno habere nos nostri ordinis viros non paenitet. timide quidem, patres conscripti, egressus adsuetos familiaresque vobis provinciarum terminos sum, sed destricte iam Comatae Galliae causa agenda est, in qua si quis hoc intuetur quod bello per decem annos exercuerunt divom Iulium idem opponat centum annorum immobilem fidem obsequiumque multis trepidis rebus nostris plus quam expertum. illi patri meo Druso Germaniam subigenti tutam quiete sua securamque a tergo pacem praestituerunt, et quidem cum a census novo tum opere et inadsueto Gallis ad bellum avocatus esset; quod opus quam
arduum sit nobis nunc cum maxime, quamvis nihil ultra quam ut
publice notae sint facultates nostrae, exquiratur, nimis magno
experimento cognoscimus.

\noindent (Textus: CIL XIII, n. 1668 (Epigraphik Datenbank Clauss/Slaby); \url{https://droitromain.univ-grenoble-alpes.fr/Senatus/Claudia.htm}.)

%\newpage

\section*{Tacitus, annales (110–120 a. D) 11, 23–24}
\label{sec:annales}

A.~Vitellio L.~Vipstano consulibus cum de supplendo senatu agitaretur primoresque Galliae, quae Comata appellatur, foedera et civitatem Romanam pridem adsecuti, ius adipiscendorum in urbe honorum expeterent, multus ea super re variusque rumor. et studiis diversis apud principem certabatur adseverantium non adeo aegram Italiam ut senatum suppeditare urbi suae nequiret. suffecisse olim indigenas consanguineis populis nec paenitere veteris rei publicae. quin adhuc memorari exempla quae priscis moribus ad virtutem et gloriam Romana indoles prodiderit. an parum quod Veneti et Insubres curiam inruperint, nisi coetus alienigenarum velut captivitas inferatur? quem ultra honorem residuis nobilium, aut si quis pauper e Latio senator foret? oppleturos omnia divites illos, quorum avi proavique hostilium nationum duces exercitus nostros ferro vique ceciderint, divum Iulium apud Alesiam obsederint. recentia haec: quid si memoria eorum moreretur qui sub Capitolio et arce Romana manibus eorundem perissent satis; fruerentur sane vocabulo civitatis; insignia patrum, decora magistratuum ne vulgarent.

His atque talibus haud permotus princeps et statim contra disseruit et vocato senatu ita exorsus est: »maiores mei, quorum antiquissimus Clausus origine Sabina simul in civitatem Romanam et in familias patriciorum adscitus est, hortantur uti paribus consiliis in re publica capessenda, transferendo huc quod usquam egregium fuerit. neque enim ignoro Iulios Alba, Coruncanios Camerio, Porcios Tusculo, et ne vetera scrutemur, Etruria Lucaniaque et omni Italia in senatum accitos, postremo ipsam ad Alpis promotam ut non modo singuli viritim, sed terrae, gentes in nomen nostrum coalescerent. tunc solida domi quies et adversus externa floruimus, cum Transpadani in civitatem recepti, cum specie deductarum per orbem terrae legionum additis provincialium validissimis fesso imperio subventum est. num paenitet Balbos ex Hispania nec minus insignis viros e Gallia Narbonensi transivisse? manent posteri eorum nec amore in hanc patriam nobis concedunt. quid aliud exitio Lacedaemoniis et Atheniensibus fuit, quamquam armis pollerent, nisi quod victos pro alienigenis arcebant? at conditor nostri Romulus tantum sapientia valuit ut plerosque populos eodem die hostis, dein civis habuerit. advenae in nos regnaverunt; libertinorum filiis magistratus mandare non, ut plerique falluntur, repens, sed priori populo factitatum est. at cum
Senonibus pugnavimus; scilicet Vulsci et Aequi numquam adversam nobis aciem instruxere. capti a Gallis sumus; sed et Tuscis obsides dedimus et Samnitium iugum subiimus. ac tamen, si cuncta bella recenseas, nullum breviore spatio quam adversus Gallos confectum; continua inde ac fida pax. iam moribus artibus adfinitatibus nostris mixti aurum et opes suas inferant potius quam separati habeant. omnia, patres conscripti, quae nunc vetustissima creduntur, nova fuere; plebeii magistratus post patricios, Latini post plebeios, ceterarum Italiae gentium post Latinos. inveterascet hoc quoque, et quod hodie exemplis tuemur, inter exempla erit.«
