%\section*{O autoru}

\section*{Apologeticum (197), 1, 4, 29-30}

Si non licet vobis, Romani imperii antistites, in aperto et edito, in ipso fere vertice civitatis praesidentibus ad iudicandum palam dispicere et coram examinare, quid sit liquido in causa Christianorum, si ad hanc solam speciem auctoritas vestra de iustitiae diligentia in publico aut timet aut erubescit inquirere, si denique, quod proxime accidit, domesticis iudiciis nimis operata infestatio sectae huius obstruit defensioni: liceat veritati vel occulta via tacitarum litterarum ad aures vestras pervenire.

Nihil de causa sua deprecatur, quia nec de condicione miratur. Scit se peregrinam in terris agere, inter extraneos facile inimicos invenire, ceterum genus, sedem, spem, gratiam, dignitatem in caelis habere. Unum gestit interdum, ne ignorata damnetur. Quid hic deperit legibus in suo regno dominantibus, si audiatur? An hoc magis gloriabitur potestas earum, quo etiam auditam damnabunt veritatem? Ceterum inauditam si damnent, praeter invidiam iniquitatis etiam suspicionem merebuntur alicuius conscientiae, nolentes audire, quod auditum damnare non possint.

Hanc itaque primam causam apud vos collocamus iniquitatis odii erga nomen Christianorum. Quam iniquitatem idem titulus et onerat et revincit, qui videtur excusare, ignorantia scilicet. Quid enim iniquius, quam ut oderint homines quod ignorant, etiam si res meretur odium? Tunc etenim meretur, cum cognoscitur an mereatur.  Vacante autem meriti notitia, unde odii iustitia defenditur, quae non de eventu, sed de conscientia probanda est? Cum ergo propterea oderunt homines, quia ignorant, quale sit quod oderunt, cur non liceat eiusmodi illud esse, quod non debeant odisse? Ita utrumque ex alterutro redarguimus, et ignorare illos, dum oderunt, et iniuste odisse, dum ignorant.

(\dots)

Atque adeo quasi praefatus haec ad suggillandam odii erga nos publici iniquitatem, iam de causa innocentiae consistam, nec tantum refutabo quae nobis obiciuntur, sed etiam in ipsos retorquebo, qui obiciunt, ut ex hoc quoque sciant homines in Christianis non esse quae in se nesciunt esse, simul uti erubescant accusantes, non dico pessimi optimos, se iam, ut volunt, compares suos. Respondebimus ad singula, quae in occulto admittere dicimur – quae illos palam admittentes invenimus – in quibus scelesti, in quibus vani, in quibus damnandi, in quibus irridendi deputamur.

Sed quoniam, cum ad omnia occurrit veritas nostra, postremo legum obstruitur auctoritas adversus eam, ut aut nihil dicatur retractandum esse post leges aut ingratis necessitas obsequii praeferatur veritati, de legibus prius concurram vobiscum ut cum tutoribus legum.

Iam primum, cum dure definitis dicendo: ``Non licet esse vos!'' et hoc sine ullo retractatu humaniore praescribitis, vim profitemini et iniquam ex arce dominationem, si ideo negatis licere, quia vultis, non quia debuit non licere. Quodsi, quia non debet, ideo non vultis licere, sine dubio id non debet licere quod bene fit. Si bonum invenero esse quod lex tua prohibuit, nonne ex illo praeiudicio prohibere me non potest, quod, si malum esset, iure prohiberet? Si lex tua erravit, puto, ab homine concepta est; neque enim de caelo ruit. Miramini hominem aut errare potuisse in lege condenda aut resipuisse in reprobanda? Non enim et ipsius Lycurgi leges a Lacedaemoniis emendatae tantum auctori suo doloris incusserunt, ut in secessu inedia de semet ipso iudicarit? Nonne et vos cottidie experimentis illuminantibus tenebras antiquitatis totam illam veterem et squalentem silvam legum novis principalium rescriptorum et edictorum securibus truncatis et caeditis? Nonne vanissimas Papias leges, quae ante liberos suscipi cogunt quam Iuliae matrimonium contrahi, post tantae auctoritatis senectutem heri Severus, constantissimus principum, exclusit? Sed et iudicatos in partes secari a creditoribus leges erant; consensu tamen publico crudelitas postea erasa est, in pudoris notam capitis poena conversa est. Bonorum adhibita proscriptio suffundere maluit hominis sanguinem quam effundere.

(\dots)

Si semper latemus, quando proditum est quod admittimus? Immo a quibus prodi potuit? Ab ipsis enim reis non utique, cum vel ex forma omnibus mysteriis silentii fides debeatur. Samothracia et Eleusinia reticentur: Quanto magis talia, quae prodita interim etiam humanam animadversionem provocabunt, dum divina servatur! Si ergo non ipsi proditores sui, sequitur ut extranei. Et unde extraneis notitia, cum semper etiam piae initiationes arceant profanos et $\langle$ab$\rangle$ arbitris caveant? Nisi si impii minus metuunt.

Natura famae omnibus nota est. Vestrum est: ``Fama malum, qua non aliud velocius ullum.'' Cur malum fama? Quia velox? Quia index? An quia plurimum mendax? Quae ne tunc quidem, cum aliquid veri affert, sine mendacii vitio est, detrahens, adiciens, demutans de veritate. Quid quod ea illi condicio est, ut non nisi cum mentitur, perseveret, et tamdiu vivit, quamdiu non probat? Siquidem, ubi probavit, cessat esse et quasi officio nuntiandi functa rem tradit; et exinde res tenetur, res nominatur. Nec quisquam dicit verbi gratia: ``Hoc Romae aiunt factum'', aut: ``Fama est illum provinciam sortitum'', sed: ``Sortitus est ille provinciam'', et: ``Hoc factum est Romae''. Fama, nomen incerti, locum non habet, ubi certum est. An vero famae credat nisi inconsideratus? Quia sapiens non credit incerto. Omnium est aestimare, quantacumque illa ambitione diffusa sit, quantacumque adseveratione constructa, quod ab uno aliquando principe exorta sit necesse est. Exinde in traduces linguarum et aurium serpit, et ita modici seminis vitium cetera rumoris obscurat, ut nemo recogitet, ne primum illud os mendacium seminaverit, quod saepe fit aut ingenio aemulationis aut arbitrio suspicionis aut non nova, sed ingenita quibusdam mentiendi voluptate. Bene autem quod omnia tempus revelat, testibus etiam vestris proverbiis atque sententiis, ex dispositione naturae, quae ita ordinavit, ut nihil diu lateat, etiam quod fama non distulit.

Merito igitur fama tamdiu conscia sola est scelerum Christianorum; hanc indicem adversus nos profertis, quae quod aliquando iactavit tantoque spatio in opinionem corroboravit, usque adhuc probare non valuit.

Ut fidem naturae ipsius appellem adversus eos, qui talia credenda esse praesumunt, ecce proponimus horum facinorum mercedem: Vitam aeternam repromittunt. Credite interim! De hoc enim quaero, an et qui credideris tanti habeas ad eam tali conscientia pervenire. Veni, demerge ferrum in infantem nullius inimicum, nullius reum, omnium filium; vel, si alterius officium est, tu modo adsiste morienti homini, antequam vixit; fugientem animam novam expecta, excipe rudem sanguinem, eo panem tuum satia, vescere libenter! Interea discumbens dinumera loca, ubi mater, ubi soror; nota diligenter, ut, cum tenebrae ceciderint caninae, non erres! Piaculum enim admiseris, nisi incestum feceris. Talia initiatus et consignatus vivis in aevum. Cupio respondeas, si tanti aeternitas; aut si non, ideo nec credenda. Etiamsi credideris, nego te velle; etiamsi volueris, nego te posse. Cur ergo alii possint, si vos non potestis? cur non possitis, si alii possunt? Alia nos, opinor, natura, Cynopennae aut Sciapodes; alii ordines dentium, alii ad incestam libidinem nervi. Qui ista credis de homine, potes et facere; homo es et ipse, quod et Christianus. Qui non potes facere, non debes credere. Homo est enim et Christianus, et quod et tu. 

(\dots)

Ventum est igitur ad secundum titulum laesae augustioris maiestatis, siquidem maiore formidine et callidiore timiditate Caesarem observatis quam ipsum de Olympo Iovem. Et merito, si sciatis. Quis enim ex viventibus quilibet non mortuo potior? Sed nec hoc vos ratione facitis potius quam respectu praesentaneae potestatis; adeo et in isto irreligiosi erga deos vestros deprehendemini, cum plus timoris humano dominio dicatis. Citius denique apud vos per omnes deos quam per unum genium Caesaris peieratur.

Constet igitur prius, si isti, quibus sacrificatur, salutem imperatoribus vel cuilibet homini impertire possunt, et ita nos crimini maiestatis addicite, si angeli aut daemones substantia pessimi spiritus beneficium aliquod operantur, si perditi conservant, si damnati liberant, si denique, quod in conscientia vestra est, mortui vivos tuentur. Nam utique suas primo statuas et imagines et aedes tuerentur, quae, ut opinor, Caesarum milites excubiis salva praestant. Puto autem, eae ipsae materiae de metallis Caesarum veniunt, et tota templa de nutu Caesaris constant. Multi denique dei habuerunt Caesarem iratum; facit ad causam, si et propitium, cum illis aliquid liberalitatis aut privilegii confert. Ita qui sunt in Caesaris potestate, cuius et toti sunt, quomodo habebunt salutem Caesaris in potestate, ut eam praestare posse videantur, quam facilius ipsi a Caesare consequantur?

Ideo ergo committimus in maiestatem imperatorum, quia illos non subicimus rebus suis, quia non ludimus de officio salutis ipsorum, qui eam non putamus in manibus esse plumbatis! Sed vos irreligiosi, qui eam quaeritis ubi non est, petitis a quibus dari non potest, praeterito eo, in cuius est potestate, insuper eos debellatis, qui eam sciunt petere, qui etiam possunt impetrare, dum sciunt petere!

Nos enim pro salute imperatorum deum invocamus aeternum, deum verum, deum vivum, quem et ipsi imperatores propitium sibi praeter ceteros malunt. Sciunt, quis illis dederit imperium; sciunt, qua homines, quis et animam; sentiunt eum esse deum solum, in cuius solius potestate sint, a quo sint secundi, post quem primi, ante omnes et super omnes deos. Quidni? Cum super omnes homines, qui utique vivunt et mortuis antistant. 
