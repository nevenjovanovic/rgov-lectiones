%\section*{O autoru}

\section*{De obitu Theodosii oratio (395), 1-5}

Hoc nobis motus terrarum graves, hoc iuges pluviae minabantur, et ultra solitum caligo tenebrosior denuntiabat, quod clementissimus imperator Theodosius excessurus esset e terris. Ipsa igitur excessum eius elementa moerebant. Coelum tenebris obductum, aer perpeti horrens caligine, terra quatiebatur motibus, replebatur aquarum alluvionibus. Quidni mundus ipse defleret eum principem continuo esse rapiendum, per quem dura mundi istius temperari solerent; cum criminum poenas indulgentia praeveniret.

Et ille quidem abiit accipere sibi regnum, quod non deposuit, sed mutavit in tabernacula Christi iure pietatis adscitus, in illam Hierusalem supernam, ubi nunc positus dicit: Sicut audivimus, ita et vidimus in civitate Domini virtutum, in civitate Dei nostri, quam Deus fundavit in aeternum. Sed plurimos tamquam paterno destitutos praesidio dereliquit, ac potissimum filios. Sed non sunt destituti, quos pietatis suae reliquit haeredes: non sunt destituti, quibus Christi acquisivit gratiam, et exercitus fidem: cui documento fuit Deum favere pietati, ultoremque esse perfidiae.

Eius ergo principis et proxime conclamavimus obitum, et nunc quadragesimam celebramus, assistente sacris altaribus Honorio principe; quia sicut sanctus Ioseph patri suo Iacob quadraginta diebus humationis officia detulit, ita et hic Theodosio patri iusta persolvit. Et quia alii tertium diem et trigegesimum, alii septimum et quadragesimum observare consueverunt, quid doceat lectio, consideremus. Defuncto itaque Iacob, praecepit, inquit, Ioseph pueris sepultoribus, ut sepelirent eum. Et sepelierunt sepultores Israel, et repleti sunt ei quadraginta dies; sic enim dinumerantur dies sepulturae. Et luxit eum Aegyptus septuaginta diebus. Haec ergo sequenda solemnitas, quam praescribit lectio. Sed etiam in Deuteronomio scriptum est: Quia planxerunt filii Israel Moysen diebus triginta, et consummati sunt dies luctus. Utraque ergo observatio habet auctoritatem; quia necessarium pietatis impletur officium.

Bonus itaque Ioseph, qui formam pio muneri dedit, quem amabat pater, cuique dixit: Adiuvet te Deus meus, et benedicat te benedictione terrae habentis omnia, propter benedictionem mamillarum et vulvae, benedictiones matris tuae, et propter benedictiones patris tui: pii patris soboles bona. Celebrat ergo et iste quadragesimam patris Iacob, supplantatoris illius; et nos celebramus Theodosii quadragesimam, qui imitatus Iacob, supplantavit perfidiam tyrannorum, qui abscondit simulacra gentium; omnes enim cultus idolorum fides eius abscondit, omnes eorum ceremonias oblitteravit: qui etiam iis qui in se peccaverant, doluit quam dederat periisse indulgentiam, et veniam denegatam. Sed non negabunt filii, quod donavit pater: non negabunt, etiamsi quidam interturbare conatus sit; neque enim poterunt negare, quod in commune donavit, qui solvunt, quod singulis dedit.

Nihil gloriosius exitus tanti principis habuit, qui omnia iam filiis tradidisset, regnum, potestatem, nomen augusti: nihil, inquam, speciosius ei in morte servatum est, quam quod in aliquantis promissa annonarum exigendarum relaxatio, dum moratur, facta est successio eius indulgentiarum haereditas; ut ille qui voluit impedire, sibi odium fecerit, Theodosio tamen tantae cumulus gratiae non sit ademptus. Nec immerito, si enim privatorum ultimae voluntates, et deficientium testamenta habent perpetem firmitatem; quomodo potest tanti principis esse irritum testamentum? Gloriosius quoque in eo Theodosius, qui non communi iure testatus est; de filiis enim nihil habebat novum quod conderet, quibus totum dederat, nisi ut eos praesenti commendaret parenti: et de subditis sibi et commissis testari debuit, ut legata dimitteret, fidei commissa signaret. Praecepit dari legem indulgentiae, quam scriptam reliquit. Quid dignius, quam ut testamentum imperatoris lex sit?

Ergo tantus imperator recessit a nobis, sed non totus recessit; reliquit enim nobis liberos suos, in quibus eum debemus agnoscere, et in quibus eum et cernimus et tenemus. Nec moveat aetas: fides militum imperatoris perfecta est aetas; est enim perfecta aetas, ubi perfecta est virtus. Reciproca haec; quia et fides imperatoris militum virtus est.
