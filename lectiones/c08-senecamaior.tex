%\section*{O autoru}

\section*{L. Annaei Senecae Maioris controversiarum liber secundus, 2}

Vir et uxor iuraverunt, ut, si quid alteri optigisset, alter moreretur. vir peregre profectus misit nuntium ad uxorem, qui diceret decessisse virum. uxor se praecipitavit. recreata iubetur a patre relinquere virum; non vult. abdicatur.

\medskip


PORCI LATRONIS. Dii immortales, qua debetis providentia humanum genus regitis: effecistis, ut illud non periculum esset amantis sed experimentum.

CESTI PII. Antequam iuraremus, diu haec inter nos tacita lex erat. Si abdicata fuerit, non vivet. hoc illi pater non credit; non est novum: nec vir credidit. Quaeris, quod ius iurandum fuerit? `ita patri placeam.'

FVSCI ARELLI patris. Iam, uxor, non navigabo, non peregrinabor: fides tua me timidum fecit. `Mentitus' inquit `hic': sollemne est amantibus; ideo non nisi iurantibus credimus. Hos dividere vult socer, quos ne mors quidem dividet? `Moriar; habeo et causam et exemplum. quaedam ardentibus rogis se maritorum immiserunt, quaedam vicaria maritorum salutem anima redemerunt. quam magna gloria brevi sollicitudine pensata est!' o te felicem, uxor! inter has viva numeraris.

MARVLLI. Adsiduae contentiones erant: `ego magis amo'; `immo ego'. `sine te vivere non possum'; `immo ego sine te'. qui solet exitus esse certantium, iuravimus. respexistis nos, di, quos numquam violavimus.

(\dots)

Hanc controversiam memini ab OVIDIO NASONE declamari apud rhetorem Arellium Fuscum, cuius auditor fuit, cum diversum sequeretur dicendi genus‚ nam Latronis admirator erat. habebat ille comptum et decens et amabile ingenium. oratio eius iam tum nihil aliud poterat videri quam solutum carmen. adeo autem studiose Latronem audit ut multas illius sententias in versus suos transtulerit. in armorum iudicio dixerat Latro: `mittamus arma in hostis et petamus'. Naso dixit:
\begin{verse}
arma viri fortis medios mittantur in hostis;\\
inde iubete peti.
\end{verse}
et alium ex illa suasoria sensum aeque a Latrone mutuatus est: memini Latronem in praefatione quadam dicere (quod scholastici quasi carmen didicerant): `non vides, ut immota fax torpeat, ut exagitata reddat ignes? mollit viros otium, ferrum situ carpitur et rubiginem ducit, desidia dedocet'. Naso dixit:
\begin{verse}
vidi ego iactatas mota face crescere flammas\\
et rursus nullo concutiente mori.
\end{verse}
tunc autem, cum studeret, habebatur bonus declamator. hanc certe controversiam ante Arellium Fuscum declamavit, ut mihi videbatur, longe ingeniosius, excepto eo, quod sine certo ordine per locos discurrebat. haec illo dicente excepta memini: 

\begin{quotation}
quidquid laboris est, in hoc est, ut uxori virum et uxorem viro diligere concedas. necesse est deinde iurare permittas, si amare permiseris. Quod habuisse nos ius iurandum putas? tu nobis religiosum nomen fuisti: si mentiremur, illa sibi iratum patrem invocavit, ego socerum. parce, pater: non peieravimus. Ecce obiurgator nostri quam effrenato amore fertur! queritur quemquam esse filiae praeter se carum. quid est, quod illum ab indulgentia sua avocet? di boni, quomodo hic amavit uxorem? Amat filiam et abdicat; dolet periclitatam esse, et ab eo abducit, sine quo negat se posse vivere: quaeritur periculum eius, qua paene caruit, hic qui amare caute iubet! Facilius in amore finem impetres quam modum. tu hoc obtinebis, ut terminos quasi adprobaturi custodiant, ut nihil faciant nisi considerate, nihil promittant nisi utique facturi, omnia verba ratione et fide ponderent? senes sic amant. pauca nosti, pater, crimina: et litigavimus aliquando et cecidimus et fortasse, quod non putas, peieravimus. quid ad patrem pertinet quod amantes iurant? si vis credere, nec ad deos pertinet. Non est, quod tibi placeas, uxor, tamquam prima peccaveris: perit aliqua cum viro, perit aliqua pro viro. illas tamen omnis aetas honorabit, omne celebrabit ingenium. fer, socer, felicitatem tuam: magnum tibi quam parvo constat exemplum! in reliquum, ut iubes, diligentiores facti sumus; errorem nostrum confitemur: exciderat iurantibus esse tertium, qui magis amaret; sic, di, sit semper. Perseveras, socer? recipe filiam. ego, qui peccavi, poena dignus sum. quare uxori notae causa sim, socero orbitatis? discedam e civitate, fugiam, exulabo; utcumque potero, desiderium misera et crudeli patientia perferam. morerer, si solus moriturus essem.
\end{quotation}

Declamabat autem Naso raro controversias et non nisi ethicas. libentius dicebat suasorias. molesta illi erat omnis argumentatio. verbis minime licenter usus est, nisi in carminibus, in quibus non ignoravit vitia sua sed amavit.

\section*{L. Annaei Senecae Maioris \\excerpta controversiarum libri tertii}

Seneca Novato, Senecae, Melae filiis salutem

\medskip

\noindent Quosdam disertissimos cognovi viros non respondentes famae suae, cum declamarent; in foro maxima omnium admiratione dicentes, simul ad has domesticas exercitationes secesserant, desertos ab ingenio suo. quod accidere plerisque aeque mihi mirum quam certum est. 

(\dots)
	
Passienus noster cum coepit dicere, secundum principium statim fuga fit, ad epilogum omnes revertimur; media tantum quibus necesse est audiunt. miraris eundem non aeque bene declamare quam causas agere, aut eundem non tam bene suasorias quam iudiciales controversias dicere? Silo Pompeius sedens et facundus et litteratus est et haberetur disertus, si a praelocutione dimitteret; declamat tam male, ut videar belle optasse, cum dixi: numquam surgas. magna et varia res est eloquentia neque adhuc ulli sic indulsit, ut tota contingeret; satis felix est qui in aliquam eius partem receptus est. 

ego tamen et propriam causam videor posse reddere: adsuevi non auditorem spectare sed iudicem; adsuevi non mihi respondere sed adversario; non minus devito supervacua dicere quam contraria. in scholastica quid non supervacuum est, cum ipsa supervacua sit? indicabo tibi affectum meum: cum in foro dico, aliquid ago; cum declamo, id quod bellissime Censorinus aiebat de his, qui honores in municipiis ambitiose peterent, videor mihi in somnis laborare. deinde res ipsa diversa est: totum aliud est pugnare, aliud ventilare. hoc ita semper habitum est, scholam quasi ludum esse, forum arenam, et ille ideo primum in foro verba fracturus tiro dictus est. 

agedum istos declamatores produc in senatum, in forum: cum loco mutabuntur. velut adsueta clauso et delicatae umbrae corpora sub divo stare non possunt, non imbrem ferre, non solem sciunt; vix se inveniunt. adsuerunt enim suo arbitrio diserti esse. non est, quod oratorem in hac puerili exercitatione spectes. quid, si velis gubernatorem in piscina aestimare? 

diligentius me tibi excusarem, tamquam huic rei non essem natus, nisi scirem et Pollionem Asinium et Messalam Corvinum et Passienum, qui nunc primo loco stat, minus bene videri dicere quam Cestium aut Latronem. utrum ergo putas hoc dicentium vitium esse an audientium? non illi peius dicunt, sed hi corruptius iudicant. pueri fere aut iuvenes scholas frequentant; hi non tantum disertissimis viris, quos paulo ante rettuli, Cestium suum praeferunt, sed etiam Ciceroni praeferrent, nisi lapides timerent. quo tamen uno modo possunt, praeferunt; huius enim declamationes ediscunt, illius orationes non legunt nisi eas, quibus Cestius rescripsit.

memini me intrare scholam eius, cum recitaturus esset in Milonem. Cestius ex consuetudine sua miratus dicebat: `si Thraex essem, Fusius essem; si pantomimus essem, Bathyllus essem; si equus, Melissio.' non continui bilem et exclamavi: `si cloaca esses, maxima esses!' risus omnium ingens: scholastici intueri me, quis essem, qui tam crassas cervices haberem. Cestius Ciceroni responsurus mihi quod responderet non invenit sed negavit se executurum, nisi exissem de domo. ego negavi me de balneo publico exiturum, nisi lotus essem.

