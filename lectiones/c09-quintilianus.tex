%\section*{O autoru}


\section*{Argumenta librorum}

\subsection*{DE INSTITUTIONE ORATORIA LIBRI PRIMI ARGUMENTUM}

M.~Fabius Quintilianus Tryphoni Salutem. – Ad Marcellum Victorium prooemium. – Cap. I. Quid circa primam pueri institutionem providendum sit. A quali nutrice educandum filium pater curet, et quales praeceptores primos habere debeat. – II. Utrum utilius domi, an in scholis erudiatur. – III. Qua ratione puerorum ingenia dignoscantur, et quomodo tractanda. – IV. De grammatica, de litteris et earum potestate, ac orationis partibus. Declinare inprimis nomina, et verba scire oportet pueros, et genera nosse, ac casus. Tum de verbis et verborum appellationibus. – V. De virtutibus et vitiis orationis. – VI. De verbis propriis ac translatis, usitatis et novis, et de quatuor quibus sermo constat. De vetustis verbis, et quibus verbis sit utendum. De auctoritate. De consuetudine. – VII. De orthographia. – VIII. De lectione pueri. – IX. De officio grammatici, et quae primordia sint dicendi. – X. An oratori futuro necessaria sit plurium artium scientia. – XI. De prima pronuntiationis et gestus institutione. – XII. An plura eodem tempore doceri prima aetas possit.

\subsection*{DE INSTITUTIONE ORATORIA LIBRI SECUNDI ARGUMENTUM}

Cap. I. Quando sit rhetori tradendus puer. – II. De moribus et officiis praeceptoris. – III. An protinus praeceptore optimo sit utendum. – IV. De primis apud rhetorem exercitationibus. – V. De lectione oratorum et historicorum apud rhetorem. VI. De divisione. VII. De ediscendo. VIII. An secundum sui quisque ingenii docendus sit naturam. – IX. De officio discipulorum. X. De utilitate et ratione declamandi. — XI. An artis hujus necessaria cognitio sit. — XII. Quare eruditi ingeniosiores vulgo habeantur. XIII. Quis modus sit in arte. — XIV. Rhetorices etymon, et totius operis divisio. – XV. Quid sit rhetorice, et quis ejus finis. – XVI. An sit utilis rhetorice. – XVII. An rhetorice sit ars. XVIII, Generalis artium divisio, et ex quibus sit rhetorica. — XIX. Utra plus conferat eloquentiae ars, an natura. XX. An rhetorice virtus sit. – XXI. Quae sit materia rhetorices.

\subsection*{LIBRI TERTII ARGUMENTUM}

CAP. I. De scriptoribus artis rhetoricae. – II. Quod initium rhetorices. – III. Quinque esse partes rhetoricae. – IV. Tria esse genera causarum. V. Quibus contineatur omnis ratio dicendi. — VI. Quid sit status; unde ducatur; reus an actor eum faciat; quot et qui sint. VII. De demonstrativo, quod constat laude et vituperatione. – VIII. De suasoria, seu deliberativa et prosopopoeia. — IX. De partibus causarum judicialium. – X. De generibus judicialium controversiarum. – XI. Quid sit quaestio, ratio, judicatio, continens, et quatenus necessaria.

\subsection*{LIBRI QUARTI ARGUMENTUM}

PROOEMIUM. CAP. I. De exordio. – II. De narratione. – III. De egressione, seu de excursione. – IV. De propositione. V. De partitione.

\subsection*{LIBRI QUINTI ARGUMENTUM}

PROLOGUS. — Cap. I. De probationum divisione. — II. De praejudiciis. – III. De rumore et fama. – IV. De tormentis. – V. De tabulis. – VI. De jurejurando. – VII. De testibus. – VIII. De probatione artificiali. — IX. De signis. — X. De argumentis. — XI. De exemplis. – XII. De usu argumentorum. – XIII, De refutatione. — XIV. Quid enthymema, et quotuplex: quibus constet epicherema, et quomodo refellatur.

\subsection*{LIBRI SEXTI ARGUMENTUM}

PROOEMIUM, in quo de fortuna sua queritur. – Cap. I. De conclusione, seu peroratione. – II. De divisione affectuum: et quomodo movendi sint. – III. De risu. – IV. De altercatione. – V. De judicio et consilio.

\subsection*{LIBRI SEPTIMI ARGUMENTUM}

PRAEFATIO, de dispositionis utilitate. – Cap. I. De dispositione facienda. – II. De conjectura. – III. De finitione. – IV. De qualitate. – V. De actionis quaestione. — VI. De statu ex scripto et voluntate. – VII. De contrariis legibus. – VIII. De syllogismo seu ratiocinatione. – IX. De ambiguo seu amphibolia. — X. Quae sit inter status cognatio, et quae diversitas.

\subsection*{LIBRI OCTAVI ARGUMENTUM}

PROOEMIUM. — Cap. I. Quae in elocutione spectanda. — II. De perspicuitate. – III. De ornatu. – IV. De amplificatione. – V. De generibus sententiarum. – VI. De tropis.

\subsection*{LIBRI NONI ARGUMENTUM}

Cap. I. Quo differant figurae a tropis. – II. De figuris sententiarum. – III. De figuris verborum. — IV. De compositione.

\subsection*{LIBRI DECIMI ARGUMENTUM}

CAP. I. De copia verborum. — II. De imitatione. — III. Quo modo scribendum sit. IV. De emendatione. – Quae scribenda sint praecipue. – VI. De cogitatione. – VII. Quemadmodum extemporalis facultas paretur et contineatur.

\subsection*{LIBRI UNDECIMI ARGUMENTUM}

CAP. I. Praefatio, et de apte dicendo. – II. De memoria. – III. De pronunciatione.

\subsection*{LIBRI DUODECIMI ARGUMENTUM}

PROOEMIUM. Cap. I. Non posse oratorem esse, nisi virum bonum. — II. Cognoscenda esse oratori quibus mores formentur. III. Necessariam juris civilis oratori esse scientiam. – IV. Necessariam oratori cognitionem historiarum. – V. Quae sint oratori instrumenta. VI. Quod sit incipiendi causas agere tempus. VII. Quae in suscipiendis causis oratori observanda sint. VIII. Quae in discendis causis oratori observanda sint. IX. Quae servanda in agendis causis oratori sint. X. De genere dicendi. – XI. Quae post finem studia.


\section*{Ad Marcellum Victorium prohoemium}

Post impetratam studiis meis quietem, quae per viginti annos erudiendis iuvenibus impenderam, cum a me quidam familiariter postularent, ut aliquid de ratione dicendi componerem, diu sum equidem reluctatus, quod auctores utriusque linguae clarissimos non ignorabam multa, quae ad hoc opus pertinerent, diligentissime scripta posteris reliquisse. Sed qua ego ex causa faciliorem mihi veniam meae deprecationis arbitrabar fore, hac accendebantur illi magis, quod inter diversas opiniones priorum et quasdam etiam inter se contrarias difficilis esset electio; ut mihi si non inveniendi nova at certe iudicandi de veteribus iniungere laborem non iniuste viderentur.

(\dots)

Nam ceteri fere qui artem orandi litteris tradiderunt ita sunt exorsi quasi perfectis omni alio genere doctrinae summam †in eloquentiae† manum imponerent, siue contemnentes tamquam parua quae prius discimus studia, siue non ad suum pertinere officium opinati, quando diuisae professionum vices essent, seu, quod proximum vero, nullam ingenii sperantes gratiam circa res etiamsi necessarias, procul tamen ab ostentatione positas, ut operum fastigia spectantur, latent fundamenta. Ego cum existimem nihil arti oratoriae alienum sine quo fieri non posse oratorem fatendum est, nec ad ullius rei summam nisi praecedentibus initiis perveniri, ad minora illa, sed quae si neglegas non sit maioribus locus, demittere me non recusabo, nec aliter quam si mihi tradatur educandus orator studia eius formare ab infantia incipiam. Quod opus, Marcelle Vitori, tibi dicamus, quem cum amicissimum nobis tum eximio litterarum amore flagrantem non propter haec modo, quamquam sint magna, dignissimum hoc mutuae inter nos caritatis pignore iudicabamus, sed quod erudiendo Getae tuo, cuius prima aetas manifestum iam ingenii lumen ostendit, non inutiles fore libri videbantur quos ab ipsis dicendi uelut incunabulis per omnes quae modo aliquid oratori futuro conferant artis ad summam eius operis perducere festinabimus, atque eo magis quod duo iam sub nomine meo libri ferebantur artis rhetoricae neque editi a me neque in hoc comparati. Namque alterum sermonem per biduum habitum pueri quibus id praestabatur exceperant, alterum pluribus sane diebus, quantum notando consequi potuerant, interceptum boni iuvenes sed nimium amantes mei temerario editionis honore vulgauerant. Quare in his quoque libris erunt eadem aliqua, multa mutata, plurima adiecta, omnia uero compositiora et quantum nos poterimus elaborata.

Oratorem autem instituimus illum perfectum, qui esse nisi vir bonus non potest, ideoque non dicendi modo eximiam in eo facultatem sed omnis animi virtutes exigimus. Neque enim hoc concesserim, rationem rectae honestaeque vitae, ut quidam putaverunt, ad philosophos relegandam, cum vir ille vere civilis et publicarum privatarumque rerum administrationi accommodatus, qui regere consiliis urbes, fundare legibus, emendare iudiciis possit, non alius sit profecto quam orator. 

Quare, tametsi me fateor usurum quibusdam quae philosophorum libris continentur, tamen ea iure vereque contenderim esse operis nostri proprieque ad artem oratoriam pertinere. An, si frequentissime de iustitia fortitudine temperantia ceterisque similibus disserendum est, adeo ut vix ulla possit causa reperiri in quam non aliqua ex his incidat quaestio, eaque omnia inventione atque elocutione sunt explicanda, dubitabitur, ubicumque vis ingenii et copia dicendi postulatur, ibi partes oratoris esse praecipuas? fueruntque haec, ut Cicero apertissime colligit, quemadmodum iuncta natura, sic officio quoque copulata, ut idem sapientes atque eloquentes haberentur. Scidit deinde se studium, atque inertia factum est ut artes esse plures viderentur. 

Nam ut primum lingua esse coepit in quaestu institutumque eloquentiae bonis male uti, curam morum qui diserti habebantur reliquerunt; ea vero destituta infirmioribus ingeniis velut praedae fuit. Inde quidam contempto bene dicendi labore ad formandos animos statuendasque vitae leges regressi partem quidem potiorem, si dividi posset, retinuerunt, nomen tamen sibi insolentissimum adrogaverunt, ut soli studiosi sapientiae vocarentur; quod neque summi imperatores neque in consiliis rerum maximarum ac totius administratione rei publicae clarissime versati sibi umquam vindicare sunt ausi; facere enim optima quam promittere maluerunt. 

Ac veterum quidem sapientiae professorum multos et honesta praecepisse et, ut praeceperint, etiam vixisse facile concesserim; nostris vero temporibus sub hoc nomine maxima in plerisque vitia latuerunt. Non enim virtute ac studiis ut haberentur philosophi laborabant, sed vultum et tristitiam et dissentientem a ceteris habitum pessimis moribus praetendebant. Haec autem quae velut propria philosophiae adseruntur, passim tractamus omnes. Quis enim non de iusto, aequo ac bono, modo non et vir pessimus, loquitur? Quis non etiam rusticorum aliqua de causis naturalibus quaerit? Nam verborum proprietas ac differentia omnibus qui sermonem curae habent debet esse communis. Sed ea et sciet optime et eloquetur orator; qui si fuisset aliquando perfectus, non a philosophorum scholis virtutis praecepta peterentur. 

Nunc necesse est ad eos auctores recurrere, qui desertam, ut dixi, partem oratoriae artis, meliorem praesertim occupaverunt, et velut nostrum reposcere, non ut illorum nos utamur inventis, sed ut illos alienis usos esse doceamus. 

Sit igitur orator vir talis qualis vere sapiens appellari possit, nec moribus modo perfectus (nam id mea quidem opinione, quamquam sunt qui dissentiant, satis non est), sed etiam scientia et omni facultate dicendi; qualis fortasse nemo adhuc fuerit, sed non ideo minus nobis ad summa tendendum est: quod fecerunt plerique veterum, qui, etsi nondum quemquam sapientem repertum putabant, praecepta tamen sapientiae tradiderunt. Nam est certe aliquid consummata eloquentia neque ad eam pervenire natura humani ingenii prohibet. Quod si non contingat, altius tamen ibunt qui ad summa nitentur quam qui praesumpta desperatione quo velint evadendi protinus circa ima substiterint. 

Quo magis impetranda erit venia si ne minora quidem illa, verum operi quod instituimus necessaria, praeteribo. Nam liber primus ea quae sunt ante officium rhetoris continebit. Secundo prima apud rhetorem elementa et quae de ipsa rhetorices substantia quaeruntur tractabimus. Quinque deinceps inventioni (nam huic et dispositio subiungitur), quattuor elocutioni, in cuius partem memoria ac pronuntiatio veniunt, dabuntur. Vnus accedet in quo nobis orator ipse informandus est: ubi qui mores eius, quae in suscipiendis discendis agendis causis ratio, quod eloquentiae genus, quis agendi debeat esse finis, quae post finem studia, quantum nostra valebit infirmitas disseremus. 

His omnibus admiscebitur, ut quisque locus postulabit, docendi ratio quae non eorum modo scientia quibus solis quidam nomen artis dederunt studiosos instruat et, ut sic dixerim, ius ipsum rhetorices interpretetur, sed alere facundiam, vires augere eloquentiae possit. Nam plerumque nudae illae artes nimiae subtilitatis adfectatione frangunt atque concidunt quidquid est in oratione generosius, et omnem sucum ingenii bibunt et ossa detegunt, quae ut esse et adstringi nervis suis debent, sic corpore operienda sunt. Ideoque nos non particulam illam, sicuti plerique, sed quidquid utile ad instituendum oratorem putabamus in hos duodecim libros contulimus, breviter omnia demonstraturi: nam si quantum de quaque re dici potest persequamur, finis operis non reperietur.

\section*{E libro III}

\subsection*{3, 4 [De progymnasmatum schola.]}
 

Hinc iam quas primas in docendo partis rhetorum putem tradere incipiam, dilata parumper illa quae sola vulgo uocatur arte rhetorica; ac mihi oportunus maxime videtur ingressus ab eo cuius aliquid simile apud grammaticos puer didicerit.

Et quia narrationum, excepta qua in causis utimur, tris accepimus species, fabulam, quae uersatur in tragoediis atque carminibus non a veritate modo sed etiam a forma veritatis remota, argumentum, quod falsum sed vero simile comoediae fingunt, historiam, in qua est gestae rei expositio, grammaticis autem poeticas dedimus; apud rhetorem initium sit historica, tanto robustior quanto verior. Sed narrandi quidem quae nobis optima ratio videatur tum demonstrabimus cum de iudiciali parte dicemus; interim admonere illud sat est, ut sit ea neque arida prorsus atque ieiuna (nam quid opus erat tantum studiis laboris inpendere si res nudas atque inornatas indicare satis uideretur?), neque rursus sinuosa et arcessitis descriptionibus, in quas plerique imitatione poeticae licentiae ducuntur, lasciviat.

(\dots)

Sed ut eo revertar unde sum egressus; narrationes stilo componi quanta maxima possit adhibita diligentia volo. (\dots) Erit suum parandae facilitati tempus, nec a nobis neglegenter locus iste transibitur. Interim satis est si puer omni cura et summo, quantum illa aetas capit, labore aliquid probabile scripserit; in hoc adsuescat, huius sibi rei naturam faciat. Ille demum in id quod quaerimus aut ei proximum poterit evadere qui ante discet recte dicere quam cito.

Narrationibus non inutiliter subiungitur opus destruendi confirmandique eas, quod \textgreek[variant=ancient]{ἀνασκευή} et \textgreek[variant=ancient]{κατασκευή} uocatur. Id porro non tantum in fabulosis et carmine traditis fieri potest, uerum etiam in ipsis annalium monumentis; ut, si quaeratur »an sit credibile super caput Valeri pugnantis sedisse coruum, qui os oculosque hostis Galli rostro atque alis euerberaret«, sit in utramque partem ingens ad dicendum materia: aut de serpente, quo Scipio traditur genitus, et lupa Romuli et Egeria Numae; nam Graecis historiis plerumque poeticae similis licentia est. Saepe etiam quaeri solet de tempore, de loco, quo gesta res dicitur, nonnumquam de persona quoque, sicut Livius frequentissime dubitat et alii ab aliis historici dissentiunt.

Inde paulatim ad maiora tendere incipiet, laudare claros viros et vituperare improbos; quod non simplicis utilitatis opus est. Namque et ingenium exercetur multiplici variaque materia et animus contemplatione recti pravique formatur, et multa inde cognitio rerum uenit exemplisque, quae sunt in omni genere causarum potentissima, iam tum instruit cum res poscet usurum.

\section*{E libro XII}

\subsection*{[Prohoemium de operis difficultate.]}
 

Ventum est ad partem operis destinati longe gravissimam; cuius equidem onus si tantum opinione prima concipere potuissem quanto me premi ferens sentio, maturius consuluissem vires meas. Sed initio pudor omittendi quae promiseram tenuit, mox, quamquam per singulas prope partis labor cresceret, ne perderem quae iam effecta erant per omnes difficultates animo me sustentavi. Quare nunc quoque, licet maior quam umquam moles premat, tamen prospicienti finem mihi constitutum est vel deficere potius quam desperare. Fefellit autem quod initium a parvis ceperamus; mox velut aura sollicitante provecti longius, dum tamen nota illa et plerisque artium scriptoribus tractata praecipimus nec adhuc a litore procul videbamur et multos circa velut isdem se ventis credere ausos habebamus; iam cum eloquendi rationem novissime repertam paucissimisque temptatam ingressi sumus, rarus qui tam procul a portu recessisset reperiebatur; postquam vero nobis ille quem instituebamus orator, a dicendi magistris dimissus, aut suo iam impetu fertur aut maiora sibi auxilia ex ipsis sapientiae penetralibus petit, quam in altum simus ablati sentire coepimus. Nunc »caelum undique et undique pontus«. Vnum modo in illa inmensa vastitate cernere videmur M. Tullium, qui tamen ipse, quamvis tanta atque ita instructa nave hoc mare ingressus, contrahit vela inhibetque remos et de ipso demum genere dicendi quo sit usurus perfectus orator satis habet dicere. At nostra temeritas etiam mores ei conabitur dare et adsignabit officia. Ita nec antecedentem consequi possumus et longius eundum est ut res feret. Probabilis tamen cupiditas honestorum, et velut tutioris audentiae est temptare quibus paratior venia est.
 

\subsection*{12, 1 [Non posse oratorem esse nisi virum bonum.]}
 

Sit ergo nobis orator quem constituimus is qui a M.~Catone finitur vir bonus dicendi peritus, verum, id quod et ille posuit prius et ipsa natura potius ac maius est, utique vir bonus; id non eo tantum quod, si vis illa dicendi malitiam instruxerit, nihil sit publicis privatisque rebus perniciosius eloquentia, nosque ipsi, qui pro virili parte conferre aliquid ad facultatem dicendi conati sumus, pessime mereamur de rebus humanis si latroni comparamus haec arma, non militi. Quid de nobis loquor? Rerum ipsa natura, in eo quod praecipue indulsisse homini videtur quoque nos a ceteris animalibus separasse, non parens sed noverca fuerit si facultatem dicendi sociam scelerum, adversam innocentiae, hostem veritatis invenit. Mutos enim nasci et egere omni ratione satius fuisset quam providentiae munera in mutuam perniciem conuertere. Longius tendit hoc iudicium meum. Neque enim tantum id dico, eum qui sit orator virum bonum esse oportere, sed ne futurum quidem oratorem nisi virum bonum. Nam certe neque intellegentiam concesseris iis qui proposita honestorum ac turpium via peiorem sequi malent, neque prudentiam, cum in grauissimas frequenter legum, semper vero malae conscientiae poenas a semet ipsis inproviso rerum exitu induantur. Quod si neminem malum esse nisi stultum eundem non modo a sapientibus dicitur sed vulgo quoque semper est creditum, certe non fiet umquam stultus orator. Adde quod ne studio quidem operis pulcherrimi vacare mens nisi omnibus vitiis libera potest: primum quod in eodem pectore nullum est honestorum turpiumque consortium, et cogitare optima simul ac deterrima non magis est unius animi quam eiusdem hominis bonum esse ac malum; tum illa quoque ex causa, quod mentem tantae rei intentam vacare omnibus aliis, etiam culpa carentibus, curis oportet. Ita demum enim libera ac tota, nulla distringente atque alio ducente causa, spectabit id solum ad quod accingitur. 

Quod si agrorum nimia cura et sollicitior rei familiaris diligentia et venandi uoluptas et dati spectaculis dies multum studiis auferunt (huic enim rei perit tempus quodcumque alteri datur), quid putamus facturas cupiditatem avaritiam invidiam, quarum inpotentissimae cogitationes somnos etiam ipsos et illa per quietem visa perturbent? Nihil est enim tam occupatum, tam multiforme, tot ac tam uariis adfectibus concisum atque laceratum quam mala mens. Nam et cum insidiatur, spe curis labore distringitur, et, etiam cum sceleris compos fuit, sollicitudine, paenitentia, poenarum omnium expectatione torquetur. Quis inter haec litteris aut ulli bonae arti locus? Non hercule magis quam frugibus in terra sentibus ac rubis occupata. 

Age, non ad perferendos studiorum labores necessaria frugalitas? Quid ergo ex libidine ac luxuria spei? Non praecipue acuit ad cupiditatem litterarum amor laudis? Num igitur malis esse laudem curae putamus? Iam hoc quis non videt, maximam partem orationis in tractatu aequi bonique consistere? Dicetne de his secundum debitam rerum dignitatem malus atque iniquus? Denique, ut maximam partem quaestionis eximam, demus, id quod nullo modo fieri potest, idem ingenii studii doctrinae pessimo atque optimo viro: uter melior dicetur orator? Nimirum qui homo quoque melior. Non igitur umquam malus idem homo et perfectus orator. Non enim perfectum est quicquam quo melius est aliud. Sed, ne more Socraticorum nobismet ipsi responsum finxisse videamur, sit aliquis adeo contra veritatem opstinatus ut audeat dicere eodem ingenio studio doctrina praeditum nihilo deteriorem futurum oratorem malum virum quam bonum: conuincamus huius quoque amentiam. Nam hoc certe nemo dubitabit, omnem orationem id agere ut iudici quae proposita fuerint vera et honesta videantur. Vtrum igitur hoc facilius bonus vir persuadebit an malus? Bonus quidem et dicet saepius vera atque honesta. Sed etiam si quando aliquo ductus officio (quod accidere, ut mox docebimus, potest) falso haec adfirmare conabitur, maiore cum fide necesse est audiatur. At malis hominibus ex contemptu opinionis et ignorantia recti nonnumquam excidit ipsa simulatio: inde inmodeste proponunt, sine pudore adfirmant. Sequitur in iis quae certum est effici non posse deformis pertinacia et inritus labor; nam sicut in uita, ita in causis quoque spes improbas habent; frequenter autem accidit ut iis etiam vera dicentibus fides desit videaturque talis aduocatus malae causae argumentum.

