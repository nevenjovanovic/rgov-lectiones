%\section*{O autoru}

\section*{Institutio oratoria (95), prohoemium}

Oratorem autem instituimus illum perfectum, qui esse nisi vir bonus non potest, ideoque non dicendi modo eximiam in eo facultatem sed omnis animi virtutes exigimus. Neque enim hoc concesserim, rationem rectae honestaeque vitae, ut quidam putaverunt, ad philosophos relegandam, cum vir ille vere civilis et publicarum privatarumque rerum administrationi accommodatus, qui regere consiliis urbes, fundare legibus, emendare iudiciis possit, non alius sit profecto quam orator. 

Quare, tametsi me fateor usurum quibusdam quae philosophorum libris continentur, tamen ea iure vereque contenderim esse operis nostri proprieque ad artem oratoriam pertinere. An, si frequentissime de iustitia fortitudine temperantia ceterisque similibus disserendum est, adeo ut vix ulla possit causa reperiri in quam non aliqua ex his incidat quaestio, eaque omnia inventione atque elocutione sunt explicanda, dubitabitur, ubicumque vis ingenii et copia dicendi postulatur, ibi partes oratoris esse praecipuas? fueruntque haec, ut Cicero apertissime colligit, quemadmodum iuncta natura, sic officio quoque copulata, ut idem sapientes atque eloquentes haberentur. Scidit deinde se studium, atque inertia factum est ut artes esse plures viderentur. 

Nam ut primum lingua esse coepit in quaestu institutumque eloquentiae bonis male uti, curam morum qui diserti habebantur reliquerunt: ea vero destituta infirmioribus ingeniis velut praedae fuit. Inde quidam contempto bene dicendi labore ad formandos animos statuendasque vitae leges regressi partem quidem potiorem, si dividi posset, retinuerunt, nomen tamen sibi insolentissimum adrogaverunt, ut soli studiosi sapientiae vocarentur; quod neque summi imperatores neque in consiliis rerum maximarum ac totius administratione rei publicae clarissime versati sibi umquam vindicare sunt ausi: facere enim optima quam promittere maluerunt. 

Ac veterum quidem sapientiae professorum multos et honesta praecepisse et, ut praeceperint, etiam vixisse facile concesserim: nostris vero temporibus sub hoc nomine maxima in plerisque vitia latuerunt. Non enim virtute ac studiis ut haberentur philosophi laborabant, sed vultum et tristitiam et dissentientem a ceteris habitum pessimis moribus praetendebant. Haec autem quae velut propria philosophiae adseruntur, passim tractamus omnes. Quis enim non de iusto, aequo ac bono, modo non et vir pessimus, loquitur? Quis non etiam rusticorum aliqua de causis naturalibus quaerit? Nam verborum proprietas ac differentia omnibus qui sermonem curae habent debet esse communis. Sed ea et sciet optime et eloquetur orator: qui si fuisset aliquando perfectus non a philosophorum scholis virtutis praecepta peterentur. 

Nunc necesse est ad eos [aliquando] auctores recurrere, qui desertam, ut dixi, partem oratoriae artis, meliorem praesertim occupaverunt, et velut nostrum reposcere, non ut illorum nos utamur inventis, sed ut illos alienis usos esse doceamus. 

Sit igitur orator vir talis qualis vere sapiens appellari possit, nec moribus modo perfectus (nam id mea quidem opinione, quamquam sunt qui dissentiant, satis non est), sed etiam scientia et omni facultate dicendi; qualis fortasse nemo adhuc fuerit, sed non ideo minus nobis ad summa tendendum est: quod fecerunt plerique veterum, qui, etsi nondum quemquam sapientem repertum putabant, praecepta tamen sapientiae tradiderunt. Nam est certe aliquid consummata eloquentia neque ad eam pervenire natura humani ingenii prohibet. Quod si non contingat, altius tamen ibunt qui ad summa nitentur quam qui praesumpta desperatione quo velint evadendi protinus circa ima substiterint. 

Quo magis impetranda erit venia si ne minora quidem illa, verum operi quod instituimus necessaria, praeteribo. Nam liber primus ea quae sunt ante officium rhetoris continebit. Secundo prima apud rhetorem elementa et quae de ipsa rhetorices substantia quaeruntur tractabimus. Quinque deinceps inventioni (nam huic et dispositio subiungitur), quattuor elocutioni, in cuius partem memoria ac pronuntiatio veniunt, dabuntur. Vnus accedet in quo nobis orator ipse informandus est: ubi qui mores eius, quae in suscipiendis discendis agendis causis ratio, quod eloquentiae genus, quis agendi debeat esse finis, quae post finem studia, quantum nostra valebit infirmitas disseremus. 

His omnibus admiscebitur, ut quisque locus postulabit, docendi ratio quae non eorum modo scientia quibus solis quidam nomen artis dederunt studiosos instruat et, ut sic dixerim, ius ipsum rhetorices interpretetur, sed alere facundiam, vires augere eloquentiae possit. Nam plerumque nudae illae artes nimiae subtilitatis adfectatione frangunt atque concidunt quidquid est in oratione generosius, et omnem sucum ingenii bibunt et ossa detegunt, quae ut esse et adstringi nervis suis debent, sic corpore operienda sunt. Ideoque nos non particulam illam, sicuti plerique, sed quidquid utile ad instituendum oratorem putabamus in hos duodecim libros contulimus, breviter omnia demonstraturi: nam si quantum de quaque re dici potest persequamur, finis operis non reperietur.
