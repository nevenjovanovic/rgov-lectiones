%\section*{O autoru}

\section*{C.\ Plinii Secundi ep. 3, 13}

C.~Plinius Voconio Romano suo s.

Librum, quo nuper optimo principi consul gratias egi, misi exigenti tibi, missurus etsi non exegisses.

In hoc consideres velim ut pulchritudinem materiae ita difficultatem. In ceteris enim lectorem novitas ipsa intentum habet, in hac nota vulgata dicta sunt omnia; quo fit ut quasi otiosus securusque lector tantum elocutioni vacet, in qua satisfacere difficilius est cum sola aestimatur.

Atque utinam ordo saltem et transitus et figurae simul spectarentur! Nam invenire praeclare, enuntiare magnifice interdum etiam barbari solent, disponere apte, figurare varie nisi eruditis negatum est.

Nec vero affectanda sunt semper elata et excelsa. Nam ut in pictura lumen non alia res magis quam umbra commendat, ita orationem tam summittere quam attollere decet.

Sed quid ego haec doctissimo viro? Quin potius illud: adnota, quae putaveris corrigenda. Ita enim magis credam cetera tibi placere, si quaedam displicuisse cognovero. Vale.



\section*{C.\ Plinii Secundi consulis Panegyricus \\Nervae Traiano Augusto dictus (100–103 a. D)}

Tractantur nempe :

I. Quae ante Trajani imperium Romae administratum facta sunt, cap. V-XXIV. quo pertinet
\begin{enumerate}
\item Electio, quae ipsi multis nominibus gloriam afferebat, cap. V-X.
\item Pietas in Nervam, cui ex more quidem, ceterum optima mente divinos honores decrevit, cap. XI.
\item Virtutes bellicae, quarum gloria hostes terrebat, cap. XII-XVIII.
\item Reverentia legatorum, cap. XIX.
\item Ingressus in urbem, cap. XX-XXIV.
\end{enumerate}
II. Vita ejus post reditum.

I. Publica, cap. XXV-LXXXII.
\begin{enumerate}
\item  Instituta et decreta, cap. XXV-XLIII. Initia imperii commendavit congiario et donativo, quorum altero populum, altero milites locupletavit, cap. XXV-XXVIII. Annonam curavit non magis Romae, quam in provinciis, in Aegypto inprimis, cap. XXIX-XXXII. Populum oblectavit spectaculis, cap. XXXIII. et suppliciis delatorum, cap. XXXIV-XXXV. Lites circa aerarium et fiscum coercuit, cap. XXXVI. Vicesimae onus magis etiam minuit, quam Nerva, cap. XXXVII-XL. Majestatis crimen, quod et ipsum fiscum et aerarium locupletabat, penitus sustulit, cap. XLII. Testamenta secura esse jussit, cap. XLIII.
\item Mores et animus in vita publica. Amat et provehit bonos, cap. XLIV-XLVI. Amat bonas literas, earumque doctores. Facile omnium ordinum homines admittit, atque etiam remissionibus suis frequenter interesse jubet, cap. XLVII-XLIX. Justitiam colit, nemini suum eripiens, et liberalitatem, multa ex patrimonio suo donans, cap. L. In aedificando privatim parcus reperitur, sed idem in publicum magnificus, cap. LI. Parcus etiam est in honoribus recipiendis; adulantium voces spernit, ludicras artes a sui cultu removet, caducos titulos deprecatur, quia scit, ubi vera sit gloria, cap. LII-LV. Sed inprimis copiosus est in moribus amplificandis, quos Trajanus in secundo et tertio consulatu exhibuerat, cap. LVI-LXXX.
\item Remissiones, quibus negotia publica subinde interrumpit, quae nil nisi mutationem laboris habent, cap. LXXXI-LXXXIII.
\end{enumerate}
II. Privata. Uxorem et sororem ad suos mores conformavit, cap. LXXXIII-LXXXIV. Amicitiam sanctissime colit, cap. LXXXV-LXXXVII. Libertis justum honorem habet, cap. LXXXVIII.


\subsection*{Synopsis panegyrici}
{\noindent
In Exordio primum reddit rationem cur initio Deos precetur: quod haec instituta sit a majoribus consuetudo, quam nemini ait potius quam sibi observandum, cap. I. Quia consulem decet religio, quia laudat optimum Principem, qualis Deorum munus est, cap. II. Quia inprimis Trajanus a Diis datus, cujus adoptio ante Jovis pulvinar in Capitolio facta. Deinde ostendit hanc laudationem non tam ut Principi serviliter obsequatur a se susceptam, quam ut senatusconsulto pareat; utque boni principes, quae facere debeant, in laudatis Trajani virtutibus recognoscant. Quare abesse oportere adulationis suspicionem. Haeret in commendatione ejus humanitatis, cap. III. Huiusque exordium. 

Distributio videtur haec esse: I. Vita Trajani publica. II. Vita ejusdem privata laudatur, cap. IV. 

I. Pars. Universas Trajani laudes primum ponit sub uno aspectu, quas postea sigillatim persequatur. Adoptatur a Nerva, cap. V. Quam gratus erga eum fuerit, cap. X. Quam metuendus esset hostibus, cap. XII. Quam carus et venerabilis militibus suis, cap. XIII. Traducta in bello prima ejus pueritia, cap. XIV. Jam tum signa dedit victoriarum ab eo referendarum. Plinius quasi vaticinans triumphum describit, cap. XV. Trajanus disciplinam militarem emendat, cap. XVIII. Romam redit, cap. XX. Patris patriae appellationem accipit a senatu, cap. XXI. Describitur placidus ejus reditus, cap. XXII. Romam redux primum tributa remittit: congiaria, donativa, alimenta dat, cap. XXV. Ubertatem in Urbem invehit, cap. XXVIII. Inde Aegypti sterilitati succurrit, cap. XXX. Ludos dat populo, cap. XXXIII. Delatores punit, cap. XXXIV. Vicesimale tributum temperat, cap. XXXVII. Testamentis securitatem inducit, cap. XLIII. Bonis et maxime nobilitati favet, cap. XLIV. Censoris nomen recusat, cap. XLV. Literarum studia promovet: honorem dicendi magistris habet. Ejus comitas et humanitas, cap. XLVII. In publicis aedificiis ejus magnificentia, domi frugalitas, cap. L. Tanto major ceteris imperatoribus, quanto illis modestior, cap. LII. Secundum consulatum admittit sub Nerva, cap. LVI. Quo mortuo, jam sui juris factus, tertium recusat, cap. LVII. Exoratus tandem a republica eum suscipit anno sequenti, servatis tamen, quasi in privati hominis electione, comitiorum legibus, cap. LIX. In tertio consulatu nobiles vires promovet ad magistratus, maxime quos provinciae commendarent, cap. LXIX. Quam fuerit comis in candidatos, in cives, cap. LXXI. Quas acclamationes modestia meruerit, cap. LXXIV. Assiduus in foro, ibi omnes boni consulis partes agit, cap. LXXVI. Rogatur, ut quartum constilatum accipiat, cap. LXXVII. Quicquid supererat ab reip. rebus otii, non luxu aut desidia consumit, sed venatione ceterisque corporis exercitationibus impendit, cap. LXXXI. 

II. Pars. Hactenus qualis in imperio fuerit : nunc qualis in privata vita. Laus Plotinae Trajani conjugis, Marcianae sororis, quas exemplo suo instituit, cap. LXXXIV. Quanta esset in colendis amicitiis ejus constantia, cap. LXXXV. Quanta ipsius procuratorum integritas, cap. LXXXVII. Neque libertis plus quam par esset tribuit: ob quae omnia merito dicitur optimus a senatu, cap. LXXXVIII.

In fine compellat Nervam et Trajanum patrem; quibus de filii virtutibus gratulatur, cap. LXXXIX. Senatores; a quibus dicendi munus sibi impositum erat, cap. XC. Trajanum Imperatorem; cui et pro se et pro collega gratias agit, cap. XCI. Custodes imperii Deos, praecipue Capitolinum Jovem, quem orat pro aeternitate imperii et Principis, cap. XCIV. Iterum senatores, quibus obsequium et reverentiam pollicetur, cap. XCV. 
}



\subsection*{Paneg. IV–V}



Sed parendum est Senatusconsulto, quo ex utilitate publica placuit, ut Consulis voce, sub titulo gratiarum agendarum, boni principes, quae facerent, recognoscerent; mali, quae facere deberent. Id nunc eo magis solemne ac necessarium est, quod parens noster privatas gratiarum actiones cohibet et comprimit, intercessurus etiam publicis, si permitteret sibi vetare, quod Senatus iuberet. Utrumque, Caesar Auguste, moderate, et quod alibi tibi gratias agi non sinis, et quod hic sinis. Non enim a te ipso tibi honor iste, sed agentibus habetur. Cedis affectibus nostris, nec nobis munera tua praedicare, sed audire tibi necesse est. Saepe ego mecum, Patres Conscripti, tacitus agitavi, qualem quantumque esse oporteret, cuius ditione nutuque maria, terrae, pax, bella regerentur: quum interea fingenti formantique mihi principem, quem aequata diis immortalibus potestas deceret, nunquam voto saltem concipere succurrit similem huic, quem videmus. Enituit aliquis in bello, sed obsolevit in pace: alium toga, sed non et arma honestarunt: reverentiam ille terrore, alius amorem humanitate captavit: ille quaesitam domi gloriam in publico, hic in publico partam domi perdidit. Postremo adhuc nemo exstitit, cuius virtutes nullo vitiorum confinio laederentur. At Principi nostro quanta concordia, quantusque concentus omnium laudum omnisque gloriae contigit! Ut nihil severitati eius hilaritate, nihil gravitati simplicitate, nihil maiestati humanitate detrahitur! Iam firmitas, iam proceritas corporis, iam honor capitis, et dignitas oris, ad hoc aetatis indeflexa maturitas, nec sine quodam munere deum festinatis senectutis insignibus ad augendam maiestatem ornata caesaries, nonne longe lateque principem ostentant?

Talem esse oportuit, quem non bella civilia nec armis oppressa respublica, sed pax, et adoptio, et tandem exorata terris numina, dedissent. An fas erat, nihil differre inter imperatorem, quem homines, et quem dii fecissent? quorum quidem in te, Caesar Auguste, iudicium et favor, tunc statim, quum ad exercitum proficiscereris, et quidem inusitato indicio enituit. Nam ceteros principes aut largus cruor hostiarum, aut sinister volatus avium consulentibus nuntiavit: tibi ascendenti de more Capitolium, quamquam non id agentium civium clamor, ut iam principi, occurrit. Siquidem omnis turba, quae limen insederat, ad ingressum tuum foribus reclusis, illa quidem ut tunc arbitrabatur, deum, ceterum, ut docuit eventus, te consalutavit imperatorem. Nec aliter a cunctis omne acceptum est. Nam ipse intelligere nolebas: recusabas enim imperare, recusabas; quod bene erat imperaturi. Igitur cogendus fuisti. Cogi porro non poteras, nisi periculo patriae, et nutatione reipublicae. Obstinatum enim tibi non suscipere imperium, nisi servandum fuisset. Quare ego illum ipsum furorem motumque castrensem reor exstitisse, quia magna vi magnoque terrore modestia tua vincenda erat. Ac sicut maris coelique temperiem turbines tempestatesque commendant; ita ad augendam pacis tuae gratiam illum tumultum praecessisse crediderim. Habet has vices conditio mortalium, ut adversa ex secundis, ex adversis secunda nascantur. Occultat utrorumque semina deus, et plerumque bonorum malorumque caussae sub diversa specie latent.

\subsection*{Paneg. XXXIII–XXXIV}

Satisfactum qua civium, qua sociorum utilitatibus. Visum est spectaculum inde non enerve, nec fluxum, nec quod animos virorum molliret et frangeret, sed quod ad pulchra vulnera contemptumque mortis accenderet; quum in servorum etiam noxiorumque corporibus amor laudis et cupido victoriae cerneretur. Quam deinde in edendo liberalitatem, quam iustitiam exhibuit, omni affectione aut intactus, aut maior! Impetratum est, quod postulabatur; oblatum, quod non postulabatur. Institit ultro, et, ut concupisceremus, admonuit; ac sic quoque plura inopinata, plura subita. Iam quam libera spectantium studia, quam securus favor! Nemini impietas, ut solebat, obiecta, quod odisset gladiatorem: nemo e spectatore spectaculum factus, miseras voluptates unco et ignibus expiavit. Demens ille, verique honoris ignarus, qui crimina maiestatis in arena colligebat, ac se despici et contemni, nisi etiam gladiatores eius veneraremur, sibi maledici in illis, suam divinitatem, suum numen violari, interpretabatur; quum se idem quod deos, idem gladiatores quod se putabat.

At tu, Caesar, quam pulchrum spectaculum pro illo nobis exsecrabili reddidisti! Vidimus delatorum iudicium, quasi grassatorum quasi latronum. Non solitudinem illi, non iter, sed templum, sed forum insederant: nulla iam testamenta secura, nullus status certus: non orbitas, non liberi proderant. Auxerat hoc malum principum avaritia. Advertisti oculos, atque ut ante castris, ita postea pacem foro reddidisti: exscidisti intestinum malum: et provida severitate cavisti, ne fundata legibus civitas eversa legibus videretur. 

Licet ergo cum fortuna, tum liberalitas tua visenda nobis praebuerit, ut praebuit, nunc ingentia robora virorum, et pares animos, nunc immanitatem ferarum, nunc mansuetudinem incognitam; nunc secretas illas et arcanas, ac sub te primum communes opes: nihil tamen gratius, nihil saeculo dignius, quam quod contigit desuper intueri delatorum supina ora, retortasque cervices. Agnoscebamus et fruebamur, quum velut piaculares publicae solicitudinis victimae, supra sanguinem noxiorum, ad lenta supplicia gravioresque poenas ducerentur. Congesti sunt in navigia raptim conquisita, ac tempestatibus dediti. Abirent, fugerentque vastatas delationibus terras; ac, si quem fluctus ac procellae scopulis reservassent, hic nuda saxa et inhospitale litus incoleret; ageret duram et anxiam vitam relictaque post tergum totius generis humani securitate maereret.

