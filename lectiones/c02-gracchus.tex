%\section*{O autoru}
\begin{quotation}

\noindent Novem annis minor natu quam Tiberius frater, longe et praestantiore ingenio et uberiore facundia eum superavit. Postquam prima stipendia a. 616/138 meruit, et triumvir a.\ d.\ i.\ a.\ ex lege Sempronia factus est (a.~621/133), quaestor creatus Sardiniam provinciam a.~628/126 sortitus est: Romam autem post biennium sua sponte reversus, tribunus plebis a.~631/123 creatus est et iterum insequenti anno: quo in magistratu cum multas promulgasset leges plebi gratissimas, eius adversarii enixe restiterunt, quominus rursus tribunus crearetur: repulsam enim tertio tribunatu petendo tulit. Cum autem hac de causa valde eius potentia imminuta esset, postquam coloniam Carthaginem lege Rubria deduxit a.~633/121, L.~Opimio inimicissimo eius consule, seditione orta, interfectus est eiusque corpus, capite praeciso, quibusdam auctoribus\footnote{Plut. C. Gr. 17, 6; Vell. II 6, 7.} in Tiberim proiectum, aliis\footnote{Oros. V 12, 9.} vero Misenum ad Corneliam matrem latum est.\footnote{Veterum scriptorum testimonia ad Gai vitam pertinentia collecta videre potes apud Nataliam Haepke, pp. 25-28.}

De eius oratoria laude concors est veterum scriptorum iudicium, maximique ponderis Ciceronis testimonium, qui, quamvis oratoribus popularis partis minime faveret, haud dubitavit affirmare\footnote{Cic. Brut. 33, 125.} neminem pleniorem aut uberiorem eo ad dicendum fuisse.

Fuit Gaius, quod ad inventionem pertinet, frequens sententiis, quod ad elocutionem, grandis verbis et splendidus, actione autem vehemens et concitatus atque acrimoniae impetusque plenus et ad animos movendos maxime idoneus. Quanta vero eius fuerit cura in voce modulanda, quae maximam ad actionis usum atque laudem partem obtinet,\footnote{Cic. de orat. III 60, 224.} ut canora ea atque ad rem apta uteretur, id docet, quod Cicero de Licinio fistulatore narrat, quem Gaius, cum contionaretur, adhibuerit, qui fistulam inflando aut eum remissum excitaret, aut a contentione revocaret.\footnote{Cic. de orat. III 60, 225, ex quo cetera pendent hac de re testimonia: Quint. I 10, 27; Val. Max. VIII 10, 1; Plut. Ti. Gr. 2, 6; Gell. I 11, 10; Amm. Marc. XXX 4, 19.}

Quando orationem `de legibus promulgatis' – hunc enim titulum Gellius\footnote{Gell. IX 14, 16; X 3, 2.} et Scholia Bobiensia\footnote{Schol. Bob. pro Sulla p. 81, 19 St.} tradiderunt, Festus\footnote{Fest. p. 218, 30 L.} autem `de legibus a se promulgatis' – Gaius habuerit, virorum doctorum sententiae discrepant: alii\footnote{Meyer p. 234, qui vero parum diligenter ait: "antequam leges suas promulgasset, orationem in concione habuit, in qua cum orationem post leges promulgatas habitam esse ex eius indice appareat; Lange III, p. 34, qui a Gaio orationes, de suis legibus habitas, sub hoc indice `orationes de legibus promulgatis' esse collectas censuit: verum una modo huiuscemodi oratio memeoratur; Cima, p. 123, qui orationem a Gaio habitam esse censuit, ut leges suas populo commendaret, ``mostrando quale fosse in genere lo spirito di esse, ma riservandosi di sostenerle particolarmente man mano che fossero venute in discussione e sottoposte all'approvazione delle tribù''; item fere Mommsen, Röm. Gesch. 99, p. 104.} enim ad initium prioris tribunatus eam esse referendam censuerunt; alii\footnote{Nitzsch, Die Gracchen und ihre nächsten Vorgänger, Berlin 1847, p. 398; Neumann, I, p. 251, Kornemann, p. 51, Fraccaro, Studi storici VI, p. 105.} alteri tribunatui vindicarunt. Cum autem constet orationum indices non ad ipsos, qui eas dixerint pertinere, verum ad grammaticos qui earum fragmenta tradiderint,\footnote{Fraccaro, Catoniana, in ``Studi storici'' III (1910) p. 265.} hanc orationem habitam esse existimemus oportet de rogationibus quae non perlatae sunt: cum autem post initum alterum tribunatum leges a Gaio promulgatas esse de coloniis Tarentum et Capuam\footnote{Haec Plutarchus, C. Gr. 8, 3, et de vir. illustr. 65, 3: colonos etiam Capuam et Tarentum mittendos censuit: alii vero Tarentum et Scolacium tradiderunt.} deducendis deque adsciscendis in civitatem Latinis, compertum sit, et cum mox senatus consultum esset factum, ne quis in urbe moraretur, nisi qui ius suffragii haberet, et Livio Druso,\footnote{App. b. c. I 23: cfr. Plut. C. Gr. 8, 5-6.} tribuno plebis, persuasisset senatus, ut Gracchi rogationibus intercederet, utque legem de duodecim coloniis deducendis ferret, ad plebis animos conciliandos, plebem laetatam Gracchi leges contemnere coepisse, veri simile videtur ad eas leges promulgatas neque tamen perlatas orationem esse referendam; eo magis quod de libidine atque intemperantia magistratuum populi Romani fusius tunc eum locutum esse fragmenta (45-46) doceant neque eundem impetum eandemque fiduciam atque antea tunc ei fuisse sed praesentes difficultates instantesque calamitates animo eum sensisse (fr. 44).\footnote{Cfr. Fraccaro, ``Studi storici'' VI, p. 106, 108.}

\noindent Henrica Malcovati, \textit{Oratorum Romanorum fragmenta} II, Torino, 1930. Prolegomena, 35-51.
\end{quotation}

\section*{Aulus Gellius (ca. 130 – post 170), Noctes Atticae 10.3.3}

\textit{Locorum quorundam inlustrium conlatio contentioque facta ex orationibus C.~Gracchi et M.~Ciceronis et M.~Catonis.}

\medskip

Fortis ac uehemens orator existimatur esse C.~Gracchus. Nemo id negat. Sed quod nonnullis uidetur seuerior, acrior ampliorque esse M. Tullio, ferri id qui potest? Legebamus adeo nuper orationem Gracchi de legibus promulgatis, in qua M.~Marium et quosdam ex municipiis Italicis honestos uiros uirgis per iniuriam caesos a magistratibus populi Romani, quanta maxima inuidia potest, conqueritur.

Verba haec sunt, quae super ea re fecit:
\begin{quotation}
\noindent Nuper Teanum Sidicinum consul uenit. Vxor eius dixit se in balneis uirilibus lauari uelle. Quaestori Sidicino M.\ Mario datum est negotium, uti balneis exigerentur, qui lauabantur. Vxor renuntiat uiro parum cito sibi balneas traditas esse et parum lautas fuisse. Idcirco palus destitutus est in foro, eoque adductus suae ciuitatis nobilissimus homo M.\ Marius. Vestimenta detracta sunt, uirgis caesus est. Caleni, ubi id audierunt, edixerunt, ne quis in balneis lauisse uellet, cum magistratus Romanus ibi esset. Ferentini ob eandem causam praetor noster quaestores arripi iussit: alter se de muro deiecit, alter prensus et uirgis caesus est.
\end{quotation}


In tam atroci re ac tam misera atque maesta iniuriae publicae contestatione ecquid est, quod aut ampliter insigniterque aut lacrimose atque miseranter aut multa copiosaque inuidia grauique et penetrabili querimonia dixerit? breuitas sane et uenustas et mundities orationis est, qualis haberi ferme in comoediarum festiuitatibus solet.

Item Gracchus alio in loco ita dicit:
\begin{quotation}
\noindent Quanta libido quantaque intemperantia sit hominum adulescentium, unum exemplum uobis ostendam. His annis paucis ex Asia missus est, qui per id tempus magistratum non ceperat, homo adulescens pro legato. Is in lectica ferebatur. Ei obuiam bubulcus de plebe Venusina aduenit et per iocum, cum ignoraret, qui ferretur, rogauit, num mortuum ferrent. Vbi id audiuit, lecticam iussit deponi, struppis, quibus lectica deligata erat, usque adeo uerberari iussit, dum animam efflauit.
\end{quotation}

Haec quidem oratio super tam uiolento atque crudeli facinore nihil profecto abest a cotidianis sermonibus.
