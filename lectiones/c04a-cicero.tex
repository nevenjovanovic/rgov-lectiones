%\section*{O autoru}

\section*{M. Tullii Ciceronis ad Q.\ fratrem \\dialogi tres de oratore (55 a. Chr. n)}

\subsection*{Argumentum}

In his libris, in quibus ipse sibi mirifice placuit Cicero, de eloquentia ita praecipit, ut, quod alios docet, ipsum plane assecutum esse appareat. Explicat autem suas de dicendi arte sententias sub Lucii Licinii Crassi et Marci Antonii personis, quod ad hos homines eloquentiae principatum omnes, superiori seculo, detulerant. 

Et in primo quidem libro Crassus et Antonius de universa oratoris perfecti vi disputant, quorum alter ad Ciceronis mentem de ea re disserit, ita, ut oratori omnium scientiarum et artium cognitionem tribuat; alter vero Antonius fratris Quinti sententia eloquentiam ab elegantia doctrinae segregat, et in quodam ingenii et exercitationis genere ponit. Primus disputat Crassus, deinde ejus argumentis quae responderi possunt, affert Antonius. 

In secundo libro disserendi partes imprimis ad Antonium deferuntur, qui de inventione praecipit, et de collocatione rerum locorumque, qua inprimis valuisse credebatur, cum contra non elegantissimo sermone uteretur atque diligenter loquendi laude careret. Ad inventionem rerum autem quoniam pertinet etiam de jocis et facetiis locus, et Caesar credebatur omnium facetissimus et in jocando lepidissimus esse, ei de his disputandi partes traduntur.

Redit ad Crassum in libro tertio disserendi munus, induciturque de ornamentis orationis et tota elocutione praecipiens, in qua ceteris omnibus palmam praeripere putabatur. Docet autem dicendum esse latine, plane, ornate, et apte, in quorum tertio inprimis multus est. Ornatum autem existere ostendit cum e verborum delectu, tum e tropis, figuris, et numero. Adduntur tandem quaedam de actione. Cap. 14. seq. digreditur ad laudes eloquentiae, eique adjungendam esse docet philosophiam, non Epicuream quidem aut Stoicam, sed Academicam et Peripateticam. In exordio Cicero funestam mortem eorum deplorat, qui in his libris disputantes inducuntur.

\subsection*{De oratore 1, 7 (24-29)}

Cum igitur vehementius inveheretur in causam principum consul Philippus Drusique tribunatus pro senatus auctoritate susceptus infringi iam debilitarique videretur, dici mihi memini ludorum Romanorum diebus L.~Crassum quasi conligendi sui causa se in Tusculanum contulisse; venisse eodem, socer eius qui fuerat, Q.~Mucius dicebatur et M.~Antonius, homo et consiliorum in re publica socius et summa cum Crasso familiaritate coniunctus. Exierant autem cum ipso Crasso adulescentes et Drusi maxime familiares et in quibus magnam tum spem maiores natu dignitatis suae conlocarent, C.~Cotta, qui [tum] tribunatum plebis petebat, et P.~Sulpicius, qui deinceps eum magistratum petiturus putabatur.  Hi primo die de temporibus deque universa re publica, quam ob causam venerant, multum inter se usque ad extremum tempus diei conlocuti sunt; quo quidem sermone multa divinitus a tribus illis consularibus Cotta deplorata et commemorata narrabat, ut nihil incidisset postea civitati mali, quod non impendere illi tanto ante vidissent. Eo autem omni sermone confecto, tantam in Crasso humanitatem fuisse, ut, cum lauti accubuissent, tolleretur omnis illa superioris tristitia sermonis eaque esset in homine iucunditas et tantus in loquendo lepos, ut dies inter eos curiae fuisse videretur, convivium Tusculani; postero autem die, cum illi maiores natu satis quiessent et in ambulationem ventum esset, [dicebat] tum Scaevolam duobus spatiis tribusve factis dixisse ``cur non imitamur, Crasse, Socratem illum, qui est in Phaedro Platonis? Nam me haec tua platanus admonuit, quae non minus ad opacandum hunc locum patulis est diffusa ramis, quam illa, cuius umbram secutus est Socrates, quae mihi videtur non tam ipsa acula, quae describitur, quam Platonis oratione crevisse, et quod ille durissimis pedibus fecit, ut se abiceret in herba atque ita [illa], quae philosophi divinitus ferunt esse dicta, loqueretur, id meis pedibus certe concedi est aequius.'' Tum Crassum ``immo vero commodius etiam''; pulvinosque poposcisse et omnis in eis sedibus, quae erant sub platano, consedisse dicebat.

\subsection*{De oratore 1, 24-26 (110-122)}

Tum Antonius vehementer se adsentiri Crasso dixit, quod neque ita amplecteretur artem, ut ei solerent, qui omnem vim dicendi in arte ponerent, neque rursus eam totam, sicut plerique philosophi facerent, repudiaret. ``Sed existimo'' inquit ``gratum te his, Crasse, facturum, si ista exposueris quae putas ad dicendum plus quam ipsam artem posse prodesse.'' ``Dicam equidem, quoniam institui, petamque a vobis,'' inquit ``ne has meas ineptias efferatis; quamquam moderabor ipse, ne ut quidam magister atque artifex, sed quasi unus ex togatorum numero atque ex forensi usu homo mediocris neque omnino rudis videar non ipse a me aliquid promisisse, sed fortuito in sermonem vestrum incidisse. Equidem cum peterem magistratum, solebam in prensando dimittere a me Scaevolam, cum ita ei dicerem, me velle esse ineptum; id erat petere blandius, quod, nisi inepte fieret, bene non posset fieri – hunc autem esse unum hominem ex omnibus, quo praesente ego ineptum esse me minime vellem – quem quidem nunc mearum ineptiarum testem et spectatorem fortuna constituit: nam quid est ineptius quam de dicendo dicere, cum ipsum dicere numquam sit non ineptum, nisi cum est necessarium?'' ``Perge vero,'' inquit ``Crasse,'' Mucius; ``istam enim culpam, quam vereris, ego praestabo.''

``Sic igitur'' inquit ``sentio,'' Crassus ``naturam primum atque ingenium ad dicendum vim adferre maximam; neque vero istis, de quibus paulo ante dixit Antonius, scriptoribus artis rationem dicendi et viam, sed naturam defuisse; nam et animi atque ingeni celeres quidam motus esse debent, qui et ad excogitandum acuti et ad explicandum ornandumque sint uberes et ad memoriam firmi atque diuturni; et si quis est qui haec putet arte accipi posse, – quod falsum est; praeclare enim res se habeat, si haec accendi aut commoveri arte possint; inseri quidem et donari ab arte non possunt; omnia sunt enim illa dona naturae – quid de illis dicam, quae certe cum ipso homine nascuntur, linguae solutio, vocis sonus, latera, vires, conformatio quaedam et figura totius oris et corporis? Neque enim haec ita dico, ut ars aliquos limare non possit – neque enim ignoro, et quae bona sint, fieri meliora posse doctrina, et, quae non optima, aliquo modo acui tamen et corrigi posse – sed sunt quidam aut ita lingua haesitantes aut ita voce absoni aut ita vultu motuque corporis vasti atque agrestes, ut, etiam si ingeniis atque arte valeant, tamen in oratorum numerum venire non possint; sunt autem quidam ita in eisdem rebus habiles, ita naturae muneribus ornati, ut non nati, sed ab aliquo deo ficti esse videantur. 

Magnum quoddam est onus atque munus suscipere atque profiteri se esse, omnibus silentibus, unum maximis de rebus magno in conventu hominum audiendum; adest enim fere nemo, quin acutius atque acrius vitia in dicente quam recta videat; ita quicquid est, in quo offenditur, id etiam illa, quae laudanda sunt, obruit.  Neque haec in eam sententiam disputo, ut homines adulescentis, si quid naturale forte non habeant, omnino a dicendi studio deterream: quis enim non videt C.~Coelio, aequali meo, magno honori fuisse, homini novo, illam ipsam, quamcumque adsequi potuerat, in dicendo mediocritatem? Quis vestrum aequalem, Q.~Varium, vastum hominem atque foedum, non intellegit illa ipsa facultate, quamcumque habuit, magnam esse in civitate gratiam consecutum?

Sed quia de oratore quaerimus, fingendus est nobis oratione nostra detractis omnibus vitiis orator atque omni laude cumulatus. Neque enim, si multitudo litium, si varietas causarum, si haec turba et barbaria forensis dat locum vel vitiosissimis oratoribus, idcirco nos hoc, quod quaerimus, omittemus. Itaque in eis artibus, in quibus non utilitas quaeritur necessaria, sed animi libera quaedam oblectatio, quam diligenter et quam prope fastidiose iudicamus! Nullae enim lites neque controversiae sunt, quae cogant homines sicut in foro non bonos oratores, item in theatro actores malos perpeti. Est igitur oratori diligenter providendum, non uti eis satis faciat, quibus necesse est, sed ut eis admirabilis esse videatur, quibus libere liceat iudicare; ac, si quaeritis, plane quid sentiam enuntiabo apud homines familiarissimos, quod adhuc semper tacui et tacendum putavi: mihi etiam qui optime dicunt quique id facillime atque ornatissime facere possunt, tamen, nisi timide ad dicendum accedunt et in ordienda oratione perturbantur, paene impudentes videntur, – tametsi id accidere non potest; ut enim quisque optime dicit, ita maxime dicendi difficultatem variosque eventus orationis exspectationemque hominum pertimescit; – qui vero nihil potest dignum re, dignum nomine oratoris, dignum hominum auribus efficere atque edere, is mihi, etiam si commovetur in dicendo, tamen impudens videtur; non enim pudendo, sed non faciendo id, quod non decet, impudentiae nomen effugere debemus;  quem vero non pudet, – id quod in plerisque video – hunc ego non reprehensione solum, sed etiam poena dignum puto. Equidem et in vobis animum advertere soleo et in me ipso saepissime experior, ut et exalbescam in principiis dicendi et tota mente atque artubus omnibus contremiscam; adulescentulus vero sic initio accusationis exanimatus sum, ut hoc summum beneficium Q.~Maximo debuerim, quod continuo consilium dimiserit, simul ac me fractum ac debilitatum metu viderit.'' Hic omnes adsensi significare inter sese et conloqui coeperunt; fuit enim mirificus quidam in Crasso pudor, qui tamen non modo non obesset eius orationi, sed etiam probitatis commendatione prodesset.

\section*{De claris oratoribus liber, qui dicitur Brutus (46 a. Chr. n)}

\subsection*{Summarium}

In praefatione libri Cicero primum Q.~Hortensii mortem deplorat, additis consolationis causis, c. 1. 2. deinde occasionem huius dialogi exponit. Nimirum ad eum M.~Brutus cum T.~Pomponio Attico venerant, eumque rogaverant, ut, quem nuper in Tusculano, solo audiente Attico, inchoasset sermonem de claris oratoribus, nunc utrique plenum exponeret. c. 3-5. 

Sequitur ipse dialogus, in quo Cicero, postquam, unde ductus esset sermo cum Pomponio in Tusculano habitus, exposuerat c. 6., primum breviter Graecos oratores artisque rhetoricae scriptores recenset, c. 7-13; deinde ad Romanos accedit, in quibus primum antiquiores, de quorum eloquentia nihil certi habebat dicere, enumerat, ut L.~Brutum, M.~Valerium Maximum, L.~Valerium Potitum, Ap.~Claudium, C.~Fabricium, Ti.~Coruncanium , M'.~Curium, M. Popillium, C.~Flaminium, Q.~Fabium Maximum Verrucosum, Q.~Metellum. c. 14.


Sequuntur ii, de quibus certior eloquentiae memoria constabat; in hisque primus M.~Cornelius Cethegus c. 15. M.~Cato Censorius, cuius orationes et oratoriae laudes cum Lysia comparantur, et cur hic felicior sit ab omni laude, exponitur. c. 16-19. Post Catonem nominantur quidam, qui cum eo grandiores natu vixerunt, C.~Flaminius, C.~Varro, Q.~Maximus, Q.~Metellus, P.~Lentulus, P.~Crassus, P.~Scipio Africanus, eiusdem filius, Sex.~Aelius; deinde minores aetate C.~Sulpicius Gallus, Ti.~Gracchus, P.~F., P.~Scipio Nasica Corculum, Q.\ et M.\ Nobiliores, T.~Annius Luscus, L.~Paulus Macedonicus, Africani Iunioris pater; c. 20. Tum A.~Albinus, Ser.~Fabius Pictor, Q.~Fabius Labeo, Q.~Metellus, L.~Cotta, C.~Laelius, P.~Africanus Minor, Ser.~Galba. c. 21. Inprimis de horum trium, Laelii, Africani et Galbae laudibus dicitur, c. 22-24.

Post illos nominantur L.\ et Sp.~Mummii fratres, Sp.~Albinus, L.\ et C.~Aurelii Orestae, P.~Popillius, eiusque filius, Caius. Porro C.~Tuditanus, M.~Octavius, praecipue autem M.~Aemilius Lepidus Porcina. Deinde Q.~Pompeius, L.~Cassius, M.~Antius Briso, duo Caepiones, Cn.\ et Q.\ c.\ 25.

Sequuntur P.~Crassus, valde probatus orator, eiusque aetati iuncti duo C.~Fannii, C.\ et M.\ filii; Q.~Scaevola augur, iuris civilis intelligentia, non oratoria laude clarus; L.~Coelius Antipater, c.\ 26. 

Praecipue autem eminent Tib.~Gracchus Caiusque Carbo. c. 27. Minorem vero eloquentiae laudem adepti D.~Brutus, M.\ F., Q.~Maximus, Pauli nepos, P.~Scipio Nasica Serapio, P.~Lentulus, L.~Furius Philus; P.~Scaevola, M'.~Manilius, M.~Fulvius Flaccus, C.~Cato, P.~Decius, M.~Drusus C.~F.\ eiusque frater C.~Drusus; M.~Pennus et T.~Flamininus. c. 98. 

His adiunguntur M.~Scaurus c. 29. P.~Rutilius c. 30. Q.~Aelius Tubero, omninoque Stoici oratores c. 31. C.~Curio, illustris orator, c. 32. et praestantissimus ingenio, studio, doctrina C.~Gracchus. c. 33. Huic successit aetati C.~Galba, P.~Scipio, L.~Bestia, C.~Licinius Nerva, C.~Fimbria, C.~Sextius Calvinus, M.~Brutus, accusator ille vehemens et molestus, L.~Caesulenus, T.~Albucius. c. 34.

Porro Q.~Catulus, pater et filius, Q.~Metellus Numidicus, M.~Silanus, M.~Aurelius Scaurus, A.~Albinus, Q.~Caepio, C.\ et L.~Memmii, Sp.~Thorius, M.~Marcellus, P.~Lentulus, L.~Cotta. c. 35.

Sequuntur summi oratores L.~Licinius Crassus, et M.~Antonius, de quibus diligenter agitur c. 36-44. interiecto Q.~Scaevola, qui cum Ser.~Sulpicio Rufo, Ciceronis aequali, comparatur. c. 41.

Tum recensentur Cn.~Domitius, C.~Coelius, M.~Herennius, C.~Clodius, C.~Titius, L.~Afranius, Q.~Varro, M.~Gratidius. c. 45.

His adiunguntur nonnulli ex sociis et Latinis oratoribus, Q.~Vettius Vettianus, Q.\ et D.~Valerii Sorani, C.~Rusticellus Bononiensis, T.~Betucius Barrus Asculanus, et prior aetate L.~Papirius Fregellanus. c. 46.

Hos excipiunt e Romanis L.~Philippus, orator Crasso et Antonio, sed longo intervallo, proximus, et eius aetati fere coniuncti, L.~Gellius, D.~Brutus, L.~Scipio, Cn.~Pompeius, Sexti F., M.~Brutus, C.~Bellienus, Cn.~Octavius, c. 47. Tum C.~Iulius L.~F.\ Caesar, P.~Cethegus, Q.~Lucretius Vispillo, Ofella, T.~Annius Velina, T.~Iuvencius, P.~Orbius, T.~Aufidius eiusque frater M.~Virgilius, P.~Magius, Q.~Sertorius, C.~Gorgonius, T.~Iunius. c. 48.

Sequuntur ii, cum quibus Cicero magis iam vixit et viguit. Inter quos facile primas tulisse ait, cum suo tum omnium iudicio C.~Cottam et P.~Sulpicium, c. 49. idque eum in hanc disputationem ducit, ut, num recte populus de oratoribus iudicet, disquirat. c. 49. — 54. Tum laudes Cottae et Sulpicii fusius exponit c. 55. 56.; agitque porro de Pomponio c. 57.; de Curione c. 58. — 60., ubi simul de domestica institutione ad linguae puritatem multum valente disputat; porro de C.~Carbone; Q.~Vario, L.~Fufio, compluribus aliis vel e numero oratorum exclusis, vel brevissime memoratis. c. 61. 62.

Sequitur P.~Antistius c. 63. L.~Sisenna c. 64. et Hortensio, de quo postea pluribus verbis agit, memorato, agitur de M.~Crasso, C.~Fimbria, Cn.~et P.~Lentulis c. 66. de M.~Pisone, P.~Murena, C.~Censorino, L.~Turio , C.~Macro c.~67. de C.~Pisone, L.~Torquato, Cn.~Pompeio M., D.~Silano, Q.~Pompeio, P.~Autronio, L.~Octavio, C.~Staleno c. 68.; porro de C. et L.~Caepasiis fratribus, de Cosconio et Arrio c. 69. His adiunguntur L.~Torquatus, M.~Messala, Celer et Nepos Metelli, Cn.~Lentulus Marcellimus et C.~Memmius. c. 70. Tum rogatus a Bruto Cicero, ut de Caesare et M.~Marcello, utroque vivo, iudicium suum exponat, primum Marcellum ipsum laudat, Caesaris autem laudes fere ad Atticum remittit, ita tamen, ut iis ipse non minus ! quam Brutus assentiatur. c. 71-75. Post illos recensentur C.~Sicinius, C.~Visellius Varro, L.~Torquatus, C.~Triarius c.~76. M.~Bibulus, Ap.~Claudius, L.~Domitius, P.~et L.~Lentuli, T.~Postumius c. 77. P.~Cominius, T.~Attius, C.~Piso c. 78. Sequuntur M.~Coelius c.\ 79. M.~Calidius c. 80. C.~Curio, P.~Crassus, C.~Licinius Calvus c.\ 81. 82. a cuius exilitate sumit occasionem, de Attico dicendi genere disputandi c. 83. qua degressione laudata Atticus Ciceronem pervellit, quod nimiis laudibus multos Romanos extulerit oratores c. 83—87. Sequuntur iam Q.~Hortensii laudes c. 88., quibuscum coniungit Cicero de suis studiis laboribusque forensibus narrationem, c. 88-94. Tum quaestioni, cur Hortensius magis adolescens quam provecta aetate orator floruerit, respondetur. c.\ 95. 96. Epilogus hortatur Brutum, ut, quamquam iniqua nunc sint reipublicae tempora et oratorum studiis, tamen eloquentiae laudem tueri, seque ex turba patronorum eripere velit. c.\ 97.

\subsection*{Brutus 9-13}

Itaque ei mihi videntur fortunate beateque vixisse cum in ceteris civitatibus tum maxume in nostra, quibus cum auctoritate rerumque gestarum gloria tum etiam sapientiae laude perfrui licuit. quorum memoria et recordatio in maxumis nostris gravissimisque curis iucunda sane fuit, cum in eam nuper ex sermone quodam incidissemus.

Nam cum inambularem in xysto et essem otiosus domi, M.\ ad me Brutus, ut consueverat, cum T.~Pomponio venerat, homines cum inter se coniuncti tum mihi ita cari itaque iucundi, ut eorum aspectu omnis quae me angebat de re publica cura consederit. quos postquam salutavi: ``Quid vos'', inquam, ``Brute et Attice? numquid tandem novi?'' ``Nihil sane'', inquit Brutus, ``quod quidem aut tu audire velis aut ego pro certo dicere audeam.''

Tum Atticus: ``eo'', inquit, ``ad te animo venimus, ut de re publica esset silentium et aliquid audiremus potius ex te, quam te adficeremus ulla molestia.'' ``Vos vero'', inquam, ``Attice, et praesentem me cura levatis et absenti magna solacia dedistis. nam vestris primum litteris recreatus me ad pristina studia revocavi.'' Tum ille: ``legi'', inquit, ``perlubenter epistulam, quam ad te Brutus misit ex Asia, qua mihi visus est et monere te prudenter et consolari amicissume.''

``Recte'', inquam, ``est visus: nam me istis scito litteris ex diuturna perturbatione totius valetudinis tamquam ad aspiciendam lucem esse revocatum. atque ut post Cannensem illam calamitatem primum Marcelli ad Nolam proelio populus se Romanus erexit posteaque prosperae res deinceps multae consecutae sunt, sic post rerum nostrarum et communium gravissumos casus nihil ante epistulam Bruti mihi accidit, quod vellem aut quod aliqua ex parte sollicitudines adlevaret meas.''

Tum Brutus: ``volui id quidem efficere certe et capio magnum fructum, si quidem quod volui tanta in re consecutus sum. sed scire cupio, quae te Attici litterae delectaverint.'' ``Istae vero'', inquam, ``Brute, non modo delectationem mihi, sed etiam, ut spero, salutem adtulerunt.'' ``Salutem?'' inquit ille. ``quodnam tandem genus istuc tam praeclarum litterarum fuit?'' ``An mihi potuit'', inquam, ``esse aut gratior ulla salutatio aut ad hoc tempus aptior quam illius libri, quo me hic adfatus quasi iacentem excitavit?''

\subsection*{Brutus 21-26}

``Scio'', inquit, ``ab isto initio tractum esse sermonem teque Bruti dolentem vicem quasi deflevisse iudiciorum vastitatem et fori.''

``Feci'', inquam, ``istuc quidem et saepe facio. nam mihi, Brute, in te intuenti crebro in mentem venit vereri, ecquodnam curriculum aliquando sit habitura tua et natura admirabilis et exquisita doctrina et singularis industria. cum enim in maxumis causis versatus esses et cum tibi aetas nostra iam cederet fascisque submitteret, subito in civitate cum alia ceciderunt tum etiam ea ipsa, de qua disputare ordimur, eloquentia obmutuit.''

Tum ille: ``ceterarum rerum causa'', inquit, ``istuc et doleo et dolendum puto; dicendi autem me non tam fructus et gloria quam studium ipsum exercitatioque delectat: quod mihi nulla res eripiet te praesertim tam studiosum et * * * . dicere enim bene nemo potest nisi qui prudenter intellegit; quare qui eloquentiae verae dat operam, dat prudentiae, qua ne maxumis quidem in bellis aequo animo carere quisquam potest.''

``Praeclare'', inquam, ``Brute, dicis eoque magis ista dicendi laude delector, quod cetera, quae sunt quondam habita in civitate pulcherrima, nemo est tam humilis qui se non aut posse adipisci aut adeptum putet; eloquentem neminem video factum esse victoria. sed quo facilius sermo explicetur, sedentes, si videtur, agamus.'' Cum idem placuisset illis, tum in pratulo propter Platonis statuam consedimus.

Hic ego: ``laudare igitur eloquentiam et quanta vis sit eius expromere quantamque eis, qui sint eam consecuti, dignitatem afferat, neque propositum nobis est hoc loco neque necessarium. hoc vero sine ulla dubitatione confirmaverim, sive illa arte pariatur aliqua sive exercitatione quadam sive natura, rem unam esse omnium difficillumam. quibus enim ex quinque rebus constare dicitur, earum una quaeque est ars ipsa magna per sese. quare quinque artium concursus maxumarum quantam vim quantamque difficultatem habeat existimari potest.

Testis est Graecia, quae cum eloquentiae studio sit incensa iamdiuque excellat in ea praestetque ceteris, tamen omnis artes vetustiores habet et multo ante non inventas solum, sed etiam perfectas, quam haec est a Graecis elaborata dicendi vis atque copia. in quam cum intueor, maxime mihi occurrunt, Attice, et quasi lucent Athenae tuae, qua in urbe primum se orator extulit primumque etiam monumentis et litteris oratio est coepta mandari (\dots)''

\subsection*{Brutus 307-317}

``(\dots) Occiderat Sulpicius illo anno tresque proxumo trium aetatum oratores erant crudelissume interfecti, Q.~Catulus M.~Antonius C.~Iulius. eodem anno etiam Moloni Rhodio Romae dedimus operam et actori summo causarum et magistro. haec etsi videntur esse a proposita ratione diversa, tamen idcirco a me proferuntur, ut nostrum cursum perspicere, quoniam voluisti, Brute, possis — nam Attico haec nota sunt — et videre quem ad modum simus in spatio Q.~Hortensium ipsius vestigiis persecuti.

Triennium fere fuit urbs sine armis; sed oratorum aut interitu aut discessu aut fuga — nam aberant etiam adulescentes M.~Crassus et Lentuli duo — primas in causis agebat Hortensius, magis magisque cotidie probabatur Antistius, Piso saepe dicebat, minus saepe Pomponius, raro Carbo, semel aut iterum Philippus. at vero ego hoc tempore omni noctes et dies in omnium doctrinarum meditatione versabar.

Eram cum Stoico Diodoto, qui cum habitavisset apud me $\langle$se$\rangle$ cumque vixisset, nuper est domi meae mortuus. a quo cum in aliis rebus tum studiosissime in dialectica exercebar, quae quasi contracta et astricta eloquentia putanda est; sine qua etiam tu, Brute, iudicavisti te illam iustam eloquentiam, quam dialecticam esse dilatatam putant, consequi non posse. huic ego doctori et eius artibus variis atque multis ita eram tamen deditus ut ab exercitationibus oratoriis nullus dies vacuus esset.

Commentabar declamitans — sic enim nunc loquuntur — saepe cum M.~Pisone et cum Q.~Pompeio aut cum aliquo cotidie, idque faciebam multum etiam Latine sed Graece saepius, vel quod Graeca oratio plura ornamenta suppeditans consuetudinem similiter Latine dicendi adferebat, vel quod a Graecis summis doctoribus, nisi Graece dicerem, neque corrigi possem neque doceri.

Tumultus interim recuperanda re publica et crudelis interitus oratorum trium, Scaevolae Carbonis Antisti, reditus Cottae Curionis Crassi Lentulorum Pompei; leges et iudicia constituta, recuperata res publica; ex numero autem oratorum Pomponius Censorinus Murena sublati. tum primum nos ad causas et privatas et publicas adire coepimus, non ut in foro disceremus, quod plerique fecerunt, sed ut, quantum nos efficere potuissemus, docti in forum veniremus.

Eodem tempore Moloni dedimus operam; dictatore enim Sulla legatus ad senatum de Rhodiorum praemiis venerat. itaque prima causa publica pro Sex.~Roscio dicta tantum commendationis habuit, ut non ulla esset quae non digna nostro patrocinio videretur. deinceps inde multae, quas nos diligenter elaboratas et tamquam elucubratas adferebamus.

Nunc quoniam totum me non naevo aliquo aut crepundiis sed corpore omni videris velle cognoscere, complectar nonnulla etiam quae fortasse videantur minus necessaria. erat eo tempore in nobis summa gracilitas et infirmitas corporis, procerum et tenue collum: qui habitus et quae figura non procul abesse putatur a vitae periculo, si accedit labor et laterum magna contentio. eoque magis hoc eos quibus eram carus commovebat, quod omnia sine remissione, sine varietate, vi summa vocis et totius corporis contentione dicebam.

Itaque cum me et amici et medici hortarentur ut causas agere desisterem, quodvis potius periculum mihi adeundum quam a sperata dicendi gloria discedendum putavi. sed cum censerem remissione et moderatione vocis et commutato genere dicendi me et periculum vitare posse et temperatius dicere, ut consuetudinem dicendi mutarem, ea causa mihi in Asiam proficiscendi fuit. itaque cum essem biennium versatus in causis et iam in foro celebratum meum nomen esset, Roma sum profectus.

Cum venissem Athenas, sex menses cum Antiocho veteris Academiae nobilissumo et prudentissumo philosopho fui studiumque philosophiae numquam intermissum a primaque adulescentia cultum et semper auctum hoc rursus summo auctore et doctore renovavi. eodem tamen tempore Athenis apud Demetrium Syrum veterem et non ignobilem dicendi magistrum studiose exerceri solebam. post a me Asia tota peragrata est cum summis quidem oratoribus, quibuscum exercebar ipsis lubentibus; quorum erat princeps Menippus Stratonicensis meo iudicio tota Asia illis temporibus disertissimus; et, si nihil habere molestiarum nec ineptiarum Atticorum est, hic orator in illis numerari recte potest.

adsiduissime autem mecum fuit Dionysius Magnes; erat etiam Aeschylus Cnidius, Adramyttenus Xenocles. hi tum in Asia rhetorum principes numerabantur. quibus non contentus Rhodum veni meque ad eundem quem Romae audiveram Molonem adplicavi cum actorem in veris causis scriptoremque praestantem tum in notandis animadvertendisque vitiis et instituendo docendoque prudentissimum. is dedit operam, si modo id consequi potuit, ut nimis redundantis nos et supra fluentis iuvenili quadam dicendi impunitate et licentia reprimeret et quasi extra ripas diffluentis coerceret. ita recepi me biennio post non modo exercitatior sed prope mutatus. nam et contentio nimia vocis resederat et quasi deferverat oratio lateribusque vires et corpori mediocris habitus accesserat.

Duo tum excellebant oratores qui me imitandi cupiditate incitarent, Cotta et Hortensius; quorum alter remissus et lenis et propriis verbis comprendens solute et facile sententiam, alter ornatus, acer et non talis qualem tu eum, Brute, iam deflorescentem cognovisti, sed verborum et actionis genere commotior. itaque cum Hortensio mihi magis arbitrabar rem esse, quod et dicendi ardore eram propior et aetate coniunctior. etenim videram in isdem causis, ut pro M.~Canuleio, pro Cn.~Dolabella consulari, cum Cotta princeps adhibitus esset, priores tamen agere partis Hortensium. acrem enim oratorem, incensum et agentem et canorum concursus hominum forique strepitus desiderat. (\dots)''
