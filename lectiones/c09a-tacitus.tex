%\section*{O autoru}

\section*{Dialogus de oratoribus (ca.~101)}

\subsection*{Breviarium dialogi}

\begin{quotation}

\noindent 1. Praefatio, scribendique dialogi occasio. 2. Personae, Curiatius Maternus, M.~Aper, Julius Secundus. 3. Maternum poetam a poeticis studiis avocare studet Secundus. 4. Excusat se Maternus. 5. Instat Aper; et contendit utilitate, voluptate, dignitate, fama oratoriam artem arti poeticae anteponendam. 8. Quod Marcelli Eprii et Crispi Vibii exemplis et opibus confirmat. 9. Contra poeticae inanem et infructuosam laudem esse. 10. Hortatur ergo Maternum ut ab auditoriis et theatris ad forum causasque se convertat. 11. Maternus poetas defendit. 12. Iis pura, innocentia, divina sunt studia : magna fama. 13. Speciosas opes : felix contubernium. Inquieta vero et anxia oratorum vita. 14. Haec tractantibus, intervenit Vipstanius Messala. Is, antiquorum admirator, recentium contemtor, Aprum a novorum rhetorum more ad antiquorum oratorum formam trahere nititur. 15. Inde de antiquis et recentibus disputatio. 16. In antiquorum venerationem conspirant Messala, Secundus, et Maternus. Aper antiquos insectatur, et seculi sui partes tuetur. 17. Incertum enim esse et obscurum antiquorum nomen. 18. Plures quoque eloquentiee species. Malignitate humana, vetera in laude, praesentia in fastidio esse. 19. Cassium Severum, quem antiquitatis terminum constituunt, non inscitia, sed judicio et intellectu ad novum dicendi genus se transtulisse. 20. Multa veteris eloquentiae vitia. Recentem esse laetiorem et pulchriorem. 21. De Calvo, Coelio, C. Caesare, Bruto, eorum Ciceronisque carminibus, de Asinio et Corvino judicia. 22. Ciceronis laudes et vitia. 23. Summam eloquentiae artem esse, antiquorum egregia quaeque imitari, et a recentibus feliciter reperta miscere. 24. Maternus Messalam hortatur non ad laudationem antiquorum, quos fama sua satis laudat, sed ut causas, cur in tantum ab eorum eloquentia praesens aetas recesserit, aperiat. 25. Messala quaedam tamen de antiquorum nomine, et judicia de Calvo, Asinio, Caesare, Coelio, Bruto, Cicerone, retractat. 26. C.~Gracchum, L.~Crassum laudat, Maecenatem vero, Gallionem et Cassium Severum reprehendit. 27. Messalam longius vagantem ad propositum retrahit Maternus. 28. Causas ergo cur eloquentia ceteraeque artes a vetere gloria desciverint adfert Messala, desidiam juventutis, negligentiam parentum, inscientiam praecipientium, et oblivionem moris antiqui. 33. Haec cum evolveret Messala, eum restitantem impellit Maternus ut prosequatur, et quomodo veteres ingenia alerent, doceat, 34. Messala veteris juvenum institutionis artes explicat. 35. Recentioris vitia retegit. 

\textit{At in hac parte mutilus est Messalae sermo. Periit integra Julii Secundi oratio. In Materno desunt quoque initia. Quae desiderantur supplere juvat, ut facilius sit hujusce dialogi sensum adsequi, ejusque utilitates percipere.} 1. Retectis pravae institutionis vitiis, Messala juvenum inscitiam, cum ad veros judices ventum est, explodit, eosque, sine ulla publici privatique juris experientia, exili tantum elegantia turgidave granditate ridiculos scenicisque praeceptorum suorum artibus instructos prodire ostendit. 2. Histrionica hac institutione perditam esse eloquentiam concludit. 3. Maternum et Secundum hortatur ut altiores alias, si quae sint, causas aperiant. 4. Julius Secundus non tam novas corruptae eloquentiae causas expromit, quam a Messala allatas explicat, earumque originem demonstrat. 5. Mutata nempe republica, mutata sunt ingenia; et sub Principibus non eloquentia, sed ingenii fama, curae fuit. In Annaeo Seneca coepit illa ingenii prurigo. 6. Ejus exemplo et fortuna fuit incitata. 7. Neque priscis Vespasiani virtutibus repressa. Immo in dies audacior forum inrupit. 8. Suasque in laudes subsellia conduxit, duce Largio Licinio. 9. Fessam tantis propudiis eloquentiam cessasse ait Julius Secundus; Maternumque iis deterritum Musas esse sectatum. 10. Fatetur Maternus se irrepentibus in forum vitiis fuisse commotum; trahente tamen natura, ad Musarum castra transfugisse. 11. Mox ut motam quaestionem dirimat, eloquentiae disertique generis dicendi vices atque ingenia expendit : Rhetoribus indulget. Quintilianum laudat. 12. Nec miratur eloquentiam desiisse, quae Ciceroni superstes esse non valuit. 13. Neque Principum arbitriis erat accommodata. Nusquam enim eloquentiam esse, nisi ubi sit libertas. 14. Inde tantam eloquentiae vim fuisse in Demosthene. \textit{Cetera habet Tacitus.} 

36. Ipsa temporum perturbatione et licentia, cum vigeret respublica, eloquentiam olim valuisse. 37. Iis temporibus neminem magnam potentiam sine eloquentia fuisse consecutum. 38. Cn. Pompeium eloquentiae nocuisse, cum eam adstrinxit, fraenosque ei imposuit. 39. Immo ipsis oratorum poenulis fractam eam fuisse et dejectam, quod pompa plausuque indigeat. 40. Probitate quoque et modestia eam elanguisse, cum natura sit contumax, temeraria et arrogans. 41. Pro varietate ergo temporum variam fuisse eloquentiam; suasque esse quibusvis temporibus laudes et utilitates. 42. Dialogi conclusio.

\medskip

\noindent Habitus est hic dialogus A.U.C. DCCCXXVIII, J.C. 75, Coss. Flavio Vespasiano Aug.\ IV, Tito Vespasiano Caesare IV. 

\end{quotation}


\subsection*{1-3}

Saepe ex me requiris, Iuste Fabi, cur, cum priora saecula tot eminentium oratorum ingeniis gloriaque floruerint, nostra potissimum aetas deserta et laude eloquentiae orbata vix nomen ipsum oratoris retineat; neque enim ita appellamus nisi antiquos, horum autem temporum diserti causidici et advocati et patroni et quidvis potius quam oratores vocantur. Cui percontationi tuae respondere et tam magnae quaestionis pondus excipere, ut aut de ingeniis nostris male existimandum [sit], si idem adsequi non possumus, aut de iudiciis, si nolumus, vix hercule auderem, si mihi mea sententia proferenda ac non disertissimorum, ut nostris temporibus, hominum sermo repetendus esset, quos eandem hanc quaestionem pertractantis iuvenis admodum audivi. 

Ita non ingenio, sed memoria et recordatione opus est, ut quae a praestantissimis viris et excogitata subtiliter et dicta graviter accepi, cum singuli diversas [vel easdem] sed probabilis causas adferrent, dum formam sui quisque et animi et ingenii redderent, isdem nunc numeris isdemque rationibus persequar, servato ordine disputationis. Neque enim defuit qui diversam quoque partem susciperet, ac multum vexata et inrisa vetustate nostrorum temporum eloquentiam antiquorum ingeniis anteferret.

Nam postero die quam Curiatius Maternus Catonem recitaverat, cum offendisse potentium animos diceretur, tamquam in eo tragoediae argumento sui oblitus tantum Catonem cogitasset, eaque de re per urbem frequens sermo haberetur, venerunt ad eum Marcus Aper et Iulius Secundus, celeberrima tum ingenia fori nostri, quos ego utrosque non modo in iudiciis studiose audiebam, sed domi quoque et in publico adsectabar mira studiorum cupiditate et quodam ardore iuvenili, ut fabulas quoque eorum et disputationes et arcana semotae dictionis penitus exciperem, quamvis maligne plerique opinarentur, nec Secundo promptum esse sermonem et Aprum ingenio potius et vi naturae quam institutione et litteris famam eloquentiae consecutum. Nam et Secundo purus et pressus et, in quantum satis erat, profluens sermo non defuit, et Aper omni eruditione imbutus contemnebat potius litteras quam nesciebat, tamquam maiorem industriae et laboris gloriam habiturus, si ingenium eius nullis alienarum artium adminiculis inniti videretur.

Igitur ut intravimus cubiculum Materni, sedentem ipsum[que], quem pridie recitaverat librum, inter manus habentem deprehendimus. Tum Secundus ``nihilne te'' inquit, ``Materne, fabulae malignorum terrent, quo minus offensas Catonis tui ames? An ideo librum istum adprehendisti, ut diligentius retractares, et sublatis si qua pravae interpretationi materiam dederunt, emitteres Catonem non quidem meliorem, sed tamen securiorem?''

\subsection*{36-40}

\textit{(Materni oratio)} ``\dots\ Magna eloquentia, sicut flamma, materia alitur et motibus excitatur et urendo clarescit. Eadem ratio in nostra quoque civitate antiquorum eloquentiam provexit. Nam etsi horum quoque temporum oratores ea consecuti sunt, quae composita et quieta et beata re publica tribui fas erat, tamen illa perturbatione ac licentia plura sibi adsequi videbantur, cum mixtis omnibus et moderatore uno carentibus tantum quisque orator saperet, quantum erranti populo persuaderi poterat. Hinc leges assiduae et populare nomen, hinc contiones magistratuum paene pernoctantium in rostris, hinc accusationes potentium reorum et adsignatae etiam domibus inimicitiae, hinc procerum factiones et assidua senatus adversus plebem certamina. Quae singula etsi distrahebant rem publicam, exercebant tamen illorum temporum eloquentiam et magnis cumulare praemiis videbantur, quia quanto quisque plus dicendo poterat, tanto facilius honores adsequebatur, tanto magis in ipsis honoribus collegas suos anteibat, tanto plus apud principes gratiae, plus auctoritatis apud patres, plus notitiae ac nominis apud plebem parabat. Hi clientelis etiam exterarum nationum redundabant, hos ituri in provincias magistratus reverebantur, hos reversi colebant, hos et praeturae et consulatus vocare ultro videbantur, hi ne privati quidem sine potestate erant, cum et populum et senatum consilio et auctoritate regerent. Quin immo sibi ipsi persuaserant neminem sine eloquentia aut adsequi posse in civitate aut tueri conspicuum et eminentem locum. Nec mirum, cum etiam inviti ad populum producerentur, cum parum esset in senatu breviter censere, nisi qui ingenio et eloquentia sententiam suam tueretur, cum in aliquam invidiam aut crimen vocati sua voce respondendum haberent, cum testimonia quoque in publicis [iudiciis] non absentes nec per tabellam dare, sed coram et praesentes dicere cogerentur. Ita ad summa eloquentiae praemia magna etiam necessitas accedebat, et quo modo disertum haberi pulchrum et gloriosum, sic contra mutum et elinguem videri deforme habebatur.

Ergo non minus rubore quam praemiis stimulabantur, ne clientulorum loco potius quam patronorum numerarentur, ne traditae a maioribus necessitudines ad alios transirent, ne tamquam inertes et non suffecturi honoribus aut non impetrarent aut impetratos male tuerentur. Nescio an venerint in manus vestras haec vetera, quae et in antiquariorum bibliothecis adhuc manent et cum maxime a Muciano contrahuntur, ac iam undecim, ut opinor, Actorum libris et tribus Epistularum composita et edita sunt. Ex his intellegi potest Cn.~Pompeium et M.~Crassum non viribus modo et armis, sed ingenio quoque et oratione valuisse; Lentulos et Metellos et Lucullos et Curiones et ceteram procerum manum multum in his studiis operae curaeque posuisse, nec quemquam illis temporibus magnam potentiam sine aliqua eloquentia consecutum. His accedebat splendor reorum et magnitudo causarum, quae et ipsa plurimum eloquentiae praestant. Nam multum interest, utrumne de furto aut formula et interdicto dicendum habeas, an de ambitu comitiorum, expilatis sociis et civibus trucidatis. Quae mala sicut non accidere melius est isque optimus civitatis status habendus est, in quo nihil tale patimur, ita cum acciderent, ingentem eloquentiae materiam subministrabant. Crescit enim cum amplitudine rerum vis ingenii, nec quisquam claram et inlustrem orationem efficere potest nisi qui causam parem invenit. Non, opinor, Demosthenem orationes inlustrant, quas adversus tutores suos composuit, nec Ciceronem magnum oratorem P. Quintius defensus aut Licinius Archias faciunt: Catilina et Milo et Verres et Antonius hanc illi famam circumdederunt, non quia tanti fuerit rei publicae malos ferre cives, ut uberem ad dicendum materiam oratores haberent, sed, ut subinde admoneo, quaestionis meminerimus sciamusque nos de ea re loqui, quae facilius turbidis et inquietis temporibus existit. Quis ignorat utilius ac melius esse frui pace quam bello vexari? Pluris tamen bonos proeliatores bella quam pax ferunt. Similis eloquentiae condicio. Nam quo saepius steterit tamquam in acie quoque pluris et intulerit ictus et exceperit quoque maiores adversarios acrioresque pugnas sibi ipsa desumpserit, tanto altior et excelsior et illis nobilitata discriminibus in ore hominum agit, quorum ea natura est, ut secura velint, [periculosa mirentur].

Transeo ad formam et consuetudinem veterum iudiciorum. Quae etsi nunc aptior est [ita erit], eloquentiam tamen illud forum magis exercebat, in quo nemo intra paucissimas horas perorare cogebatur et liberae comperendinationes erant et modum in dicendo sibi quisque sumebat et numerus neque dierum neque patronorum finiebatur. primus haec tertio consulatu Cn.~Pompeius adstrinxit imposuitque veluti frenos eloquentiae, ita tamen ut omnia in foro, omnia legibus, omnia apud praetores gererentur: apud quos quanto maiora negotia olim exerceri solita sint, quod maius argumentum est quam quod causae centumvirales, quae nunc primum obtinent locum, adeo splendore aliorum iudiciorum obruebantur, ut neque Ciceronis neque Caesaris neque Bruti neque Caelii neque Calvi, non denique ullius magni oratoris liber apud centumviros dictus legatur, exceptis orationibus Asinii, quae pro heredibus Urbiniae inscribuntur, ab ipso tamen Pollione mediis divi Augusti temporibus habitae, postquam longa temporum quies et continuum populi otium et assidua senatus tranquillitas et maxime principis disciplina ipsam quoque eloquentiam sicut omnia alia pacaverat.

Parvum et ridiculum fortasse videbitur quod dicturus sum, dicam tamen, vel ideo ut rideatur. Quantum humilitatis putamus eloquentiae attulisse paenulas istas, quibus adstricti et velut inclusi cum iudicibus fabulamur? Quantum virium detraxisse orationi auditoria et tabularia credimus, in quibus iam fere plurimae causae explicantur? Nam quo modo nobilis equos cursus et spatia probant, sic est aliquis oratorum campus, per quem nisi liberi et soluti ferantur, debilitatur ac frangitur eloquentia. Ipsam quin immo curam et diligentis stili anxietatem contrariam experimur, quia saepe interrogat iudex, quando incipias, et ex interrogatione eius incipiendum est. frequenter probationibus et testibus silentium \textdagger patronus\textdagger\ indicit. unus inter haec dicenti aut alter adsistit, et res velut in solitudine agitur. Oratori autem clamore plausuque opus est et velut quodam theatro; qualia cotidie antiquis oratoribus contingebant, cum tot pariter ac tam nobiles forum coartarent, cum clientelae quoque ac tribus et municipiorum etiam legationes ac pars Italiae periclitantibus adsisteret, cum in plerisque iudiciis crederet populus Romanus sua interesse quid iudicaretur. Satis constat C.~Cornelium et M.~Scaurum et T.~Nilonem et L.~Bestiam et P.~Vatinium concursu totius civitatis et accusatos et defensos, ut frigidissimos quoque oratores ipsa certantis populi studia excitare et incendere potuerint. Itaque hercule eius modi libri extant, ut ipsi quoque qui egerunt non aliis magis orationibus censeantur.

Iam vero contiones assiduae et datum ius potentissimum quemque vexandi atque ipsa inimicitiarum gloria, cum se plurimi disertorum ne a Publio quidem Scipione aut [L.] Sulla aut Cn.~Pompeio abstinerent, et ad incessendos principes viros, ut est natura invidiae, populi quoque ut histriones auribus uterentur, quantum ardorem ingeniis, quas oratoribus faces admovebant. Non de otiosa et quieta re loquimur et quae probitate et modestia gaudeat, sed est magna illa et notabilis eloquentia alumna licentiae, quam stulti libertatem vocitant, comes seditionum, effrenati populi incitamentum, sine obsequio, sine severitate, contumax, temeraria, adrogans, quae in bene constitutis civitatibus non oritur. Quem enim oratorem Lacedaemonium, quem Cretensem accepimus? Quarum civitatum severissima disciplina et severissimae leges traduntur. Ne Macedonum quidem ac Persarum aut ullius gentis, quae certo imperio contenta fuerit, eloquentiam novimus. Rhodii quidam, plurimi Athenienses oratores extiterunt, apud quos omnia populus, omnia imperiti, omnia, ut sic dixerim, omnes poterant. Nostra quoque civitas, donec erravit, donec se partibus et dissensionibus et discordiis confecit, donec nulla fuit in foro pax, nulla in senatu concordia, nulla in iudiciis moderatio, nulla superiorum reverentia, nullus magistratuum modus, tulit sine dubio valentiorem eloquentiam, sicut indomitus ager habet quasdam herbas laetiores. Sed nec tanti rei publicae Gracchorum eloquentia fuit, ut pateretur et leges, nec bene famam eloquentiae Cicero tali exitu pensavit.

Sic quoque quod superest [antiquis oratoribus fori] non emendatae nec usque ad votum compositae civitatis argumentum est. Quis enim nos advocat nisi aut nocens aut miser? Quod municipium in clientelam nostram venit, nisi quod aut vicinus populus aut domestica discordia agitat? Quam provinciam tuemur nisi spoliatam vexatamque? Atqui melius fuisset non queri quam vindicari. Quod si inveniretur aliqua civitas, in qua nemo peccaret, supervacuus esset inter innocentis orator sicut inter sanos medicus. Quo modo tamen minimum usus minimumque profectus ars medentis habet in iis gentibus, quae firmissima valetudine ac saluberrimis corporibus utuntur, sic minor oratorum honor obscuriorque gloria est inter bonos mores et in obsequium regentis paratos. Quid enim opus est longis in senatu sententiis, cum optimi cito consentiant? Quid multis apud populum contionibus, cum de re publica non imperiti et multi deliberent, sed sapientissimus et unus? Quid voluntariis accusationibus, cum tam raro et tam parce peccetur? Quid invidiosis et excedentibus modum defensionibus, cum clementia cognoscentis obviam periclitantibus eat? credite, optimi et in quantum opus est disertissimi viri, si aut vos prioribus saeculis aut illi, quos miramur, his nati essent, ac deus aliquis vitas ac [vestra] tempora repente mutasset, nec vobis summa illa laus et gloria in eloquentia neque illis modus et temperamentum defuisset: nunc, quoniam nemo eodem tempore adsequi potest magnam famam et magnam quietem, bono saeculi sui quisque citra obtrectationem alterius utatur.''
