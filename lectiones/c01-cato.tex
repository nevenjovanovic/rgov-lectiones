%\section*{O autoru}


\begin{quotation}
\noindent Orationem de sumptu suo (XLI), quam ad consulatus tempus priores rettulere editores, a. 590/164 habitam esse docuit Fraccaro\footnote{``L'orazione di Catone 'de sumptu suo' '' in ``Studi storici per l'ant. class.'' III (1910), pp. 378-386.} Oratio in iudicio quodam de moribus apud censores a Catone dicta videtur, cum ei nimius effususque sumptus crimini daretur, domesticus scilicet, quod pecuniae publicae parcum praeter modum eum fuisse comperimus.\footnote{Plutarch., Cat. mai. 5, 6, ait equum etiam, quo consul in obeundis ducis muneribus usus fuisset, in Hispania eum reliquisse, ne portorium eius civitati imputaretur.} Cum enim parsimonia atque frugalitate rem familiarem auxisset, dulcius vivere coeperat et convivia liberalius instruere, ad quae familiares invitabantur:\footnote{Plutarch Cat. mai. 25, 3.} eius igitur inimici, qui per totam vitam eum exagitavere, siquidem quater et quadragies diem ei dixerunt, ob nimium sumptum apud censores accusasse eum videntur. Cum autem e fr.\ 173 Catonem septuagesimo aetatis anno orationem dixisse constet, a.~590/164, L.\ Aemilio Paulo Q.\ Marcio Philippo censoribus, id factum esse efficiamus licet. Absolutus tamen fuisse videtur, et quod Plinius\footnote{Plin.\ n.\ h. VII 27, 100.} testis est quotiens accusatus sit, totiens absolutum eum esse, et quia si ille vir notatus esset, fieri non potest ut eius notationis memoria omnino excideret.

\noindent Henrica Malcovati, \textit{Oratorum Romanorum fragmenta} I, Torino, 1930. Prolegomena, 69-70.

\end{quotation}


\section*{Marcus Cornelius Fronto (floruit ca. 143), \\Ad M. Antoninum Imp. epist. 1.2.11.4}


Quoniam mentio παραλείψεως habita est, non omittam quin te impertiam quod de figura ista studiosius animadverterim, neque Graecorum oratorum neque Romanorum, quos ego legerim, elegantius hac figura usum quemquam quam M.\ Porcium in ea oratione, quae 'De sumptu suo' inscribitur, in qua sic ait: 
\begin{quotation}
\noindent Iussi caudicem proferri, ubi mea oratio scripta erat de ea re, quod sponsionem feceram cum M.\ Cornelio. Tabulae prolatae: maiorum bene facta perlecta; deinde quae ego pro re p.\ fecissem leguntur. Ubi id utrumque perlectum est, deinde scriptum erat in oratione: `Numquam ego pecuniam neque meam neque sociorum per ambitionem dilargitus sum.' `Attat, noli noli scribere', inquam, `istud: nolunt audire'. Deinde recitavit: `Numquam praefectos per sociorum vestrorum oppida inposivi, qui eorum bona liberos diriperent.' `Istud quoque dele: nolunt audire; recita porro.' `Numquam ego praedam neque quod de hostibus captum esset neque manubias inter pauculos amicos meos divisi, ut illis eriperem qui cepissent.' `Istuc quoque dele: nihil eo minus volunt dici; non opus est recitato.' `Numquam ego evectionem datavi, quo amici mei per symbolos pecunias magnas caperent'. `Perge istuc quoque uti cum maxime delere.' `Numquam ego argentum pro vino congiario inter apparitores atque amicos meos disdidi neque eos malo publico divites feci.' `Enim vero usque istuc ad lignum dele.' Vide sis, quo loco res p.\ siet, uti quod rei p.\ bene fecissem, unde gratiam capiebam, nunc idem illud memorare non audeo, ne invidiae siet. Ita inductum est male facere inpoene, bene facere non inpoene licere.
\end{quotation}

\noindent Haec forma παραλείψεως nova nec ab ullo alio, quod ego sciam, usurpata est. Iubet enim legi tabulas et quod lectum sit, iubet praeteriri.
