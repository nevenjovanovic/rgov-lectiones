%\section*{O autoru}


\section*{Breviarium}
\begin{quotation}
\noindent 9. Lar Porsena, Clusinus rex, precibus Tarquiniorum motus, Romam cum infesto exercitu venit. 10. Horatius Cocles, Sp.~Lartius et T.~Herminius hostes, pontem sublicium transituros, repellunt. 11. Urbs obsidetur. 12. Magnum audaxque facinus C. Mucii Scaevolae. 13. Virtus Cloeliae. Pax.
\end{quotation}


\section*{Titus Livius, Ab Urbe condita (27-9 a.~Chr.~n), 2.12}

adprobant patres; abdito intra uestem ferro proficiscitur. ubi eo uenit, in confertissima turba prope regium tribunal constitit. ibi cum stipendium militibus forte daretur et scriba cum rege sedens pari fere ornatu multa ageret eum$\langle$que$\rangle$ milites uolgo adirent, timens sciscitari uter Porsinna esset, ne ignorando regem semet ipse aperiret quis esset, quo temere traxit fortuna facinus, scribam pro rege obtruncat. 

uadentem inde qua per trepidam turbam cruento mucrone sibi ipse fecerat uiam, cum concursu ad clamorem facto comprehensum regii satellites retraxissent, ante tribunal regis destitutus, tum quoque inter tantas fortunae minas metuendus magis quam metuens, ``Romanus sum'' inquit, ``ciuis; C.~Mucium uocant. hostis hostem occidere uolui, nec ad mortem minus animi est, quam fuit ad caedem; et facere et pati fortia Romanum est. nec unus in te ego hos animos gessi; longus post me ordo est idem petentium decus. proinde in hoc discrimen, si iuuat, accingere, ut in singulas horas capite dimices tuo, ferrum hostemque in uestibulo habeas regiae. hoc tibi iuuentus Romana indicimus bellum. nullam aciem, nullum proelium timueris; uni tibi et cum singulis res erit.'' cum rex simul ira infensus periculoque conterritus circumdari ignes minitabundus iuberet nisi expromeret propere quas insidiarum sibi minas per ambages iaceret, ``en tibi'' inquit, ``ut sentias quam uile corpus sit iis qui magnam gloriam uident''; dextramque accenso ad sacrificium foculo inicit. quam cum uelut alienato ab sensu torreret animo, prope attonitus miraculo rex cum ab sede sua prosiluisset amouerique ab altaribus iuuenem iussisset, ``tu uero abi'' inquit, ``in te magis quam in me hostilia ausus. iuberem macte uirtute esse, si pro mea patria ista uirtus staret; nunc iure belli liberum te, intactum inuiolatumque hinc dimitto.'' 

tunc Mucius, quasi remunerans meritum, ``quando quidem'' inquit, ``est apud te uirtuti honos, ut beneficio tuleris a me quod minis nequisti, trecenti coniurauimus principes iuuentutis Romanae ut in te hac uia grassaremur. mea prima sors fuit; ceteri ut cuiusque ceciderit primi quoad te opportunum fortuna dederit, suo quisque tempore aderunt.''
