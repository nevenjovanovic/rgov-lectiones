%\section*{O autoru}

\section*{De auctore, et quo loco ac tempore pronuntiatus sit} 

\begin{quotation}
\noindent Habita est haec oratio anno Christi 313. post Maxentii cladem, mense opinor Januario, Treveris, antequam Constantinus inde ad pacandos Germaniae tumultus discederet. Auctor incertus. Tamen Puteanus hunc Nazario tribuit: ``Quod'' inquit ``conjicio ex stylo et ex loco Panegyrici Nazarii Constantino dicti qui sic incipit : `Non commemorabo' inquit `hic tectas continuis stragibus ripas, non oppletum acervis corporum Tiberim: perstringi haec satis est, quod et pridie prolixius mihi dicta sunt.' Nempe in hac ipsa Oratione, qua fusius describitur praelium, quo Maxentius debellatus est a Constantino : deinde ex ordine, quo omnes nostri Panegyrici constituuntur in veteribus libris: subjicitur enim huic Panegyricus Nazarii, qui incipit, `Dicturus Constantini' etc.
\end{quotation}

\section*{Synopsis panegyrici}

\begin{quotation}

\noindent In Exordio accessurum se ad Imperatorem laudandum negat, nisi nefas esset ei, quem propter res leviores saepe laudasset, propter recuperatum imperium non gratulari.

\noindent Distributio: I. Quam fortiter Constantinus adversus Maxentium bellum susceperit. II. Quam feliciter confecerit.


\noindent I. Pars. Dissuadentibus amicis, auspiciis adversantibus, imperii sociis paene desperantibus, cum quarta tantum exercitus parte Alpes transgressus, iturus adversus hostes duplo plures, cataphractos, et pro facinorum conscientia nunquam cessuros, cap. II. Maxentium Severus, Galerius, et Maximianus adorti erant cum ingenti exercitu, omnes infeliciter, cap. III. Sed pro Constantino justitia pugnabat. Hujus virtutes cum Maxentii vitiis conferuntur, cap.\ IV. 

\noindent II. Pars. Segusium, c. V. Taurinos, c. VI. Mediolanum, c. VII. Veronam, cap. VIII. Aquileiam capit, ubi ex pertinacioribus alios morte afficit, alios gladiis in vincula mutatis constringi jubet, cap. XI. Victo ad septimum ab urbe milliare Maxentio, eoque inter fugiendum e Mulvio ponte in Tiberim collapso, cap.XIV. victor et triumphans in urbem ingreditur, cap. XIX. Quam dissimilis Cinnae, Mario, et Syllae, cap. XX. Quam clemens erga cives, erga milites bonus, quos victoriarum socios habiturus in inferiorem Germaniam secum deducit, cap. XXI. Constantii gloriam superat, cap. XXIV. Ob quas omnia singularibus a senatu honoribus donatur, cap. XXV.

\noindent In fine Jovem precatur pro aeternitate Principis, cap. XXVI.
\end{quotation}

\section*{Panegyricus} 

Vnde mihi tantum confidentiae, sacratissime imperator, ut post tot homines disertissimos, quos et in Vrbe sacra et hic rursus audisti, dicere auderem, nisi nefas esse ducerem et commissi cuiusdam sacrilegii religionem vererer si is, qui semper res a numine tuo gestas praedicare solitus essem, haec tanto maiora pristinis silentio praeterirem, per quae non pars aliqua servata sed universa sibi est res publica restituta? Neque enim ignoro quanto inferiora nostra sint ingenia Romanis, siquidem latine et diserte loqui illis ingeneratum est, nobis elaboratum et, si quid forte commode dicimus, ex illo fonte et capite [et] facundiae imitatio nostra derivat. Sed quamvis conscius mihi infirmitatis ingenitae et inchoati potius studii quam eruditi, cohibere me silentio nequeo, quominus de recuperata Vrbe imperioque Romano [et] tandem ex diuturna convulsione solidato et ipse aliquid coner effari, ut inter tantos sonitus disertorum mea quoque vox tenuis exaudita videatur. Etenim si in rebus bellicis ipsisque proeliis non solum tubae ac litui sed etiam Spartanae tibiae incentivum aliquod feruntur habuisse (credo quia magnos animos parvo licet modulo sufficit incitari), cur ego in me de tuo favore diffidam, ut sermonem meum studio potius tui cultus quam suis viribus metiaris? Experiar igitur, ut possum, quamquam oppletis auribus tuis, ut sic dixerim, insusurrare, sine aemulandi fiducia cupidus imitandi.

Ac primum illud adripiam quod credo adhuc neminem ausum fuisse, ut ante de constantia expeditionis tuae dicam quam de laude victoriae.  Transacto enim motu adversi hominis et offensione revocata, utar libertate erga te nostri amoris, quem tunc inter metus et rei publicae vota suspendimus. Tene, imperator, tantum animo potuisse praesumere, ut bellum tantis opibus, tanto consensu avaritiae, tanta scelerum contagione, tanta veniae desperatione conflatum quiescentibus cunctantibusque tunc imperii tui sociis primus invaderes! Quisnam te deus, quae tam praesens hortata est maiestas ut, omnibus fere tuis comitibus et ducibus non solum tacite mussantibus sed etiam aperte timentibus, contra consilia hominum, contra haruspicum monita ipse per temet liberandae Vrbis tempus venisse sentires? Habes profecto aliquod cum illa mente divina, Constantine, secretum, quae delegata nostri diis minoribus cura uni se tibi dignatur ostendere. Alioquin, fortissime imperator, sic quoque cum viceris redde rationem. Rhenum tu quidem toto limite dispositis exercitibus tutum reliqueras, sed hoc maiores pro te suscepimus metus quod nobis potius quam tibi consulebas, nostramque pacem magis quam bellum quod adgrediebare firmaveras. 

(\dots)

Studium et humanitas tua hortata est, imperator, ut hucusque exsultatio nostra verbis eluderet; sed iam ad potiora redeamus. Recuperata omni cis Padum Italia ipsa iam ad te supplices manus Roma tendebat, cui portentum illud insederat nihil conari ausum ad tot nuntios suarum cladium. Quippe ipsa se vilissimi hominis obsidebat ignavia et ``degeneres'', ut dictum est, ``animos timor arguebat''. Stultum et nequam animal nusquam extra parietes egredi audebat. Ita enim aut prodigiis aut metus sui praesagiis monebatur.  Pro pudor, intra parietum custodias imperator! Non ille adspirare in campum, non exerceri armis, non pulverem pati: astutus quidem, ne virorum officia temptantem contemnerent qui viderent; deambulare et tantum in illo palatio marmoratis parietibus incedere: nam in Sallustianos hortos ire peregrinatio et expeditio putabatur. Et hae quidem deliciae toto illo tempore quo Vrbem obsederat, semet incluserat, turpissimam hanc eius formidinem protegebant. Non enim se imbellem sed beatum, non inertem sed securum videri volebat. Quotiens milites in contionem vocabat, se solum cum illis imperare, alios per limites pro se militare iactabat. ``Fruimini'', aiebat, ``dissipate, prodigite''. Haec erat miserorum brevis et caduca felicitas.

Ac ne tum quidem, cum tot adversa suorum proelia comperisset, obviam ire conatus est $\langle$ut$\rangle$ ad resistendum Padi limite aut Appennini iugis uteretur, sed litteras calamitatum suarum indices supprimebat; interdum etiam palam ut usque ad portas veniretur optabat, non intellegens maiestatem illam Vrbis, quae pridem admotos exercitus sollicitaverat, iam flagitiis ipsius deformatam et sedibus suis pulsam ad tua auxilia transisse, nec ullis praemiis posse corrumpi quos tibi praeter liberalitatem tuam et sacramenti fidem tot victoriarum gloriae dicavissent. Quis enim miles, qui ductu atque auspiciis tuis totiens felicissime dimicasset, vulnera illi sua venderet, belli iam paene confecti summam concederet? At enim tu id ipsum de ardore totius exercitus sentiens sine ulla haesitandi mora, qua brevissimum per Venetos iter est, rapto agmine advolasti, celeritatem illam in re gerenda Scipionis et Caesaris tunc maxime cupienti Romae repraesentans.  Haec est fiducia imperatoris invicti et suorum mentibus freti, non dubitare nec trahere bellum, sed proximum quodque pugnae tempus putare victoriam. Non enim res erat tibi ut Q.~Maximo cum Hannibale victore, ut post res asperas locum tempusque captares, sed urgere te successus tuos et continuare victorias et quam primum subvenire Vrbi decebat. Sapientis enim est imperatoris fractis rebus cunctando consulere, secundis non deesse Fortunae.

Itaque unum iam illud timebatur, ne ille conterritus, his viribus graviter adflictus et in artum redactus, boni consuleret et debitas rei publicae poenas obsidione differret. Quippe omni Africa quam delere statuerat exhausta, omnibus insulis exinanitis, infiniti temporis annonam congesserat. Sed divina mens et ipsius Vrbis aeterna maiestas nefario homini eripuere consilium, ut ex inveterato illo torpore ac foedissimis latebris subito prorumperet et consumpto per desidias sexennio ipsum diem natalis sui ultima sua caede signaret, ne septenarium illum numerum sacrum et religiosum vel inchoando violaret. At quomodo instruxit aciem tot annorum vernula purpuratus? Ita prorsus ne quis evadere, ne quis, ut fit, loco motus referre gradum et instaurare proelium posset, cum a fronte armis, a tergo Tiberi amne premeretur. In quo ille non mehercule cogitavit necessitatem resistendi sed propinquitatem refugiendi, nisi forte iam sentiens diem sibi venisse fatalem ad solacium suae mortis voluit secum trahere quam plurimos, ut omnes essent exitus sui comites qui scelerum participes exstitissent.  Quid enim aliud illum sperasse credendum est, qui iam ante biduum palatio emigraverat, cum uxore ac filio in privatam domum sponte concesserat, ut res est, somniis terribilibus agitatus et nocturnis pulsus Vltricibus, ut tu iam olim exspectatus habitator sacris illis aedibus diu exalatis expiatisque succederes? [et] Adeo ipse verum sibi dixerat et venturo tibi cesserat, quamvis in proelium ingrederetur armatus, cum excedendo palatio iam se abdicasset imperio.

Ad primum igitur adspectum maiestatis tuae primumque impetum totiens tui victoris exercitus hostes territi fugatique et angustiis Mulvii pontis exclusi, exceptis latrocinii illius primis auctoribus qui desperata venia locum quem pugnae sumpserant texere corporibus, ceteri omnes in fluvium abiere praecipites, ut tandem aliquod caedis compendium fessis tuorum dexteris eveniret. Cum impios Tiberis hausisset, ipsum etiam illum cum equo et armis insignibus frustra conatum per abrupta ripae ulterioris evadere, idem Tiberis correptum gurgite devoravit, ne tam deforme prodigium vel hanc obitus sui relinqueret famam, quod alicuius viri fortis gladio telove cecidisset. Et aliorum quidem hostium corpora et arma praeceps fluvius volvendo devexit; illum autem eodem quo exstinxerat loco tenuit, ne diu populus Romanus dubitaret si putaretur aliquo profugisse cuius mortis probatio quaereretur.

Sancte Thybri, quondam hospitis monitor Aeneae, mox Romuli conservator expositi, tu nec falsum Romulum diu vivere nec parricidam Vrbis passus es enatare. Tu Romae tuae altor copiis subvehendis, tu munitor moenibus ambiendis, merito Constantini victoriae particeps esse voluisti, ut ille hostem in te propelleret, tu necares. Neque enim semper es rapidus et torrens, sed pro temporum ratione moderatus. Tu quietus armatum Coclitem revexisti, tibi se placido Cloelia virgo commisit; at nunc violentus et turbidus hostem rei publicae sorbuisti et, ne tuum lateret obsequium, eructato cadavere prodidisti. Reperto igitur et trucidato corpore universus in gaudia et vindictam populus Romanus exarsit, nec desiit a tota Vrbe, qua suffixum hasta ferebatur, caput illud piaculare foedari, cum interim, ut sunt ioci triumphales, rideretur gestantis iniuria, cum alieni capitis merita pateretur. 

(\dots)

Facile est vincere timidos et imbelles, quales amoena Graeciae et deliciae Orientis educunt, vix leve pallium et sericos sinus vitando sole tolerantes et, si quando in periculum venerint, libertatis immemores, ut servire liceat orantes. Romanum vero militem, quem qualemque ordinat disciplina et sacramenti religio confirmat, aut trucem Francum ferina sola carne distentum, qui vitam pro victus sui vilitate contemnat, quantae molis sit superare vel capere! Quod tu, imperator, et nuper in Italia et in ipso conspectu barbariae paulo ante fecisti. Ita sine ullo discrimine omnia genera bellorum armorum hostium uni tibi cedunt, cedunt ex omni etiam memoria condita litteris monimenta virtutum. Nec vero tantummodo vetera illa dictatorum et consulum ac deinceps magnorum principum, sed etiam recentissima et pulcherrima divi patris tui facta superasti (sordet enim alios ex proximo tempore comparare); ipsum, inquam, divum Constantium iam primis imperii tui lustris rerum gestarum laude cumulasti.

Invitus hoc forte accipis, imperator, sed ille dum dicimus gaudet e caelo, et iam pridem vocatus ad sidera adhuc crescit in filio et gloriarum tuarum gradibus ascendit. Purgavit ille Bataviam advena hoste depulso, tibi se ex ultima barbaria indigenae populi dedidere. Ille Oceanum classe transmisit, tu et Alpes gradu et classibus portus Italicos occupasti. Recuperavit ille Britanniam, tu nobilissimas Africi maris insulas, quae populi Romani fuere provinciae. Ignoscat, inquam, divus ipse Constantius: quid habeo quod comparem Italiae Africae Romae? Merito igitur tibi, Constantine, et nuper senatus signum dei et paulo ante Italia scutum et coronam, cuncta aurea, dedicarunt, ut conscientiae debitum aliqua ex parte relevarent. Debetur enim et semper debebitur et divinitati simulacrum [aureum] et virtuti scutum et corona pietati.

Quamobrem te, summe rerum sator, cuius tot nomina sunt quot gentium linguas esse voluisti (quem enim te ipse dici velis, scire non possumus), sive tute quaedam vis mensque divina es, quae toto infusa mundo omnibus miscearis elementis, et sine ullo extrinsecus accedente vigoris impulsu per te ipsa movearis, sive aliqua supra omne caelum potestas es, quae hoc opus tuum ex altiore Naturae arce despicias: te, inquam, oramus et quaesumus ut hunc in omnia saecula principem serves. Parum est enim optare tantae virtuti tantaeque pietati quem longissimum habet vita processum. Et certe summa in te bonitas est $\langle$et$\rangle$ potestas, et ideo quae iusta sunt velle debes, nec abnuendi est causa cum possis; nam si est aliquid quod a te bene meritis denegetur, aut potestas cessavit aut bonitas. Fac igitur ut, quod optimum humano generi dedisti, permaneat aeternum, omnesque Constantinus in terris degat aetates. Quamvis enim, imperator invicte, iam divina suboles tua ad rei publicae vota successerit et adhuc speretur futura numerosior, illa tamen erit vere beata posteritas ut, cum liberos tuos gubernaculis orbis admoveris, tu sis omnium maximus imperator.
