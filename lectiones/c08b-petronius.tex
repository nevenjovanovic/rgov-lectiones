%\section*{O autoru}


\section*{Satiricon liber (tempore Neronis, 54-66 a.~D)}

``Num alio genere Furiarum declamatores inquietantur, qui clamant: `Haec vulnera pro libertate publica excepi; hunc oculum pro vobis impendi: date mihi ducem, qui me ducat ad liberos meos, nam succisi poplites membra non sustinent'? Haec ipsa tolerabilia essent, si ad eloquentiam ituris viam facerent. Nunc et rerum tumore et sententiarum vanissimo strepitu hoc tantum proficiunt ut, cum in forum venerint, putent se in alium orbem terrarum delatos. Et ideo ego adulescentulos existimo in scholis stultissimos fieri, quia nihil ex his, quae in usu habemus, aut audiunt aut vident, sed piratas cum catenis in litore stantes, sed tyrannos edicta scribentes quibus imperent filiis ut patrum suorum capita praecidant, sed responsa in pestilentiam data, ut virgines tres aut plures immolentur, sed mellitos verborum globulos, et omnia dicta factaque quasi papavere et sesamo sparsa.

Qui inter haec nutriuntur, non magis sapere possunt quam bene olere qui in culina habitant. Pace vestra liceat dixisse, primi omnium eloquentiam perdidistis. Levibus enim atque inanibus sonis ludibria quaedam excitando, effecistis ut corpus orationis enervaretur et caderet. Nondum iuvenes declamationibus continebantur, cum Sophocles aut Euripides invenerunt verba quibus deberent loqui. Nondum umbraticus doctor ingenia deleverat, cum Pindarus novemque lyrici Homericis versibus canere timuerunt. Et ne poetas quidem ad testimonium citem, certe neque Platona neque Demosthenen ad hoc genus exercitationis accessisse video. Grandis et, ut ita dicam, pudica oratio non est maculosa nec turgida, sed naturali pulchritudine exsurgit. Nuper ventosa istaec et enormis loquacitas Athenas ex Asia commigravit animosque iuvenum ad magna surgentes veluti pestilenti quodam sidere adflavit, semelque corrupta regula eloquentia stetit et obmutuit. Ad summam, quis postea Thucydidis, quis Hyperidis ad famam processit? Ac ne carmen quidem sani coloris enituit, sed omnia quasi eodem cibo pasta non potuerunt usque ad senectutem canescere. Pictura quoque non alium exitum fecit, postquam Aegyptiorum audacia tam magnae artis compendiariam invenit.''

Non est passus Agamemnon me diutius declamare in porticu, quam ipse in schola sudaverat, sed: ``Adulescens'', inquit, ``quoniam sermonem habes non publici saporis et, quod rarissimum est, amas bonam mentem, non fraudabo te arte secreta. $\langle$Nihil$\rangle$ nimirum in his exercitationibus doctores peccant qui necesse habent cum insanientibus furere. Nam nisi dixerint quae adulescentuli probent, ut ait Cicero, `soli in scolis relinquentur'. Sicut ficti adulatores cum cenas divitum captant, nihil prius meditantur quam id quod putant gratissimum auditoribus fore — nec enim aliter impetrabunt quod petunt, nisi quasdam insidias auribus fecerint — sic eloquentiae magister, nisi tanquam piscator eam imposuerit hamis escam, quam scierit appetituros esse pisciculos, sine spe praedae morabitur in scopulo.

Quid ergo est? Parentes obiurgatione digni sunt, qui nolunt liberos suos severa lege proficere. Primum enim sic ut omnia, spes quoque suas ambitioni donant. Deinde cum ad vota properant, cruda adhuc studia in forum impellunt, et eloquentiam, qua nihil esse maius confitentur, pueris induunt adhuc nascentibus. Quod si paterentur laborum gradus fieri, ut sapientiae praeceptis animos componerent, ut verba atroci stilo effoderent, ut quod vellent imitari diti audirent, $\langle$ut persuaderent$\rangle$ sibi nihil esse magnificum quod pueris placeret: iam illa grandis oratio haberet maiestatis suae pondus. Nunc pueri in scholis ludunt, iuvenes ridentur in foro, et quod utroque turpius est, quod quisque $\langle$puer$\rangle$ perperam didicit, in senectute confiteri non vult\dots''
