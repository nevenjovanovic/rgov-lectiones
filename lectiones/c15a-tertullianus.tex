%\section*{O autoru}

\section*{Apologeticum (197)}

Pars I: iniquitas odii contra Christianorum (1-3). Pars II: praemunitio (4-6). Pars III: adversus intentationem scelerum domesticorum Christianorum (7-9). Pars IV: scelera Christianorum in publico, adversus intentationem laesae divinitatis (10-27). Pars V: scelera Christianorum in publico, adversus intentationem laesae maiestatis (28-45). Pars VI: philosophi et Christiani (46-50).

\subsection*{1}

Si non licet vobis, Romani imperii antistites, in aperto et edito, in ipso fere vertice civitatis praesidentibus ad iudicandum palam dispicere et coram examinare, quid sit liquido in causa Christianorum, si ad hanc solam speciem auctoritas vestra de iustitiae diligentia in publico aut timet aut erubescit inquirere, si denique, quod proxime accidit, domesticis iudiciis nimis operata infestatio sectae huius obstruit defensioni: liceat veritati vel occulta via tacitarum litterarum ad aures vestras pervenire.

Nihil de causa sua deprecatur, quia nec de condicione miratur. Scit se peregrinam in terris agere, inter extraneos facile inimicos invenire, ceterum genus, sedem, spem, gratiam, dignitatem in caelis habere. Unum gestit interdum, ne ignorata damnetur. Quid hic deperit legibus in suo regno dominantibus, si audiatur? An hoc magis gloriabitur potestas earum, quo etiam auditam damnabunt veritatem? Ceterum inauditam si damnent, praeter invidiam iniquitatis etiam suspicionem merebuntur alicuius conscientiae, nolentes audire, quod auditum damnare non possint.

Hanc itaque primam causam apud vos collocamus iniquitatis odii erga nomen Christianorum. Quam iniquitatem idem titulus et onerat et revincit, qui videtur excusare, ignorantia scilicet. Quid enim iniquius, quam ut oderint homines quod ignorant, etiam si res meretur odium? Tunc etenim meretur, cum cognoscitur an mereatur.  Vacante autem meriti notitia, unde odii iustitia defenditur, quae non de eventu, sed de conscientia probanda est? Cum ergo propterea oderunt homines, quia ignorant, quale sit quod oderunt, cur non liceat eiusmodi illud esse, quod non debeant odisse? Ita utrumque ex alterutro redarguimus, et ignorare illos, dum oderunt, et iniuste odisse, dum ignorant.

\subsection*{7}

Si semper latemus, quando proditum est quod admittimus? Immo a quibus prodi potuit? Ab ipsis enim reis non utique, cum vel ex forma omnibus mysteriis silentii fides debeatur. Samothracia et Eleusinia reticentur: Quanto magis talia, quae prodita interim etiam humanam animadversionem provocabunt, dum divina servatur! Si ergo non ipsi proditores sui, sequitur ut extranei. Et unde extraneis notitia, cum semper etiam piae initiationes arceant profanos et ab arbitris caveant? Nisi si impii minus metuunt.

Natura famae omnibus nota est. Vestrum est: ``Fama malum, qua non aliud velocius ullum.'' Cur malum fama? Quia velox? Quia index? An quia plurimum mendax? Quae ne tunc quidem, cum aliquid veri affert, sine mendacii vitio est, detrahens, adiciens, demutans de veritate. Quid quod ea illi condicio est, ut non nisi cum mentitur, perseveret, et tamdiu vivit, quamdiu non probat? Siquidem, ubi probavit, cessat esse et quasi officio nuntiandi functa rem tradit; et exinde res tenetur, res nominatur. Nec quisquam dicit verbi gratia: ``Hoc Romae aiunt factum'', aut: ``Fama est illum provinciam sortitum'', sed: ``Sortitus est ille provinciam'', et: ``Hoc factum est Romae''. Fama, nomen incerti, locum non habet, ubi certum est. An vero famae credat nisi inconsideratus? Quia sapiens non credit incerto. Omnium est aestimare, quantacumque illa ambitione diffusa sit, quantacumque adseveratione constructa, quod ab uno aliquando principe exorta sit necesse est. Exinde in traduces linguarum et aurium serpit, et ita modici seminis vitium cetera rumoris obscurat, ut nemo recogitet, ne primum illud os mendacium seminaverit, quod saepe fit aut ingenio aemulationis aut arbitrio suspicionis aut non nova, sed ingenita quibusdam mentiendi voluptate. Bene autem quod omnia tempus revelat, testibus etiam vestris proverbiis atque sententiis, ex dispositione naturae, quae ita ordinavit, ut nihil diu lateat, etiam quod fama non distulit.

Merito igitur fama tamdiu conscia sola est scelerum Christianorum; hanc indicem adversus nos profertis, quae quod aliquando iactavit tantoque spatio in opinionem corroboravit, usque adhuc probare non valuit.

Ut fidem naturae ipsius appellem adversus eos, qui talia credenda esse praesumunt, ecce proponimus horum facinorum mercedem: Vitam aeternam repromittunt. Credite interim! De hoc enim quaero, an et qui credideris tanti habeas ad eam tali conscientia pervenire. Veni, demerge ferrum in infantem nullius inimicum, nullius reum, omnium filium; vel, si alterius officium est, tu modo adsiste morienti homini, antequam vixit; fugientem animam novam expecta, excipe rudem sanguinem, eo panem tuum satia, vescere libenter! Interea discumbens dinumera loca, ubi mater, ubi soror; nota diligenter, ut, cum tenebrae ceciderint caninae, non erres! Piaculum enim admiseris, nisi incestum feceris. Talia initiatus et consignatus vivis in aevum. Cupio respondeas, si tanti aeternitas; aut si non, ideo nec credenda. Etiamsi credideris, nego te velle; etiamsi volueris, nego te posse. Cur ergo alii possint, si vos non potestis? cur non possitis, si alii possunt? Alia nos, opinor, natura, Cynopennae aut Sciapodes; alii ordines dentium, alii ad incestam libidinem nervi. Qui ista credis de homine, potes et facere; homo es et ipse, quod et Christianus. Qui non potes facere, non debes credere. Homo est enim et Christianus, et quod et tu. 

