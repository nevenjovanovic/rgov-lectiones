%\section*{O autoru}


\section*{Operis conspectus}
\begin{quotation}
\noindent 1.1   Prooemium\\
1.2   Tria genera causarum: demonstrativum, deliberativum, iudicale\\
1.3   Officia oratoris: inventio, dispositio, elocutio, memoria, pronuntiatio; ars, imitatio, exercitatio\\
1.4 - 3.15   Inventio\\
1.4   Partes orationis: exordium, narratio, divisio, confirmatio, confutatio, conclusio\\
1.5   Quattuor genera causarum: honestum, turpe, dubium, humile\\
3.16 - 3.18   Dispositio\\
3.19 - 3.27   Pronuntiatio\\
3.28 - 3.40   Memoria\\
4.1 - 4.68   Elocutio\\
4.1 - 4.10   Prooemium\\
4.69   Epilogus
\end{quotation}


\section*{1.1–4}

Etsi negotiis familiaribus inpediti vix satis otium studio suppeditare possumus et id ipsum, quod datur otii, libentius in philosophia consumere consuevimus, tamem tua nos, Gai Herenni, voluntas commovit, ut de ratione dicendi conscriberemus, ne aut tua causa noluisse aut fugisse nos laborem putares. Et eo studiosius hoc negotium suscepimus, quod te non sine causa velle cognoscere rhetoricam intellegebamus: non enim in se parum fructus habet copia dicendi et commoditas orationis, si recta intellegentia et definita animi moderatione gubernetur. 

Quas ob res illa, quae Graeci scriptores inanis adrogantiae causa sibi adsumpserunt, reliquimus. Nam illi, ne parum multa scisse viderentur, ea conquisierunt, quae nihil adtinebant, ut ars difficilior cognitu putaretur, nos autem ea, quae videbantur ad rationem dicendi pertinere, sumpsimus. Non enim spe quaestus aut gloria commoti venimus ad scribendum, quemadmodum ceteri, sed ut industria nostra tuae morem geramus voluntati. 

Nunc, ne nimium longa sumatur oratio, de re dicere incipiemus, si te unum illud monuerimus, artem sine adsiduitate dicendi non multum iuvare, ut intellegas hanc rationem praeceptionis ad exercitationem adcommodari oportere.


Oratoris officium est de iis rebus posse dicere, quae res ad usum civilem moribus et legibus constitutae sunt, cum adsensione auditorum, quoad eius fieri poterit.

Tria genera sunt causarum, quae recipere debet orator: demonstrativum, deliberativum, iudiciale.

Demonstrativum est, quod tribuitur in alicuius certae personae laudem vel vituperationem.

Deliberativum est in consultatione, quod habet in se suasionem et dissuasionem.

Iudiciale est, quod positum est in controversia et quod habet accusationem aut petitionem cum defensione.

Nunc quas res oratorem habere oporteat, docebimus, deinde quo modo has causas tractari conveniat, ostendemus.


Oportet igitur esse in oratore inventionem, dispositionem, elocutionem, memoriam, pronuntiationem.

Inventio est excogitatio rerum verarum aut veri similium, quae causam probabilem reddant.

Dispositio est ordo et distributio rerum, quae demonstrat, quid quibus locis sit conlocandum.

Elocutio est idoneorum verborum et sententiarum ad inventionem adcommodatio.

Memoria est firma animi rerum et verborum et dispositionis perceptio.

Pronuntiatio est vocis, vultus, gestus moderatio cum venustate.

Haec omnia tribus rebus adsequi poterimus: arte, imitatione, exercitatione.

Ars est praeceptio, quae dat certam viam rationemque dicendi.

Imitatio est, qua inpellimur cum diligenti ratione ut aliquorum similes in dicendo valeamus esse.

Exercitatio est adsiduus usus consuetudoque dicendi.

Quoniam ergo demonstratum est, quas causas oratorem recipere quasque res habere conveniat, nunc, quemadmodum possit oratio ad rationem oratoris officii adcommodari, dicendum videtur.

Inventio in sex partes orationis consumitur: in exordium, narrationem, divisionem, confirmationem, confutationem, conclusionem.

Exordium est principium orationis, per quod animus auditoris constituitur ad audiendum.

Narratio est rerum gestarum aut proinde ut gestarum expositio.

Divisio est, per quam aperimus, quid conveniat, quid in controversia sit, et per quam exponimus, quibus de rebus simus acturi.

Confirmatio est nostrorum argumentorum expositio cum adseveratione.

Confutatio est contrariorum locorum dissolutio.

Conclusio est artificiosus orationis terminus.

Nunc, quoniam una cum oratoris officiis, quo res cognitu facilior esset, producti sumus, ut de orationis partibus loqueremur et eas ad inventionis rationem adcommodaremus, de exordio primum dicendum videtur.

\section*{1.7}

Causarum divisio in duas partes distributa est.

Primum per narrationem debemus aperire, quid nobis conveniat cum adversariis si ea, quae utilia sunt nobis, convenient, quid in controversiis $\langle$relictum sit$\rangle$, hoc modo:

``Interfectam esse ab Oreste matrem convenit mihi cum adversariis: iure fecerit et licueritne facere, id est in controversia.''

Item e contrario:

``Agamemnonem esse a Clytemestra occisum confitentur; cum id ita sit, me ulcisci parentem negant oportuisse.''

Deinde, cum hoc fecerimus, distributione uti debemus. Ea dividitur in duas partes: enumerationem et expositionem.

Enumeratione utemur, cum dicemus numero, quot de rebus dicturi sumus. Eam plus quam trium partium numero $\langle$esse$\rangle$ non oportet: nam et periculosum est, ne quando plus minusve dicamus; et suspicionem adfert auditori meditationis et artificii: quae res fidem abrogat orationi.

Expositio est, cum res, quibus de rebus dicturi sumus, exponimus breviter et absolute.

\section*{3.28–33}

Nunc ad thesaurum inventorum atque ad omnium partium rhetoricae custodem, memoriam, transeamus.

MEMORIA utrum habeat quiddam artificiosi, an omnis ab natura proficiscatur, aliud dicendi tempus $\langle$magis$\rangle$ idoneum dabitur. Nunc proinde atque constet in hac re multum valere artem et praeceptionem, ita de ea re loquemur. Placet enim nobis esse artificium memoriae; quare placeat, alias ostendemus; in praesentia, cuiusmodi sit ea, aperiemus.

Sunt igitur duae memoriae: una naturalis, altera artificiosa. Naturalis est ea, quae nostris animis insita est et simul cum cogitatione nata; artificiosa est ea, quam confirmat inductio quaedam et ratio praeceptionis. Sed qua via in ceteris rebus ingenii bonitas imitatur saepe doctrinam, ars porro naturae commoda confirmat et auget, item fit in hac re, ut nonnumquam naturalis memoria, si cui data est egregia, similis sit huic artificiosae, porro haec artificiosa naturae commoda retineat et amplificet ratione doctrinae; quapropter $\langle$et$\rangle$ naturalis memoria praeceptione confirmanda est, ut sit egregia, et haec, quae doctrina datur, indiget ingenii. Nec hoc magis aut minus in hac re, quam in ceteris artibus fit, ut ingenio doctrina, praeceptione natura nitescat. Quare et illis, qui natura memores sunt, utilis haec erit institutio, quod tute paulo post poteris intellegere: et si illei, freti ingenio, nostri non indigerent, tamen iusta causa daretur, quare iis, qui minus ingenii habent, adiumento velimus esse. Nunc de artificiosa memoria loquemur.

Constat igitur artificiosa memoria locis et imaginibus. Locos appellamus eos, qui breviter, perfecte, insignite aut natura aut manu sunt absoluti, ut eos facile naturali memoria conprehendere et amplecti queamus: $\langle$ut$\rangle$ aedes, intercolumnium, angulum, fornicem et alia, quae his similia sunt. Imagines sunt formae quaedam et notae et simulacra eius rei, quam meminisse volumus: quod genus equi, leones, aquilae; [memoriam] si volemus habere imagines eorum, locis certis conlocare oportebit. Nunc, cuiusmodi locos invenire et quo pacto reperire et in locis imagines constituere oporteat, ostendemus.

Quemadmodum igitur qui litteras sciunt, possunt id, quod dictatur, eis scribere et recitare quod scripserunt, item qui nemonica didicerunt, possunt, quod audierunt, in locis conlocare $\langle$et$\rangle$ ex his memoriter pronuntiare. Nam loci cerae aut cartae simillimi sunt, imagines litteris, dispositio et conlocatio imaginum scripturae, pronuntiatio lectioni.

Oportet igitur, si volumus multa meminisse, multos $\langle$nos$\rangle$ nobis locos conparare, uti multis locis multas imagines conlocare possimus. Item putamus oportere $\langle$ex ordine hos locos habere,$\rangle$ ne quando perturbatione ordinis inpediamur, quo setius, quoto quoquo loco libebit, vel ab superiore vel ab inferiore parte imagines sequi et ea, quae mandata locis erunt, edere possimus: nam ut, si in ordine stantes notos quomplures viderimus, nihil nostra intersit, utrum ab summo an ab imo an ab medio nomina eorum dicere incipiamus, item in locis ex ordine conlocatis eveniet, ut in quamlibebit partem quoque loco lubebit imaginibus commoniti dicere possimus id, quod locis mandaverimus: quare placet et ex ordine locos conparare. Locos, quos sumpserimus, egregie commeditari oportebit, ut perpetuo nobis haerere possint: nam imagines, sicuti litterae delentur, ubi nihil utimur; loci, tamquam cera, remanere debent. Et, ne forte in numero locorum falli possimus, quintum quemque placet notari: quod genus, si in quinto loco manum auream conlocemus, $\langle$si$\rangle$ in decumo aliquem notum, cui praenomen sit Decumo; deinde facile erit inceps similis notas quinto quoquo loco conlocare. Item commodius est in derelicta, quam in celebri regione locos conparare, propterea quod frequentia et obambulatio hominum conturbat et infirmat imaginum notas, solitudo conservat integras simulacrorum figuras. Praeterea dissimilis forma atque natura loci conparandi sunt, ut distincti interlucere possint: nam si qui multa intercolumnia sumpserit, conturbabitur similitudine, ut ignoret, quid in quoquo loco conlocarit. Et magnitudine modica et mediocris locos habere oportet: nam et praeter modum ampli vagas imagines reddunt et nimis angusti saepe non videntur posse capere imaginum conlocationem.

Tum nec nimis inlustris nec vehementer obscuros locos habere oportet, ne aut obcaecentur tenebris imagines aut splendore praefulgeant. Intervalla locorum mediocria placet esse, fere paulo plus aut minus pedum tricenum: nam ut aspectus item cogitatio minus valet, sive nimis procul removeris sive vehementer prope admoveris id, quod oportet videri.

Sed quamquam facile est ei, qui paulo plura noverit, quamvis multos et idoneos locos conparare, tamen si qui satis idoneos invenire se non putabit, ipse sibi constituat quam volet multos licebit. Cogitatio enim quamvis regionem potest amplecti et in ea situm loci cuiusdam ad suum arbitrium fabricari et architectari. Quare licebit, si hac prompta copia contenti non erimus, nosmet ipsos nobis cogitatione nostra regionem constituere et idoneorum locorum commodissimam distinctionem conparare. 

De locis satis dictum est; nunc ad imaginum rationem transeamus.

Quoniam ergo rerum similes imagines esse oportet, ex omnibus rebus nosmet nobis similitudines eligere debemus. Duplices igitur similitudines esse debent, unae rerum, alterae verborum. Rerum similitudines exprimuntur, cum summatim ipsorum negotiorum imagines conparamus; verborum similitudines constituuntur, cum unius cuiusque nominis et vocabuli memoria imagine notatur.

Rei totius memoriam saepe una nota et imagine simplici conprehendimus; hoc modo, ut si accusator dixerit ab reo hominem veneno necatum, et hereditatis causa factum arguerit, et eius rei multos dixerit testes et conscios esse: si hoc primum, ut ad defendendum nobis expeditum sit, meminisse volemus, in primo loco rei totius imaginem conformabimus: aegrotum in lecto cubantem faciemus ipsum illum, de quo agetur, si formam eius detinebimus; si eum non, at aliquem aegrotum non de minimo loco sumemus, ut cito in mentem venire possit. Et reum ad lectum eius adstituemus, dextera poculum, sinistra tabulas, medico testiculos arietinos tenentem: hoc modo et testium et hereditatis et veneno necati memoriam habere poterimus.


