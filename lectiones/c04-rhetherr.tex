%\section*{O autoru}


\section*{Operis conspectus}
\begin{quotation}
\noindent 1.1   Prooemium\\
1.2   Tria genera causarum: demonstrativum, deliberativum, iudicale\\
1.3   Officia oratoris: inventio, dispositio, elocutio, memoria, pronuntiatio; ars, imitatio, exercitatio\\
1.4 - 3.15   Inventio\\
1.4   Partes orationis: exordium, narratio, divisio, confirmatio, confutatio, conclusio\\
1.5   Quattuor genera causarum: honestum, turpe, dubium, humile\\
3.16 - 3.18   Dispositio\\
3.19 - 3.27   Pronuntiatio\\
3.28 - 3.40   Memoria\\
4.1 - 4.68   Elocutio\\
4.1 - 4.10   Prooemium\\
4.69   Epilogus
\end{quotation}


\section*{1.2}

Oratoris officium est de iis rebus posse dicere, quae res ad usum civilem moribus et legibus constitutae sunt, cum adsensione auditorum, quoad eius fieri poterit.

Tria genera sunt causarum, quae recipere debet orator: demonstrativum, deliberativum, iudiciale.

Demonstrativum est, quod tribuitur in alicuius certae personae laudem vel vituperationem.

Deliberativum est in consultatione, quod habet in se suasionem et dissuasionem.

Iudiciale est, quod positum est in controversia et quod habet accusationem aut petitionem cum defensione.

Nunc quas res oratorem habere oporteat, docebimus, deinde quo modo has causas tractari conveniat, ostendemus.

Oportet igitur esse in oratore inventionem, dispositionem, elocutionem, memoriam, pronuntiationem.

Inventio est excogitatio rerum verarum aut veri similium, quae causam probabilem reddant.

Dispositio est ordo et distributio rerum, quae demonstrat, quid quibus locis sit conlocandum.

Elocutio est idoneorum verborum et sententiarum ad inventionem adcommodatio.

Memoria est firma animi rerum et verborum et dispositionis perceptio.

Pronuntiatio est vocis, vultus, gestus moderatio cum venustate.

Haec omnia tribus rebus adsequi poterimus: arte, imitatione, exercitatione.

Ars est praeceptio, quae dat certam viam rationemque dicendi.

Imitatio est, qua inpellimur cum diligenti ratione ut aliquorum similes in dicendo valeamus esse.

Exercitatio est adsiduus usus consuetudoque dicendi.

\section*{1.7}

Causarum divisio in duas partes distributa est.

Primum per narrationem debemus aperire, quid nobis conveniat cum adversariis si ea, quae utilia sunt nobis, convenient, quid in controversiis $\langle$relictum sit$\rangle$, hoc modo:

``Interfectam esse ab Oreste matrem convenit mihi cum adversariis: iure fecerit et licueritne facere, id est in controversia.''

Item e contrario:

``Agamemnonem esse a Clytemestra occisum confitentur; cum id ita sit, me ulcisci parentem negant oportuisse.''

Deinde, cum hoc fecerimus, distributione uti debemus. Ea dividitur in duas partes: enumerationem et expositionem.

Enumeratione utemur, cum dicemus numero, quot de rebus dicturi sumus. Eam plus quam trium partium numero $\langle$esse$\rangle$ non oportet: nam et periculosum est, ne quando plus minusve dicamus; et suspicionem adfert auditori meditationis et artificii: quae res fidem abrogat orationi.

Expositio est, cum res, quibus de rebus dicturi sumus, exponimus breviter et absolute.

\section*{3.28}

Nunc ad thesaurum inventorum atque ad omnium partium rhetoricae custodem, memoriam, transeamus.

MEMORIA utrum habeat quiddam artificiosi, an omnis ab natura proficiscatur, aliud dicendi tempus $\langle$magis$\rangle$ idoneum dabitur. Nunc proinde atque constet in hac re multum valere artem et praeceptionem, ita de ea re loquemur. Placet enim nobis esse artificium memoriae; quare placeat, alias ostendemus; in praesentia, cuiusmodi sit ea, aperiemus.

Sunt igitur duae memoriae: una naturalis, altera artificiosa. Naturalis est ea, quae nostris animis insita est et simul cum cogitatione nata; artificiosa est ea, quam confirmat inductio quaedam et ratio praeceptionis. Sed qua via in ceteris rebus ingenii bonitas imitatur saepe doctrinam, ars porro naturae commoda confirmat et auget, item fit in hac re, ut nonnumquam naturalis memoria, si cui data est egregia, similis sit huic artificiosae, porro haec artificiosa naturae commoda retineat et amplificet ratione doctrinae; quapropter $\langle$et$\rangle$ naturalis memoria praeceptione confirmanda est, ut sit egregia, et haec, quae doctrina datur, indiget ingenii. Nec hoc magis aut minus in hac re, quam in ceteris artibus fit, ut ingenio doctrina, praeceptione natura nitescat. Quare et illis, qui natura memores sunt, utilis haec erit institutio, quod tute paulo post poteris intellegere: et si illei, freti ingenio, nostri non indigerent, tamen iusta causa daretur, quare iis, qui minus ingenii habent, adiumento velimus esse. Nunc de artificiosa memoria loquemur.

Constat igitur artificiosa memoria locis et imaginibus. Locos appellamus eos, qui breviter, perfecte, insignite aut natura aut manu sunt absoluti, ut eos facile naturali memoria conprehendere et amplecti queamus: $\langle$ut$\rangle$ aedes, intercolumnium, angulum, fornicem et alia, quae his similia sunt. Imagines sunt formae quaedam et notae et simulacra eius rei, quam meminisse volumus: quod genus equi, leones, aquilae; [memoriam] si volemus habere imagines eorum, locis certis conlocare oportebit. Nunc, cuiusmodi locos invenire et quo pacto reperire et in locis imagines constituere oporteat, ostendemus.

Quemadmodum igitur qui litteras sciunt, possunt id, quod dictatur, eis scribere et recitare quod scripserunt, item qui nemonica didicerunt, possunt, quod audierunt, in locis conlocare $\langle$et$\rangle$ ex his memoriter pronuntiare. Nam loci cerae aut cartae simillimi sunt, imagines litteris, dispositio et conlocatio imaginum scripturae, pronuntiatio lectioni.

Oportet igitur, si volumus multa meminisse, multos $\langle$nos$\rangle$ nobis locos conparare, uti multis locis multas imagines conlocare possimus. Item putamus oportere $\langle$ex ordine hos locos habere,$\rangle$ ne quando perturbatione ordinis inpediamur, quo setius, quoto quoquo loco libebit, vel ab superiore vel ab inferiore parte imagines sequi et ea, quae mandata locis erunt, edere possimus: nam ut, si in ordine stantes notos quomplures viderimus, nihil nostra intersit, utrum ab summo an ab imo an ab medio nomina eorum dicere incipiamus, item in locis ex ordine conlocatis eveniet, ut in quamlibebit partem quoque loco lubebit imaginibus commoniti dicere possimus id, quod locis mandaverimus: quare placet et ex ordine locos conparare. Locos, quos sumpserimus, egregie commeditari oportebit, ut perpetuo nobis haerere possint: nam imagines, sicuti litterae delentur, ubi nihil utimur; loci, tamquam cera, remanere debent. Et, ne forte in numero locorum falli possimus, quintum quemque placet notari: quod genus, si in quinto loco manum auream conlocemus, $\langle$si$\rangle$ in decumo aliquem notum, cui praenomen sit Decumo; deinde facile erit inceps similis notas quinto quoquo loco conlocare.

\section*{4.63-65}

Notatio est, cum alicuius natura certis describitur signis, quae, sicuti notae quae naturae sunt adtributa; ut si velis non divitem, sed ostentatorem pecuniosi describere:

``Iste'', inquies, ``iudices, qui se dici divitem putabat esse praeclarum, primum nunc videte, quo vultu nos intueatur. Nonne vobis videtur dicere: `*** dant, si mihi molesti non essetis?' Cum vero sinistra mentum sublevavit, existimat se gemmae nitore et auri splendore aspectus omnium praestringere. – Cum puerum respicit hunc unum, quem ego novi – vos non arbitror –, alio nomine appellat, deinde alio atque alio. `At eho tu', inquit, `veni, Sannio, ne quid is barbaris turbent'; ut ignoti, qui audient, unum putent selegi de multis. Ei dicit in aurem, aut ut domi lectuli sternantur, aut ab avunculo rogetur Aethiops qui ad balineas veniat, aut asturconi locus ante ostium suum detur, aut aliquod fragile falsae choragium gloriae conparetur. Deinde exclamat, ut omnes audiant: `Videto, ut diligenter numeretur, si potest, ante noctem.' Puer, qui iam bene eri naturam norit: `Tu illo $\langle$plures$\rangle$ mittas oportet', inquit, `si hodie vis transnumerari.' `Age' inquit, `duc tecum Libanum et Sosiam.' `Sane.' Deinde casu veniunt hospites homini, quos iste, dum splendide peregrinatur, $\langle$invitat.$\rangle$ Ex ea re homo hercule sane conturbatur; sed tamen a vitio naturae non recedit. `Bene', inquit, `facitis, cum venitis: sed rectius fecissetis, si ad me domum recta abissetis.' `Id fecissemus', inquiunt illi, `si domum novissemus.' `At istud quidem facile fuit undelibet invenire. Verum ite mequum.' Secuntur illi. Sermo interea huius consumitur omnis in ostentatione: quaerit, in agris frumenta cuiusmodi sint; negat se, quia villae incensae sint, accedere posse: nec aedificare etiamnunc audere; `tametsi in Tusculano quidem coepi insanire et in isdem fundamentis aedificare.' Dum haec loquitur, venit in aedes quasdam, in quibus sodalicium erat eodem die futurum; quo iste pro notitia domnaedi iam it intro cum hospitibus. `Hic', inquit, `habito.' Perspicit argentum, quod erat expositum, visit triclinium stratum: probat. Accedit servulus; dicit homini clare, dominum iam venturum, si velit exire. `Itane?' inquit. `Eamus hospitis; frater venit ex Falerno: ego illi obviam pergam; vos huc decuma venitote.' Hospites discedunt. Iste se raptim domum suam conicit; $\langle$illi$\rangle$ decuma, quo iusserat, veniunt. Quaerunt hunc; reperiunt, domus cuia sit; in diversorium derisi conferunt sese. Vident hominem posteri die; narrant, expostulant, accusant. Ait iste eos similitudine loci deceptos angiporto toto deerrasse; contra valetudinem suam ad noctem multam expectasse. Sannioni puero negotium dederat, ut vasa, vestimenta, pueros rogaret: servolus non inurbanus satis strenue et concinne conparat. Iste hospites domum deducit: ait se aedes maximas cuidam amico ad nuptias commodasse. Nuntiat puer argentum repeti: pertimuerat enim, qui commodarat. `Apage $\langle$te',$\rangle$ inquit, `aedes commodavi, familiam dedi: argentum quoque vult? Tametsi hospites habeo, tamen utatur licet, nos Samis delectabimur.' Quid ego, quae deinde efficiat, narrem? Eiusmodi est hominis natura, ut quae singulis diebus efficiat gloria atque ostentatione, ea vix annuo sermone enarrare possim.''


Huiusmodi notationes, quae describunt, quod consentaneum sit unius cuiusque naturae, vehementer habent magnam delectationem: totam enim naturam cuiuspiam ponunt ante oculos, aut gloriosi, ut nos exempli causa coeperamus, aut invidi aut tumidi aut avari, ambitiosi, amatoris, luxuriosi, furis, quadruplatoris; denique cuiusvis studium protrahi potest in medium tali notatione.
