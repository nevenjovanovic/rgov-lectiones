%\section*{O autoru}


\section*{Cic. de or. 1, 1-5}

Cogitanti mihi saepe numero et memoria vetera repetenti perbeati fuisse, Quinte frater, illi videri solent, qui in optima re publica, cum et honoribus et rerum gestarum gloria florerent, eum vitae cursum tenere potuerunt, ut vel in negotio sine periculo vel in otio cum dignitate esse possent; ac fuit cum mihi quoque initium requiescendi atque animum ad utriusque nostrum praeclara studia referendi fore iustum et prope ab omnibus concessum arbitrarer, si infinitus forensium rerum labor et ambitionis occupatio decursu honorum, etiam aetatis flexu constitisset.

Quam spem cogitationum et consiliorum meorum cum graves communium temporum tum varii nostri casus fefellerunt; nam qui locus quietis et tranquillitatis plenissimus fore videbatur, in eo maximae moles molestiarum et turbulentissimae tempestates exstiterunt; neque vero nobis cupientibus atque exoptantibus fructus oti datus est ad eas artis, quibus a pueris dediti fuimus, celebrandas inter nosque recolendas.

Nam prima aetate incidimus in ipsam perturbationem disciplinae veteris, et consulatu devenimus in medium rerum omnium certamen atque discrimen, et hoc tempus omne post consulatum obiecimus eis fluctibus, qui per nos a communi peste depulsi in nosmet ipsos redundarent. Sed tamen in his vel asperitatibus rerum vel angustiis temporis obsequar studiis nostris et quantum mihi vel fraus inimicorum vel causae amicorum vel res publica tribuet oti, ad scribendum potissimum conferam; tibi vero, frater, neque hortanti deero neque roganti, nam neque auctoritate quisquam apud me plus valere te potest neque voluntate.

Ac mihi repetenda est veteris cuiusdam memoriae non sane satis explicata recordatio, sed, ut arbitror, apta ad id, quod requiris, ut cognoscas quae viri omnium eloquentissimi clarissimique senserint de omni ratione dicendi.

Vis enim, ut mihi saepe dixisti, quoniam, quae pueris aut adulescentulis nobis ex commentariolis nostris incohata ac rudia exciderunt, vix sunt hac aetate digna et hoc usu, quem ex causis, quas diximus, tot tantisque consecuti sumus, aliquid eisdem de rebus politius a nobis perfectiusque proferri; solesque non numquam hac de re a me in disputationibus nostris dissentire, quod ego eruditissimorum hominum artibus eloquentiam contineri statuam, tu autem illam ab elegantia doctrinae segregandam putes et in quodam ingeni atque exercitationis genere ponendam.

(\dots)

\section*{Cic. de or. 1, 22-34}

Sed quia non dubito quin hoc plerisque immensum infinitumque videatur, et quod Graecos homines non solum ingenio et doctrina, sed etiam otio studioque abundantis partitionem iam quandam artium fecisse video neque in universo genere singulos elaborasse, sed seposuisse a ceteris dictionibus eam partem dicendi, quae in forensibus disceptationibus iudiciorum aut deliberationum versaretur, et id unum genus oratori reliquisse; non complectar in his libris amplius, quam quod huic generi re quaesita et multum disputata summorum hominum prope consensu est tributum; repetamque non ab incunabulis nostrae veteris puerilisque doctrinae quendam ordinem praeceptorum, sed ea, quae quondam accepi in nostrorum hominum eloquentissimorum et omni dignitate principum disputatione esse versata; non quo illa contemnam, quae Graeci dicendi artifices et doctores reliquerunt, sed cum illa pateant in promptuque sint omnibus, neque ea interpretatione mea aut ornatius explicari aut planius exprimi possint, dabis hanc veniam, mi frater, ut opinor, ut eorum, quibus summa dicendi laus a nostris hominibus concessa est, auctoritatem Graecis anteponam.

Cum igitur vehementius inveheretur in causam principum consul Philippus Drusique tribunatus pro senatus auctoritate susceptus infringi iam debilitarique videretur, dici mihi memini ludorum Romanorum diebus L. Crassum quasi conligendi sui causa se in Tusculanum contulisse; venisse eodem, socer eius qui fuerat, Q. Mucius dicebatur et M. Antonius, homo et consiliorum in re publica socius et summa cum Crasso familiaritate coniunctus.

Exierant autem cum ipso Crasso adulescentes et Drusi maxime familiares et in quibus magnam tum spem maiores natu dignitatis suae conlocarent, C. Cotta, qui tribunatum plebis petebat, et P. Sulpicius, qui deinceps eum magistratum petiturus putabatur.

Hi primo die de temporibus deque universa re publica, quam ob causam venerant, multum inter se usque ad extremum tempus diei conlocuti sunt; quo quidem sermone multa divinitus a tribus illis consularibus Cotta deplorata et commemorata narrabat, ut nihil incidisset postea civitati mali, quod non impendere illi tanto ante vidissent.

Eo autem omni sermone confecto, tantam in Crasso humanitatem fuisse, ut, cum lauti accubuissent, tolleretur omnis illa superioris tristitia sermonis eaque esset in homine iucunditas et tantus in loquendo lepos, ut dies inter eos curiae fuisse videretur, convivium Tusculani; postero autem die, cum illi maiores natu satis quiessent et in ambulationem ventum esset, tum Scaevolam duobus spatiis tribusve factis dixisse 'cur non imitamur, Crasse, Socratem illum, qui est in Phaedro Platonis? Nam me haec tua platanus admonuit, quae non minus ad opacandum hunc locum patulis est diffusa ramis, quam illa, cuius umbram secutus est Socrates, quae mihi videtur non tam ipsa acula, quae describitur, quam Platonis oratione crevisse, et quod ille durissimis pedibus fecit, ut se abiceret in herba atque ita [illa], quae philosophi divinitus ferunt esse dicta, loqueretur, id meis pedibus certe concedi est aequius.'

Tum Crassum 'immo vero commodius etiam'; pulvinosque poposcisse et omnis in eis sedibus, quae erant sub platano, consedisse dicebat.

Ibi, ut ex pristino sermone relaxarentur animi omnium, solebat Cotta narrare Crassum sermonem quendam de studio dicendi intulisse.

Qui cum ita esset exorsus: non sibi cohortandum Sulpicium et Cottam, sed magis utrumque conlaudandum videri, quod tantam iam essent facultatem adepti, ut non aequalibus suis solum anteponerentur, sed cum maioribus natu compararentur; 'neque vero mihi quicquam' inquit 'praestabilius videtur, quam posse dicendo tenere hominum mentis, adlicere voluntates, impellere quo velit, unde autem velit deducere: haec una res in omni libero populo maximeque in pacatis tranquillisque civitatibus praecipue semper floruit semperque dominata est.

Quid enim est aut tam admirabile, quam ex infinita multitudine hominum exsistere unum, qui id, quod omnibus natura sit datum, vel solus vel cum perpaucis facere possit? aut tam iucundum cognitu atque auditu, quam sapientibus sententiis gravibusque verbis ornata oratio et polita? aut tam potens tamque magnificum, quam populi motus, iudicum religiones, senatus gravitatem unius oratione converti? Quid tam porro regium, tam liberale, tam munificum, quam opem ferre supplicibus, excitare adflictos, dare salutem, liberare periculis, retinere homines in civitate?

Quid autem tam necessarium, quam tenere semper arma, quibus vel tectus ipse esse possis vel provocare integer vel te ulcisci lacessitus? Age vero, ne semper forum, subsellia, rostra curiamque meditere, quid esse potest in otio aut iucundius aut magis proprium humanitatis, quam sermo facetus ac nulla in re rudis? Hoc enim uno praestamus vel maxime feris, quod conloquimur inter nos et quod exprimere dicendo sensa possumus.

Quam ob rem quis hoc non iure miretur summeque in eo elaborandum esse arbitretur, ut, quo uno homines maxime bestiis praestent, in hoc hominibus ipsis antecellat? Ut vero iam ad illa summa veniamus, quae vis alia potuit aut dispersos homines unum in locum congregare aut a fera agrestique vita ad hunc humanum cultum civilemque deducere aut iam constitutis civitatibus leges iudicia iura describere?

Ac ne plura, quae sunt paene innumerabilia, consecter, comprehendam brevi: sic enim statuo, perfecti oratoris moderatione et sapientia non solum ipsius dignitatem, sed et privatorum plurimorum et universae rei publicae salutem maxime contineri. Quam ob rem pergite, ut facitis, adulescentes, atque in id studium, in quo estis, incumbite, ut et vobis honori et amicis utilitati et rei publicae emolumento esse possitis.' 

