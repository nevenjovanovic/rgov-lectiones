%\section*{O autoru}
\begin{quotation}

\noindent Natus est a. 614/140;\footnote{Cic. Brut. 43, 161, ubi correxit quod ipse novem annis ante minus recte scripserat: de or. II 99, 364: quadriennio minor est (scil. Crassus quam Antonius).} inter annos 642/112 et 646/108\footnote{Cfr. Fr. Sobeck, Die Quaestoren der römischen Republik, Bresl. 1909, p. 20.} quaestor in Asia fuit, unde Athenas venit:\footnote{Cic. de or. I 11, 45.} a. 647/107 tribunus plebis, inter annos 649/105 et 651/103 aedilis curulis, praetor a.~655/99, consul a.~659/95 cum Q.~Mucio Scaevola, quem in omnibus fere magistratibus collegam habuit: provinciam autem Galliam sortitus, cum obscuriores quasdam Alpinas gentes vicisset, triumphum frustra postulavit: censor denique fuit a.~662/92 cum Cn.~Domitio Aenobarbo, quo in magistratu rhetores Latinos edicto ex urbe sustulit.\footnote{Cic. de or. III 24, 93.} Mortuus est a.~663/91.

Antonium Crassumque oratores fuisse maximos et in his primum cum Graecorum gloria Latine dicendi copiam aequatam esse ait Cicero:\footnote{Cic. Brut. 36, 138.} uter autem maior exstiterit magnam fuisse controversiam e Ciceronis rhetoricis libris patet.

De rhetoricis suis studiis ipse Crassus apud Ciceronem\footnote{Cic. de or. I 34, 155.} loquitur: adulescentem enim summorum oratorum Graecas orationes se explicavisse, et postea, quae Graece legisset, ea Latine reddere diligentissime consuevisse: rhetorem autem Metrodorum, in Asia quaestor cum esset, libenter a se auditum.\footnote{Cic. de or. III 20, 75.} Sed et ceterarum litterarum studiis summa cura operam dedit: familiariter enim Archia poeta\footnote{Cic. pro Archia 3, 6.} et Antipatro Sidonio\footnote{Cic. de or. III 50, 194.} usus est: L. Coelium Antipatrum iuris magistrum habuit\footnote{Cic. Brut. 25, 102.} tantumque in hac doctrina profecit, ut eloquentium iurisperitissimus putaretur.\footnote{Cic. Brut. 39, 145.} Cum autem Athenas ex Macedonia venisset, summos philosophos audivit, Charmadam et Clitomachum academicos, Mnesarchum stoicum, Diodorum peripateticum.\footnote{Cic. de or. I 11, 45.}

Fuit in eo Latine loquendi accurata atque sine molestia diligens elegantia:\footnote{Cic. Brut. 38, 143; cfr. 40, 148.} multae et cum gravitate facetiae:\footnote{Cic. Brut. 43, 158: cfr. de or. II 54, 220; 56, 228.} non multa in dicendo iactatio corporis, nulla autem inambulatio neque crebra supplosio pedis.\footnote{Cic. Brut. 43, 158.}

Orationem pro M.' Curio apud centumviros, quam Cicero adeo extulit laudibus, ut eam unam se malle profiteretur, quam castellanos triumphos duo,\footnote{Cic. Brut. 73, 256.} habuit Crassus anno fere 661/93.\footnote{Post annum 659/95 eam habitam esse docet Cic. Brut. 39, 145: ab his patronis aequalibus et iam consularibus causa illa dicta est: sed e Cic. de or. I 39, 180: nuper, et pro Caecina 18, 53: paulo antequam nos in forum venimus, a.\ fere 661/93 efficiamus licet.} Quae fuerit causa his docet verbis Boethius:\footnote{Boethius, Commentaria in Topica Ciceronis, 64, 1117C (auctor 480-525).}

Ex eodem etiam similitudinis loco illa sumi Cicero proponit quae vocantur exempla, veluti Crassus in causa Curiana, quae fuit huiusmodi: Quidam praegnantem uxorem relinquens scripsit haeredem posthumum, eique alium substituit secundum, qui Curius vocabatur, ea conditione, ut si posthumus, qui intra menses decem proximos nasceretur, ante moreretur quam in suam tutelam pervenisset, idem ante obiret diem, quam testamentum iure facere posset, secundus haeres succederet; quod si ad id tempus pervenisset quo, iam firmo iudicio in suam tutelam receptus, iure civili instituto posset haerede defungi, secundus haeres, id est Curius, non succederet quae vocatur substitutio pupillaris: quaesitum est an valeret ita instituta ratio. Crassus igitur multa protulit exempla, quibus ita institutus haeres obtinuisse haereditatem, quae exemplorum commemoratio iudices movit.

Cum autem nullus natus esset filius, M.' Curius hereditatem petivit: obstitit M.~Coponius, illius uxoris propinquus. Pro Coponio dixit Q.~Mucius Scaevola, pro Curio autem Crassus, qui superior discessit. Quibus argumentis causam is defenderit, e Cicerone comperimus,\footnote{Brut. 53, 197; de or. I 57, 243; II 6, 24.} fuitque illud iudicium propter Crassi salem et leporem et politissimas facetias hilaritatis atque laetitiae plenum.\footnote{Cic. de or. I 57, 243.}

A.\ 663/91 paulo ante quam mortuus est, orationem habuit in senatu contra L.~Marcium Philippum consulem, M.~Livi Drusi trib.\ plebis acerrimum adversarium; qui Philippus in senatum vehementer in contione invectus erat iniuriasque iecerat. Crassus, e Tusculano Romam reversus, adeo graviter atque mirabiliter locutus est, senatus dignitatem a Philippi iniuriis defendens, ut a se ipso, teste Cicerone,\footnote{Cic. de or. III 1, 3.} eo die superatus esse putaretur, qui ceteros semper superasset.

\noindent Henrica Malcovati, \textit{Oratorum Romanorum fragmenta} II, Torino, 1930. Prolegomena, 75-82.
\end{quotation}

\section*{Marcus Tullius Cicero, De oratore (55 a. Chr. n), 2.24}

(Inquit Crassus:) Otium autem quod dicis esse, adsentior; verum oti fructus est non contentio animi, sed relaxatio. Saepe ex socero meo audivi, cum is diceret socerum suum Laelium semper fere cum Scipione solitum rusticari eosque incredibiliter repuerascere esse solitos, cum rus ex urbe tamquam e vinclis evolavissent. Non audeo dicere de talibus viris, sed tamen ita solet narrare Scaevola, conchas eos et umbilicos ad Caietam et ad Laurentum legere consuesse et ad omnem animi remissionem ludumque descendere. Sic enim res sese habet, ut, quem ad modum volucris videmus procreationis atque utilitatis suae causa effingere et construere nidos, easdem autem, cum aliquid effecerint, levandi laboris sui causa passim ac libere solutas opere volitare, sic nostri animi negotiis forensibus atque urbano opere defessi gestiant ac volitare cupiant vacui cura ac labore. Itaque illud ego, quod in causa Curiana Scaevolae dixi, non dixi secus ac sentiebam: nam ``si,'' inquam ``Scaevola, nullum erit testamentum recte factum, nisi quod tu scripseris, omnes ad te cives cum tabulis veniemus, omnium testamenta tu scribes unus. Quid igitur?'' inquam ``quando ages negotium publicum? quando amicorum? quando tuum? quando denique nihil ages?'' Tum illud addidi ``mihi enim liber esse non videtur, qui non aliquando nihil agit.'' In qua permaneo, Catule, sententia meque, cum huc veni, hoc ipsum nihil agere et plane cessare delectat.

\section*{Aulus Gellius (ca. 130 – post 170), Noctes Atticae 15.11}

\textit{Verba senatusconsulti de exigendis urbe Roma philosophis; item uerba edicti censorum, quo inprobati et coerciti sunt, qui disciplinam rhetoricam instituere et exercere Romae coeperant.}

\medskip


C.~Fannio Strabone M.~Valerio Messala coss.\ senatusconsultum de philosophis et de rhetoribus factum est:
\begin{quotation}
\noindent M.~Pomponius praetor senatum consuluit. Quod uerba facta sunt de philosophis et de rhetoribus, de ea re ita censuerunt, ut M.~Pomponius praetor animaduerteret curaretque, uti ei e republica fideque sua uideretur, uti Romae ne essent.
\end{quotation}

 	
Aliquot deinde annis post id senatusconsultum Cn.~Domitius Ahenobarbus et L.~Licinius Crassus censores de coercendis rhetoribus Latinis ita edixerunt:
\begin{quotation}
\noindent Renuntiatum est nobis esse homines, qui nouum genus disciplinae instituerunt, ad quos iuuentus in ludum conueniat; eos sibi nomen inposuisse Latinos rhetoras; ibi homines adulescentulos dies totos desidere. Maiores nostri, quae liberos suos discere et quos in ludos itare uellent, instituerunt. Haec noua, quae praeter consuetudinem ac morem maiorum fiunt, neque placent neque recta uidentur. Quapropter et his, qui eos ludos habent, et his, qui eo uenire consuerunt, uisum est faciundum, ut ostenderemus nostram sententiam nobis non placere.
\end{quotation}



\section*{Marcus Tullius Cicero, De Oratore (55 a. Chr. n), 3.1-2}

 	
\textbf{LIBER TERTIVS}

\medskip

\noindent Instituenti mihi, Quinte frater, eum sermonem referre et mandare huic tertio libro, quem post Antoni disputationem Crassus habuisset, acerba sane recordatio veterem animi curam molestiamque renovavit. Nam illud immortalitate dignum ingenium, illa humanitas, illa virtus L.~Crassi morte exstincta subita est vix diebus decem post eum diem, qui hoc et superiore libro continetur. Ut enim Romam rediit extremo ludorum scaenicorum die, vehementer commotus oratione ea, quae ferebatur habita esse in contione a Philippo, quem dixisse constabat videndum sibi esse aliud consilium; illo senatu se rem publicam gerere non posse, mane Idibus Septembribus et ille et senatus frequens vocatu Drusi in curiam venit; ibi cum Drusus multa de Philippo questus esset, rettulit ad senatum de illo ipso, quod in eum ordinem consul tam graviter in contione esset invectus.

Hic, ut saepe inter homines sapientissimos constare vidi, quamquam hoc Crasso, cum aliquid accuratius dixisset, semper fere contigisset, ut numquam dixisse melius putaretur, tamen omnium consensu sic esse tum iudicatum ceteros a Crasso semper omnis, illo autem die etiam ipsum a se superatum. 

Deploravit enim casum atque orbitatem senatus, cuius ordinis a consule, qui quasi parens bonus aut tutor fidelis esse deberet, tamquam ab aliquo nefario praedone diriperetur patrimonium dignitatis; neque vero esse mirandum, si, cum suis consiliis rem publicam profligasset, consilium senatus a re publica repudiaret. Hic cum homini et vehementi et diserto et in primis forti ad resistendum Philippo quasi quasdam verborum faces admovisset, non tulit ille et graviter exarsit pigneribusque ablatis Crassum instituit coercere. Quo quidem ipso in loco multa a Crasso divinitus dicta $\langle$esse$\rangle$ ferebantur, cum sibi illum consulem esse negaret, cui senator ipse non esset.

`An tu, cum omnem auctoritatem universi ordinis pro pignere putaris eamque in conspectu populi Romani concideris, me his existimas pigneribus terreri? Non tibi illa sunt caedenda, si L. Crassum vis coercere: haec tibi est incidenda lingua, qua vel evulsa spiritu ipso libidinem tuam libertas mea refutabit.' 

Permulta tum vehementissima contentione animi, ingeni, virium ab eo dicta esse constabat sententiamque eam, quam senatus frequens secutus est ornatissimis et gravissimis verbis, ut populo Romano satis fieret, numquam senatus neque consilium rei publicae neque fidem defuisse ab eo dictam et eundem, id quod in auctoritatibus perscriptis exstat, scribendo adfuisse. 

Illa tamquam cycnea fuit divini hominis vox et oratio, quam quasi exspectantes post eius interitum veniebamus in curiam, ut vestigium illud ipsum, in quo ille postremum institisset, contueremur\dots
