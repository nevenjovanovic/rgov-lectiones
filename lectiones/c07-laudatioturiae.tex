%\section*{O autoru}


\section*{Argumentum}

Haec inscriptio, scalpta circa annum 5 a.~Chr.~n, est laudatio funebris auctoris ignoti pro uxore sua. orationem Q.~Lucretii Vespillonis pro uxore Turia fuisse Theodor Mommsen putavit.

Q.\ Lucretium proscriptum a triumuiris uxor Turia inter cameram et tectum cubiculi abditum una conscia ancillula ab inminente exitio non sine magno periculo suo tutum praestitit singularique fide id egit, ut, cum ceteri proscripti in alienis et hostilibus regionibus per summos corporis et animi cruciatus uix euaderent, ille in cubiculo et in coniugis sinu salutem retineret. (Valeri Maximi factorum et dictorum memorabilium libri novem, 6.7.2 )


\section*{Laudatio Turiae}

rum [\dots] permansisti prob[a \dots] 

orbata es re[pente ante nuptiar]um diem utroque pa[rente in deserta soli]tudine una o[ccisis per te maxu]me cum ego in Macedo[niam provinciam issem] vir sororis tua[e C(aius) Cluvius in A]fricam provinciam [inulta non est relicta] mors parentum tanta cum industria m[unere es p]ietatis perfuncta ef[flagitando atque] vindicando ut si praest[o fu]issemus non ampliu[s potuissemus sed] haec habes communia cum [s]anctissima femina s[orore tua] 

quae dum agitabas ex patria domo propter custodia[m non cedisti sumpto] de nocentibus supplicio evest[i]gio te in domum ma[tris meae tulisti ubi] adventum meum expectast[i] 

temptatae deinde estis ut testamen[tum] quo nos eramus heredes rupt[um diceretur] coemptione facta cum uxore ita necessario te cum universis pat[ris bonis in] tutelam eorum qui rem agitabant reccidisse sororem omni[no eorum bonorum] fore expertem quod emancupata esset Cluvio qua mente ista acc]eperis qua prae]sentia animi restiteris etsi afui conpertum habeo veritate caussam communem [t]utata es testamentum ruptum non esse ut [uterque potius] hereditatem teneremus quam omnia bona sola possideres certa qui[dem sententia] te ita patris acta defensuram ut si non optinuisses partituram cum s[orore te adfir]mares nec sub condicionem tutelae legitumae venturam quoius per [legem in te ius non] esset neque enim familia[e] gens ulla probari poterat quae te id facere [impediret] 

nam etsi patris testamentum ruptum esset tamen iis qui intenderen[t non esse id] ius quia gentis eiusdem non essent cesserunt constantiae tuae neque amplius rem sollicitarunt quo facto [officii in patrem] pietatis in sororem fide[i] in nos patrocinium succeptum sola peregisti rara sunt tam diuturna matrimonia finita morte non divertio in[terrupta contigit] nobis ut ad annum XXXXI sine offensa perduceretur utinam vetust[a coniunctio habu]isset mutationem vice m[e]a qua iustius erat cedere fato maiorem 

domestica bona pudici[t]iae opsequi comitatis facilitatis lanificii stud[ii religionis] sine superstitione o[r]natus non conspicendi cultus modici cur [memorem cur dicam de cari]tate familiae pietate [c]um aeque matrem meam ac tuos parentes col[ueris non alia mente] illi quam tuis curaveris cetera innumerabilia habueris commun[ia cum omnibus] matronis dignam f[a]mam colentibus propria sunt tua quae vindico ac [paucae uxores in] similia inciderunt ut talia paterentur et praestarent quae rara ut essent [propitia] fortuna cavit

omne tuom patrimonium acceptum ab parentibus communi diligentia cons\-[ervavimus] neque enim erat adquirendi tibi cura quod totum mihi tradisti officia [ita par]titi sumus ut ego tu[t]elam tuae fortunae gererem ut meae custodiam sust[ineres multa] de hac parte omittam ne tua propria mecum communicem satis sit [hoc] mi[hi tuis] de sensibus [indi]casse [liberali]tatem tuam c[u]m plurumis necessariis tum praecipue pietati praesti[tisti] [\dots\ licet qu]is alias nominaverit 

unam dumtaxat simillimam [tui \dots] [\dots\ h]abuisti sororem tuam nam propinquas vestras d[ignas eiusmodi] [\dots\ bene]ficiis domibus vestris apud nos educavistis eadem u[t condicio][nes aptas famili]ae vestrae consequi possent dotes parastis quas quid[em a vobis] [constitutas comm]uni consilio ego et C(aius) Cluvius excepimus et probantes [sensus vestros] [ne vestro patrimo]nio vos multaretis nostram rem familiarem sub[didimus vestrae] [nostraque bona] in dotes dedimus quod non venditandi nostri c[aussa memoravi] [sed ut illa consi]lia vestra concepta pia liberalitate honori no[s duxisse consta][ret exequi de nos]tris [multa alia merit]a tua praetermittenda [mihi sunt 

