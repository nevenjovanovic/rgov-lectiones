%\section*{O autoru}

\begin{quotation}

Est tamen superstes unum eloquentiae Symmachianae specimen, quod argumentorum grauitate, copiis haud vulgaribus in breuitate, et summa arte, ad praestantidimorum oratorum opera accedit, Epistola (X, 54.) ad DDD.\ Valentinianum, Theodosium et Arcadium semper Augg. pro Ara Victoriae in Curiam reponenda; dicetis, in causa haud probanda eum ingenium exeruisse; at enim ex eius sensu et iudicio diiudicatio facienda est. Argumentorum dilectu, vi, pondere, aculeis, non minus admirabilis illa est, quam prudentia, cautione, ac verecundia; quam tanto magis sentias, si verbosam et inanem, interdum calumniosam et veteratoriam, declamationem Ambrosii compares. (\dots) Submota Gratiani iussu ara Victoriae e Curia, a.\ 382.\ paullo post premebatur urbs graui fame; quod Symmachus ad sacrilegam Impietatem refert locis pluribus, inprimis in nobilissima Epistola X, 54.\ supra memorata. Orta erat fames ex sterilitate anni maxime in Africa; delata est annonae cura in Symmachum Praefectum urbis ab Impp.\ dictum, classes aliis ex locis aduehi frumentum iussae; ita Africae damna Macedonicis maxime commeatibus funt compensata.

\noindent Chr. G. Heynii ... Opuscula academica collecta et animadversionibus locupletata. Volumen 6. Gottingae, 1812, 11-13.

\end{quotation}

\newpage


\section*{Relatio III De ara Victoriae (426)}

Dominis imperatoribus Valentiniano, Theodosio et Arcadio semper augustis Symmachus vir clarissimus praefectus urbi

\begin{quote}
\textit{Prooemium}
\end{quote}

Ubi primum senatus amplissimus semperque vester subiecta legibus vitia cognovit et a principibus piis vidit purgari famam temporum proximorum, boni saeculi auctoritatem secutus evomuit diu pressum dolorem atque iterum me querelarum suarum iussit esse legatum. Cui ideo divi principis denegata est ab inprobis audientia quia non erat iustitia defutura, domini imperatores Valentiniane, Theodosi et Arcadi incliti victores ac triumphatores, semper augusti.

Gemino igitur functus officio et ut praefectus vester gesta publica prosequor et ut legatus civium mandata commendo. Nulla est hic dissensio voluntatum, quia iam credere homines desierunt aulicorum se studio praestare, si discrepent. Amari, coli, diligi maius imperio est. Quis ferat obfuisse rei publicae privata certamina? Merito illos senatus insequitur, qui potentiam suam famae principis praetulerunt, noster autem labor pro clementia vestra ducit excubias. Cui enim magis commodat, quod instituta maiorum, quod patriae iura et fata defendimus quam temporum gloriae? Quae tum maior est, cum vobis contra morem parentum intellegitis nil licere.

\begin{quote}
\textit{Pars I}
\end{quote}

Repetimus igitur religionum statum, qui rei publicae diu profuit. Certe dinumerentur principes utriusque sectae, utriusque sententiae: Pars eorum prior caerimonias patrum coluit, recentior non removit. Si exemplum non facit religio veterum, faciat dissimulatio proximorum. Quis ita familiaris est barbaris, ut aram Victoriae non requirat? Cauti in posterum sumus, et aliarum rerum ostenta vitamus. Reddatur saltem nomini honor, qui numini denegatus est. Multa Victoriae debet aeternitas vestra et adhuc plura debebit. Aversentur hanc potestatem, quibus nihil profuit, vos amicum triumphis patrocinium nolite deserere! Cunctis potentia ista votiva est. Nemo colendam neget quam profitetur optandam.

(\dots)

Suus enim cuique mos, suus cuique ritus est. Varios custodes urbibus cultus mens divina distribuit. Ut animae nascentibus, ita populis fatales genii dividuntur. Accedit utilitas, quae maxime homini deos adserit. Nam cum ratio omnis in operto sit, unde rectius quam de memoria atque documentis rerum secundarum cognitio venit numinum? Iam si longa aetas auctoritatem religionibus faciat, servanda est tot saeculis fides et sequendi sunt nobis parentes, qui secuti sunt feliciter suos.

Romam nunc putemus adsistere atque his vobiscum agere sermonibus: Optimi principum, patres patriae, reveremini annos meos, in quos me pius ritus adduxit! Utar caerimoniis avitis; neque enim paenitet. Vivam meo more, quia libera sum! Hic cultus in leges meas orbem redegit, haec sacra Hannibalem a moenibus, a Capitolio Senonas reppulerunt. Ad hoc ergo servata sum, ut longaeva reprehendar?

Videro, quale sit, quod instituendum putatur; sera tamen et contumeliosa emendatio senectutis. Ergo diis patriis, diis indigetibus pacem rogamus. Aequum est, quidquid omnes colunt, unum putari. Eadem spectamus astra, commune caelum est, idem nos mundus involvit. Quid interest, qua quisque prudentia verum requirat? Uno itinere non potest perveniri ad tam grande secretum. Sed haec otiosorum disputatio est. Nunc preces, non certamina offerimus.

(\dots)

Nemo me putet tueri solam causam religionum! Ex huiusmodi facinoribus orta sunt cuncta Romani generis incommoda. Honoraverat lex parentum Vestales virgines ac ministros deorum victu modico iustisque privilegiis. Stetit muneris huius integritas usque ad degeneres trapezitas, qui ad mercedem vilium baiulorum sacra castitatis alimenta verterunt. Secuta est hoc factum fames publica et spem provinciarum omnium messis aegra decepit.

Non sunt haec vitia terrarum. Nihil inputemus austris! Nec rubigo segetibus obfuit nec avena fruges necavit: Sacrilegio annus exaruit. Necesse enim fuit perire omnibus, quod religionibus negabatur. Certe si est huius mali aliquod exemplum, inputemus tantam famem vicibus annorum: Gravis hanc sterilitatem causa contraxit. Silvestribus arbustis vita producitur et rursus ad Dodonaeas arbores plebis rusticae inopia convolavit.

Quid tale provinciae pertulerunt, cum religionum ministros honor publicus pasceret? Quando in usum hominum concussa quercus, quando vulsae sunt herbarum radices, quando alternos regionum defectus deseruit fecunditas mutua, cum populo et virginibus sacris communis esset annona? Commendabat enim terrarum proventum victus antistitum et remedium magis quam largitas erat. An dubium est semper pro copia omnium datum, quod nunc inopia omnium vindicavit?

